\documentclass{beamer}

\usepackage{fontspec}
\usepackage{polyglossia}

\usepackage{graphicx}
\usepackage{color}
\usepackage{url}

\usepackage[normalem]{ulem}

\usepackage{pifont}
 \newcommand{\hand}{\ding{43}}

\mode<presentation>
{
    \usetheme{HZSK}
}


\title{Workflows for Developing Finnish–German Shallow RBMT...
\scriptsize{in FinMT, Helsinki, 2017-11-01\\
\url{https://github.com/flammie/apertium-fin-deu/}}}
\author{Tommi A Pirinen \scriptsize \guilsinglleft tommi.antero.pirinen@uni-hamburg.de \guilsinglright }
\institute{HZSK.de, de.CLARIN.eu, etc.}
\date{\today}

\begin{document}

\selectlanguage{english}

\maketitle

\begin{frame}
    \frametitle{Introduction}
    \begin{itemize}
        \item A professional comp.linguist moving to Germany with ``no prior
            knowledge'' of the language
        \item A field/documentary linguistics-heavy North Eurasian language
            project (SIL FLeX!)
        \item Prior experience of FSAs, RBMT, SMT\ldots
        \item \hand RBMT+FSA+language learning+language documentation =
            cool new workflow for creating: corpora + mono- and bilingual
            lexicons, chunkers and more\textbf{!!}
    \end{itemize}
\end{frame}

\begin{frame}
    \frametitle{Language documentation workflow}
    Field linguists / documentary linguists are used to SIL tools like
    FLeX (toolbox) with workflows like:
    \begin{enumerate}
        \item split words into morphs
        \item tag morphs
        \item translate lemmas
        \item repeat
    \end{enumerate}
    Results in a dictionary, annotated corpus and some kind of ``morphological
    analyser''.
\end{frame}

\begin{frame}
    \frametitle{Demo FLeX here (Finnish, simplified)}
    \begin{tiny}\url{https://www.hs.fi/nyt/art-2000005425359.html}\end{tiny}
        Elisa
        julkais\only<2->{--}\alert<2-3>{i}
        meksikolais\only<4->{--}\alert<4-5>{i}\only<4-5>{--}\alert<4-5>{lla}
        stereotypio\only<5->{--}\alert<5>{i}\only<5>{--}\alert<5>{lla}
        ratsasta\only<6->{--}\alert<6-7>{ne}\only<6->{--}\alert<6-7>{en}
        mainokse\only<8->{--}\alert<8-9>{n}
        ,
        joka
        suututt\only<3->{--}\alert<3>{i}
        jopa
        suurlähettilää\only<9->{--}\alert<9>{n}
        –
        Nyt
        mainos
        on
        poistet\only<10->{--}\alert<10>{tu}
        ,
        eikä
        yhtiö
        halua
        enää
        edes
        puhua
        sii\only<11->{--}\alert<11>{tä}
        ..
        Teleyhtiö
        Elisa
        julkais\only<3->{--}\alert<3>{i}
        viime
        viiko\only<9->{--}\alert<9>{n}
        lopu\only<12->{--}\alert<12>{ssa}
        jatko\only<13->{--}\alert<13>{a}
        suositu\only<14->{--}\alert<14>{lle}
        Hintasaarnaaja
        -
        mainossarja\only<14->{--}\alert<14>{lle}\only<7->{--}\alert<7>{en}
        ,
        mutta
        vain
        muutama\only<9->{--}\alert<9>{n}
        päivä\only<9->{--}\alert<9>{n}
        jälke\only<7->{--}\alert<7>{en}
        mainos
        poistet\only<15->{--}\alert<15>{tiin}
        kaikkialta
        .
        \only<16>{\\
        ... after only 15 minutes of clicking and stuff you get like,
        past tense, 10 case allomorphs. passive voice, and a dozen of lemma
        allomorphs}
\end{frame}

\begin{frame}
    \frametitle{Real FLeX}
    \includegraphics[width=\textwidth]{flexpic}
\end{frame}


\begin{frame}
    \frametitle{RBMT workflow (apertium style)}
    \begin{enumerate}
        \item (try to) Translate a text
        \item Add OOVs to source language dictionary, repeat 1
        \item Add missing word-translations to bilingual dictionary, repeat 1
        \item Add ungenerateable words to target language dictionary, repeat 1
        \item Mangle mismatching lexical grammar
        \item (Advanced RBMT not relevant to this presentation, magic)
    \end{enumerate}
    Results in two morphological analyser/generators, a bilingual dictionary and
    a shallow comparative grammar!
\end{frame}

\begin{frame}
    \frametitle{RBMT workflow demo}
    \only<1>{
    Am S-Bahnhof Berliner Tor hat es gleich drei Einsätze der Bundespolizei
    gegeben.  Donnerstagabend war dort ein stark angetrunkener Mann auf die
    Gleise gefallen und musste von den Beamten gerettet werden. Freitag um halb
    acht lief ein ebenfalls angetrunkener Mann auf den Gleisen Richtung
    Hauptbahnhof.  Und gegen zehn warf ein Mann den Deckel eines Mülleimers auf
    die Gleise. In allen drei Fällen musste die Strecke \alert<1>{kurzzeitig} gesperrt
    werden.}
\only<2>{
    Päälle \alert<2>{*S-asema} berliiniläinen portilla on se samoin kolme käytöt liittovaltionpoliisin annettu.
    torstai-ilta oli siellä yksi vahvasti \alert<2>{*angetrunkener} mies
    \alert<2>{@Gleis<n><pl><ine>} miellyttää ja täytyi virkailijoista pelastetaan.
    perjantai jotta puolittainen kahdeksan saapui myös \alert<2>{*angetrunkener} mies
    \alert<2>{@Gleis<n><pl><nom>} suunnassa \alert<2>{@Hauptbahnhof<n><sg><nom>}.
    Ja vastaan kymmenen pudotti mies \alert<2>{@Deckel<n><sg><gen>}
    \alert<2>{@Mülleimer<n><sg><gen>} \alert<2>{@Gleis<n><pl><ine>}.
    kaiken kolme tapauksien/tapausten täytyi reitti lyhytaikainen estetään.
}
\only<3>{
    \alert<3>{\sout{Päälle} @S-Bahnhof<n><sg><nom>} berliiniläiseltä portilla on se samoin kolme käytöt liittovaltionpoliisin annettu.
    torstai-ilta oli siellä yksi vahvasti
    \alert<3>{@angetrunken<adj><pos><sg><nom>} mies @Gleis<n><pl><ine> miellyttää ja täytyi virkailijoista pelastetaan.
    perjantai jotta puolittainen kahdeksan saapui myös
    \alert<3>{@angetrunken<adj><pos><sg><nom>} mies @Gleis<n><pl><nom> suunnassa @Hauptbahnhof<n><sg><nom>.
Ja vastaan kymmenen pudotti mies @Deckel<n><sg><gen> @Mülleimer<n><sg><gen> @Gleis<n><pl><ine>.
 kaiken kolme tapauksien/tapausten täytyi reitti lyhytaikainen estetään.
}
\only<4>{
    \alert<4>{\#lähijuna-asema<n><sg><nom>} berliiniläiseltä portilla on se samoin kolme käytöt liittovaltionpoliisin annettu.
torstai-ilta oli siellä yksi vahvasti humalainen mies raiteissa miellyttää ja täytyi virkailijoista pelastetaan.
perjantai jotta puolittainen kahdeksan saapui myös humalainen mies raiteet
    suunnassa \alert<4>{\#päärautatieasema<n><sg><nom>}.
Ja vastaan kymmenen pudotti mies kannen roskiksen raiteissa.

}
\only<5>{
        lähijuna-asema berliiniläiseltä portilla on se samoin kolme käytöt liittovaltionpoliisin annettu.
torstai-ilta oli siellä yksi vahvasti humalainen mies raiteissa miellyttää ja täytyi virkailijoista pelastetaan.
perjantai jotta puolittainen kahdeksan saapui myös humalainen mies raiteet suunnassa päärautatieasema.
Ja vastaan kymmenen pudotti mies kannen roskiksen raiteissa.
 kaiken kolme tapauksien/tapausten täytyi reitti lyhytaikainen estetään.
}
\end{frame}


\begin{frame}
    \frametitle{Experimental ``results''}
    \begin{itemize}
        \item 1.5 years after starting at week-daily average effort of 15
            minutes
        \item approx. 10,000 translation pairs, few hundreds of new lexemes for
            analysers, few dozens of rules for WSD, chunking, lex.sel., etc.
        \item Post-edition WER is about 40, BLEU probably 8 or so,
        \item ...  but it's easy to read and understand!
        \item For realistic projects, e.g. within Uralic group, it should be
            perfectly plausible
    \end{itemize}
    Let's go: \scriptsize{\url{http://39476.s.time4vps.cloud}}
\end{frame}

\end{document}

