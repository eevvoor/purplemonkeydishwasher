\documentclass{beamer}

\usepackage{fontspec}
\usepackage{polyglossia}

\usepackage{graphicx}
\usepackage{color}
\usepackage{url}

\usepackage[normalem]{ulem}

\usepackage{pifont}
 \newcommand{\hand}{\ding{43}}

\mode<presentation>

\title{Towards Karelian UD treebanks---workflow prototype,
\scriptsize{at SyntaxFest, Paris, 2019-08-28\\
\url{https://github.com/apertium/apertium-krl/}}}
\author{Tommi A Pirinen \scriptsize \guilsinglleft flammie@iki.fi \guilsinglright }
\date{\today}

\begin{document}

\selectlanguage{english}

\maketitle

\begin{frame}
    \frametitle{Intro: what is this}
    \begin{itemize}
        \item Workflow prototype
        \item No-Resource / seriously under-resourced languages:\begin{itemize}
                \item Some 1000 lines in a Web-crawl corpus,
                \item an XML dictionary
                \item Karelian proper: no analysers or NL stuff, Livvi: some
                    basic analysers
        \end{itemize}
        \item we want to make dependency treebanking part of the kickstarting
            of language resources including analysers, dictionaries, 
    \end{itemize}
\end{frame}

\begin{frame}
    \frametitle{Karelian languages}
    \parbox{.6\textwidth}{%
    \begin{itemize}
        \item Uralic, Finnic languages
        \item Agglutinative, more morphological complexity than most IE, less
            than Eskimo-Aleut
        \item Closely related to Finnish, limited mutual intelligibility
            for experts etc.
    \end{itemize}
    }\hfill\parbox{.4\textwidth}{%
        \includegraphics[width=.33\textwidth]{KarelianMapPlusJoensuu}
    }
\end{frame}


\begin{frame}
    \frametitle{The workflow}
    \begin{enumerate}
        \item Analyse a text
        \item Add OOVs to source language dictionary
        \item Generate all plausible dependency tree suggestions
        \item Disambiguate and fix by hand
    \end{enumerate}
    Also:
    \begin{itemize}
        \item Annotate by hand / translate
    \end{itemize}
    can be interleaved for corpus building and fun
\end{frame}

\begin{frame}
    \frametitle{workflow}
    Parts of workflow are feeding each other:
    \begin{enumerate}
        \item Existing analysers and translators are used to generate
            suggestions for new translations and annotations
        \item Manual translations and annotations form a gold corpora / parallel
            texts / treebanks
        \item Manually translated / annotated lexemes are used to
            generate entries for mono- and bilingual dictionaries
        \item in future we would like to record gold annotations to improve
            rule-based disambiguation
    \end{enumerate}
\end{frame}

\begin{frame}
    \frametitle{RBMT workflow demo}
    Source text:\\

    III Eläkeläisien talvispartakiadi piettih Petroskoissa
\\
Kaunehena pakkaispäivänä 6. kevätkuuta Petroskoin Kurgan-urheilukeškukšešša keräyty yli šata ijäkäštä hiihtourheilun harraštajua ympäri Karjalašta.
Kilpailuja piettih kahteh jakšoh – kilpahiihot naisien ta miehien kešen šekä šekaviestihiihto.
Urheilupruasniekan tulokšet jätettih tyytyväisiksi kuin tapahtuman järještäjie, šamoin ni ošallistujie
\end{frame}

\begin{frame}
    \frametitle{Analysis with source OOVs}
    \alert{*III *Eläkeläisien *talvispartakiadi}
    piteä.VERB Petroskoi.PROPN

    \alert{*Kaunehena  *pakkaispäivänä} 6.ADJ kevätkuu.NOUN Petroskoi.PROPN
    \alert{*Kurgan}-urheilu.NOUNkeškuš.NOUN *keräyty yli.ADP šata.NUM ijäkäštä.ADJ
    \alert{*hiihtourheilun} \alert{*harraštajua} ympäri.ADP Karjala.PROPN ..PUNCT
\end{frame}

\begin{frame}
    \frametitle{Analysis (100 \% coverage)}
III.ADJ Eläkeläine.NOUN talvispartakiadi.NOUN
    piteä.VERB Petroskoi.PROPN

    Kaunis.ADJ  pakkaispäivä.NOUN 6.ADJ kevätkuu.NOUN Petroskoi.PROPN
    Kurgan-urheilukeškuš.NOUN keräytyö.VERB yli.ADP šata.NUM ijäkäštä.ADJ
    hiihtourheilu.NOUN harraštaja.NOUN ympäri.ADP Karjala.PROPN ..PUNCT
\end{frame}

\begin{frame}
    \frametitle{Dependency suggestions (redacted visualisation)}
    \begin{small}
    \begin{texttt}
\# sent-id: vepkar-1698.1 \\
\# text: III Eläkeläisien talvispartakiadi piettih Petroskoissa. \\
1  III  III  ADJ  ADJ  Case=Nom|Number=Sing|NumType=Ord  \_  \_  \_  Weight=500.0 \\
2  Eläkeläisien  eläkeläine  NOUN  NOUN  Case=Gen|Number=Plur  3  nmod:poss  \_  Weight=0.01 \\
3  talvispartakiadi  talvispartakiadi  NOUN  NOUN  Case=Nom|Number=Sing  \_  \_  \_  Weight=500.0 \\
        \alert{4}  piettih  piteä  VERB  VERB  Mood=Ind|Number=Plur|Person=3|Tense=Past|VerbForm=Fin|Voice=Act  0  root  \_  Weight=0.01 \\
        \alert{4}  piettih  piteä  VERB  VERB  Mood=Ind|Tense=Past|VerbForm=Fin|Voice=Pass  0  root  \_  Weight=0.21000000000000002 \\
5  Petroskoissa  Petroskoi  PROPN  PROPN  Case=Ine|Number=Sing  4  obl  \_  Weight=0.005 \\
6  .  .  PUNCT  PUNCT  \_  4  punct  \_  Weight=0.01
    \end{texttt}
    \end{small}
\end{frame}

\begin{frame}
    \frametitle{Disambiguate and fix}
    \begin{small}
    \begin{texttt}
\# sent-id: vepkar-1698.1 \\
\# text: III Eläkeläisien talvispartakiadi piettih Petroskoissa. \\
1  III  III  ADJ  ADJ  Case=Nom|Number=Sing|NumType=Ord  3  amod  \_  Weight=500.0 \\
2  Eläkeläisien  eläkeläine  NOUN  NOUN  Case=Gen|Number=Plur  3  nmod:poss  \_  Weight=0.01 \\
3  talvispartakiadi  talvispartakiadi  NOUN  NOUN  Case=Nom|Number=Sing  4  nsubj  \_  Weight=500.0 \\
4  piettih  piteä  VERB  VERB  Mood=Ind|Tense=Past|VerbForm=Fin|Voice=Pass  0  root  \_  Weight=0.21000000000000002 \\
5  Petroskoissa  Petroskoi  PROPN  PROPN  Case=Ine|Number=Sing  4  obl  \_  Weight=0.005 \\
6  .  .  PUNCT  PUNCT  \_  4  punct  \_  Weight=0.01
    \end{texttt}
    \end{small}
\end{frame}


\begin{frame}
    \frametitle{Experimental ``results''}
    In about a year I have created some resources:
\begin{enumerate}
        \item Universal Dependency treebank for Karelian
        \item Karelian morphological analyser
        \item Finnish-Karelian RBMT (limited vocabuliaries)
    \end{enumerate}
\end{frame}

\begin{frame}
    \frametitle{Future???}
    I want to make GUI that does most of the thing like so, the idea is
    loosely based on documentary linguistics tool SIL FLEX language explorer
    \includegraphics{flex}
\end{frame}

\begin{frame}
    \frametitle{Thanks, questions?}
    Thanks, questions?
\end{frame}

\end{document}

