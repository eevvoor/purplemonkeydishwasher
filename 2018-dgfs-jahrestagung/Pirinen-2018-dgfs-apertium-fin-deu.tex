\documentclass{flammie}
%\documentclass{dgfs2018}

\begin{document}

    \selectlanguage{English}

%%    \title[RBMT for fin-deu]
    \title{Rule-based machine-translation between Finnish and German}

%%    \numberofauthors{1}
%%    \shortauthor{Pirinen}
%%    \firstauthorfirstname[]{Tommi}%
%%    \firstauthorlastname[]{Pirinen}%
%%    \firstaff[]{Universit\"{a}t Hamburg}%
%%    \firstmail[]{tommi.antero.pirinen@ uni-hamburg.de}%
    \author{Tommi A Pirinen}

%% ========================================================================
%% day, date, time, room are specified by the local organizers!
%%\def\talkday{}
%%\def\talkdate{}
%%\def\talktime{}
%%\def\talkroom{}

\maketitle

\begin{abstract}
\noindent With this poster I present a work-in-progress rule-based machine
translation between German and Finnish based on the
Apertium\footnote{\url{http://wiki.apertium.org/}} machine translation
system (Forcada et al. 2011). The system is composed of canonical NLP
components for rule-based systems: morphological analysis, chunking, lexical
selection, chunk re-ordering and structural translation and morphological
generation. One of the points that I want to highlight in this poster is the
workflow and the supporting infrastructure for it; unlike a typical coursework or
research project, this machine translation has been modeled as a \textit{lexicon
and grammar engineering while language learning} type of project. On software
engineering side I have developed tools to extend mono- and bilingual lexicons
while learning the OOV words (notably, words that are OOV for RBMT are new words
for the language learner) in texts, and I am in process to extend these tools
for the grammar learning---strucutural transfer interaction. I have also
modernised the build infrastructure from sf.net SVN to github with full support
of continuous integration and automatic testing, providing an excellent platform
for language learners to extend the lexicons and grammars without fear of
breaking other existing systems that depend on these lexicons and grammars.

Apertium systems are modular pipelines combining basic NLP tools
with machine translation specific modules. The three lexical modules
that are the most important for the system development and language learning are
the morpholgical analysers and the lexical translation, which
correspond the vocabulary of the learner / MT system. The resources for Finnish
and German morphological analysis were available at the start of the project,
but to our knowledge, this is the first free and open source Finnish-German
bilingual resource of its kind. For the other parts of the pipeline that are
more specific to apertium, such as chunking (shallow syntax parsing),
re-ordering and transfer rules.
\end{abstract}

\begin{itemize}
\item    Forcada, M.L. et al. (2011): Apertium: a free/open-source platform for
    rule-based machine translation. \emph{Machine translation} 25(2). 127--144.
\end{itemize}


%\printindex
\end{document}
