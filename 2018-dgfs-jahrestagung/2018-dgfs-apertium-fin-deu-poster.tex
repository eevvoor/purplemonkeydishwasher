\documentclass[final,hyperref={pdfpagelabels}]{beamer}

\usepackage{fontspec}
\usepackage[absolute,overlay]{textpos}

\usepackage{amsmath,amsthm, amssymb, latexsym}
\boldmath

\usepackage{array,booktabs,tabularx}

\usepackage[orientation=portrait,size=a0,scale=1.4]{beamerposter}

\title[Apertium-fin-deu]{{\huge Rule-based machine translation between
Finnish and German}\\
\url{https://github.com/flammie/apertium-fin-deu/}}
\author[tommi.pirinen@uni-hamburg.de]{Tommi A Pirinen$^\star$}
\institute[UHH]{$^\star$ Universität Hamburg---Hamburger Zentrum für Sprachkorpora}
\date{\today}

%\logo{\includegraphics[height=7.5cm]{AbumatranLogo}}

\newlength{\columnheight}
\setlength{\columnheight}{105cm}
\renewcommand<>{\texttt}{\only#1{\beameroriginal{\texttt}}}

\begin{document}

\begin{frame}
      \maketitle
      \vfill
      \begin{columns}
      \begin{column}{.49\textwidth}
      \begin{beamercolorbox}[center,wd=\textwidth]{postercolumn}
          \begin{minipage}[T]{.95\textwidth}  % tweaks the width, makes a new \textwidth
          \parbox[t][\columnheight]{\textwidth}{ % must be some better way to set t
        \begin{block}{Rule-based machine translation}
            Apertium system is built from modules that pass data in pipelines:
            \begin{enumerate}
                \item Morphological analysis
                \item Morphological disambiguation
                \item Lexical translation
                \item Chunking
                \item Re-ordering
                \item Generation
            \end{enumerate}
            There can be more or less modules in any given system but these
            are the main blocks of Finnish---German at the time of writing.
            The morphological analysis and generation are implemented with
            a dictionary-based finite-state morphological parser and the
            translation is also basically just a bilingual dictionary, other
            components are based on rules.
        \end{block}

        \begin{block}{Statistics on dictionaries and rules}
            Dictionaries:
            \begin{tabular}{|l|r|}
                \hline
                \bf Languages & \bf Words \\
                Finnish & 426,947 \\
                German$^\star$ & 96,043\\
                Finnish---German & 10,937 \\
                \hline
            \end{tabular}
            \\
            Finnish-to-German rulesets:
            \begin{tabular}{|l|r|}
                \hline
                \bf Module & \bf Rules \\
                Disambiguation & 1,261 \\
                Lexical selection & 3 \\
                Chunking & 4 \\
                Re-ordering & 2\\
                \hline
            \end{tabular}
            \\
            German-to-Finnish rulesets:
            \begin{tabular}{|l|r|}
                \hline
                \bf Module & \bf Rules \\
                Disambiguation$^\star$ & 81 \\
                Lexical selection & 4 \\
                Chunking & 132 \\
                Re-ordering & 7 \\
                \hline
            \end{tabular}
            $^\star$ German dictionary and rules were written by developers
            of Apertium's German language modules.
        \end{block}


        \begin{block}{Example translations}
            From \url{http://www.dw.com/de/05032018-langsam-gesprochene-nachrichten/a-42828347}:
            \begin{tabular}{|r|l|}
                \hline
                German text & Bei der Parlamentswahl \\
                Morphological analysis & bei ADP der (ART or PRON) Parlamentswahl NOUN \\
                disambiguation & bei ADP der ART Parlamentswahl NOUN\\
                Translation & bei=luona der=>0 Parlamentswahl=Parlamentinvaalit \\
                Chunking & (luona 0 Parlamentinvaalit) NP PL ADE \\
                Generation & Parlamentinvaaleilla \\
                \hline
            \end{tabular}
            \begin{tabular}{|r|l|}
                \hline
                German text & In Italien \\
                Morphological analysis & in ADP Italien PROPN \\
                Translation & in=>0 Italien=>Italia \\
                Chunking & (0 Italia) NP SG INE \\
                Generation & Italiassa \\
                \hline
            \end{tabular}
            \begin{tabular}{|r|l|}
                \hline
                German text & sind [...] rechte Parteien \\
                Morphological analysis &  sein VERB, rechte ADJ / recht NOUN Partei NOUN\\
                Disambiguation &  sein VERB, rechte ADJ Partei NOUN\\
                Translation & sein=>olla rechte=>oikea Partei=puolue \\
                Generation & ovat oikeat puolueet \\
                \hline
            \end{tabular}
            \begin{tabular}{|r|l|}
                \hline
                German text & die große Gewinner \\
                Morphological analysis &  der PRON/DET, groß ADJ, Gewinner NOUN\\
                Disambiguation &  der PRON gros ADJ Gewinner NOUN\\
                Translation & der=>tämä groß=>suuri Gewinner=>voittaja \\
                Generation & nämä suuret voittajat \\
                \hline
            \end{tabular}
        \end{block}

        \begin{block}{References}
            Source code and dictionary datas:
            \begin{itemize}
                \item \url{https://github.com/flammie/apertium-fin-deu}
                \item \url{https://github.com/flammie/apertium-fin}
                \item \url{https://github.com/apertium/apertium-deu} (to appear)
            \end{itemize}
            Engines:
            \begin{itemize}
                \item \url{https://sf.net/p/apertium}
                \item \url{https://github.com/hfst}
            \end{itemize}
            Publications:
            \begin{itemize}
                \item \url{https://flammie.github.io/purplemonkeydishwasher}
            \end{itemize}
            Online demo (beta):
            \begin{itemize}
                \item \url{http://39476.s.time4vps.cloud}
            \end{itemize}
        \end{block}

         }
        \end{minipage}
      \end{beamercolorbox}
  \end{column}
    %\end{textblock}

    %\begin{textblock}{60}{0.1,1.6}
   \begin{column}{.49\textwidth}
      \begin{beamercolorbox}[center,wd=\textwidth]{postercolumn}
        \begin{minipage}[T]{.95\textwidth} % tweaks the width, makes a new \textwidth
          \parbox[t][\columnheight]{\textwidth}{ % must be some better way to set the the height, width and textwidth simultaneously

              \begin{block}{Code snippets}
            Whole dictionary and visualisations can be found on the website
            \url{https://flammie.github.io/apertium-fin-deu}.
              \end{block}{Code snippets}

            \begin{block}{German dictionary (apertium monodix)}
                \texttt{
                <section id="main"/> \\
... \\
  <e lm="Hauptzweig"><i>Hauptzweig</i><par n="Aal\_\_n\_m"/></e> \\
  <e lm="Haus"><i>H</i><par n="H/aus\_\_n\_nt"/></e> \\
  <e lm="Hausangestellte"><i>Hausangestellte</i><par n="Angestellte\_\_n\_f"/></e> \\
... \\
</section> \\
<pardef n="H/aus\_\_n\_nt"> \\
  <e><p><l>äusern</l><r>aus<s n="n"/><s n="nt"/><s n="pl"/><s n="dat"/></r></p></e> \\
  <e><p><l>äuser</l><r>aus<s n="n"/><s n="nt"/><s n="pl"/><s n="acc"/></r></p></e> \\
  <e>       <p><l>aus</l>       <r>aus<s n="n"/><s n="nt"/><s n="sg"/><s n="nom"/></r></p></e> \\
  ... \\
</pardef>}
            \end{block}

            \begin{block}{Finnish Dictionary (Finite-State Morphology lexc)}
            \texttt{
LEXICON NOUN \\
 \\
talniflumaatti\%<n\%>:talniflumaat        NOUN\_KORTTI     ; \\
talo\%<n\%>:talo  NOUN\_TALO       ; \\
talojohto\%<n\%>:talo{hyph?}joh   NOUN\_VETO       ; \\
 \\
LEXICON NOUN\_VETO \\
:do     NOUN\_BACK\_WEAK\_SINGULARS ; \\
:do     NOUN\_BACK\_WEAK\_PLURALS ; \\
:do     NOUN\_COMPOUND\_N ; \\
:to     NOUN\_BACK\_NOMINATIVE ; \\
            }
            \end{block}

            \begin{block}{Bilingual dictionary (apertium bidix)}
            \texttt{
  <section id="main" type="standard"> \\
    <e><p><l>talli<s n="n"/></l><r>Stall<s n="n"/><s n="m"/></r></p \\
></e> \\
    <e><p><l>talo<s n="n"/></l><r>Haus<s n="n"/><s n="nt"/></r></p></e> \\
    <e><p><l>talouskasvu<s n="n"/></l><r>Wirtschaftswachstum<s n="n"/><s n="nt"/></r></p></e> \\
    </section> \\
            }
            \end{block}

            \begin{block}{German disambiguation (Constraint Grammar)}
            \texttt{
LIST DetDef = (det def) ; \\
LIST EOS = (<<<) (sent) ; \\
REMOVE DetDef IF (1 EOS) ; \\
SELECT (vblex p3 sg) IF (-1 (prn pers p3 sg)) ; \\
            }
            \end{block}
            \begin{block}{German-to-Finnish rules (visualisations in pseudocode)}
            \texttt{
if sl[1]['lem'] is "in" and sl[1]['a\_case'] is <dat> then: \\
  let \$adpcase = <ine> \\
elseif sl[1]['lem'] is "ab" and sl[1]['a\_case'] is <dat> then: \\
  let \$adpcase = <abl> \\
elseif sl[1]['lem'] is "über" then: \\
  let \$adpcase = <ela> \\
...
            }

            \end{block}

        \begin{block}{Acknowledgements}
        The Apertium Finnish and German dictionaries have been collected and
            extended by numerous students and contributors.
      \end{block}
    %\end{textblock}
         }
          % ---------------------------------------------------------%
          % end the column
        \end{minipage}
      \end{beamercolorbox}
    \end{column}
  \end{columns}

  \end{frame}
\end{document}
