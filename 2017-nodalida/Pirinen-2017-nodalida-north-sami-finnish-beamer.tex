\PassOptionsToPackage{pdfpagelabels=false}{hyperref} 
\documentclass[xetex]{beamer}
%\documentclass[hyperref={pdfpagelabels=false}]{beamer}
\usepackage{polyglossia} % 25pt
\usepackage{fontspec}

\usepackage{mathtools}
\usepackage{graphicx}
\usepackage{multirow}

\usetheme{Montpellier} %Montpellier
\usecolortheme{beaver}
\usepackage{color}
\definecolor{DarkBlue}{rgb}{0.1,0.1,0.5}
%\definecolor{Green}{rgb}{0.1,0.7,0.6}
\definecolor{Green}{RGB}{34,139,34}
\definecolor{Red}{rgb}{0.9,0.0,0.1}
\definecolor{harmaa}{cmyk}{0.3,0.1,0.1,0.4}
\definecolor{ao}{rgb}{0.36, 0.54, 0.66}
\usepackage{linguex}

\begin{document}
\title{North Sámi to Finnish rule-based machine translation system}

% Comment out for review
\author{Ryan Johnson \and Tommi A Pirinen \and Tiina Puolakainen \\ Francis Tyers \and Trond Trosterud \and Kevin Unhammer \\ 
%UiT Norgga \'Arktala\v{s} universitehta, Giela ja kultuvrra instituhtta, Romssa, Norga \\
%Universit\"{a}t Hamburg, Hamburger Zentrum f\"{u}r Sprachkorpora, Deutschland \\
%Institute of the Estonian Language, Estonia
%ryan.txanson@gmail.com, tommi.antero.pirinen@uni-hamburg.de, \\ tiina.puolakainen@eki.ee,
%\{francis.tyers, trond.trosterud\}@uit.no \\
%kevin@unhammer.org
}


\date{}

\frame{\titlepage}

\frame{\frametitle{Outline of the talk}\tableofcontents} 


\section{Introduction}
\frame{\frametitle{Introduction}
\begin{itemize}
\item MT system between Finnish and North Sámi
\item The system is bidirectional, we present North Sámi -  Finnish.
\end{itemize}
}

\section{Previous work on Uralic MT}
\frame{\frametitle{Previous work}
\begin{itemize}
\item North Sámi - Lule Sámi
\begin{itemize}
\item  Early work (Tyers, Wiechetek and Trosterud 2009)
\end{itemize}
\item North Sámi -  South Sámi
\begin{itemize}
\item  Test version (Antonsen and Trosterud 2013)
\end{itemize}
\item North Sámi -  Norwegian
\begin{itemize}
\item  Early system  + work on measuring  quality  of gisting (Unhammer and Trosterud 2013)
\end{itemize}
\item North Sámi to other Sámi (talk later today)
\end{itemize}
}

\frame{\frametitle{Published MT systems on Uralic}
\begin{itemize}
\item North Sámi to Norwegian: \url{https://jorgal.uit.no/}
\item Between Estonian, Finnish and Hungarian via English as a pivot language (Google Translate).
\end{itemize}
}



\section{The languages: North Sámi and Finnish}


\frame{\frametitle{The languages}

\begin{itemize}
\item  Not mutually understandable ( $\approx$ Russian - Lithuanian?)
\item  But easy to learn the other
\end{itemize}
}


\frame{\frametitle{The languages}

\begin{itemize}
\item North Sámi:
\begin{itemize}
\item 24.700 speakers (SIL), in Norway, Sweden and Finland 
\item Some regional status
\end{itemize}
\item Finnish:
\begin{itemize}
\item 6 million speakers
\item  Official state language
\end{itemize}
\end{itemize}
}






\frame{\frametitle{Phonology}


\begin{itemize}
\item Similar processes
\begin{itemize}
\item Consonant gradation
\item Stem vowel changes
\item Initial stress
\end{itemize} \pause
\item Differences
\begin{itemize}
\item North Sámi consonant gradation involves the vast majority of stem-internal consonant clusters, whereas the Finnish counterpart involves only the stops \textit{p, t, k}
\item Finnish vowel length is independent of stress and consonant length, and distinctive
\item North Sámi vowel length is rarely distinctive, and linked to syllable structure and consonant length
\item North Sámi final vowel loss and \textit{p, t, k} neutralisation has given rise to extensive morphological homonymy
\end{itemize}
\end{itemize}
}




\frame{\frametitle{Orthography}

\begin{itemize}
\item Orthographically, Finnish and North Sámi follow different principles
\item Finnish \textit{p, t, k}, North Sámi  \textit{b, d, g} \\ \textit{kirja : girji} ``book''. 
\item Vowel length
\begin{itemize}
\item Finnish: double letter symbols. 
\item North Sámi Length marked for $a$ / $á$
\end{itemize}
\item Apart from this the orthographic principles of the two languages is quite similar, 
\item the almost total lack of free rides is a result of different phonology.
\end{itemize}
}



\frame{\frametitle{Morphology: Verbs}


\begin{itemize}
\item North Sámi has a separate dual number, whereas Finnish has not.  
\item The finite verb inflection is almost identical. \pause
\item The infinite  verb conjugation is more different, though
\begin{itemize}
\item Finnish has more, and inflecting, participles
\item Finnish has more infinitives
\end{itemize}
\end{itemize}
}


\frame{\frametitle{Morphology}


\begin{itemize}
\item Cases
\begin{itemize}
\item  North Sámi has local case IN and FROM Finnish has a 2x3 system of local cases inner/outer to/in/from cases,
\item North Sámi has one case for the direct object (accusative), Finnish has three  (nominative, accusative and partitive).
\end{itemize}
\item Finnish and North Sámi have the same system of possessive suffixes, but in  North Sámi its use is  far more restricted than in Finnish.
\end{itemize}
}




\frame{\frametitle{Syntax}

\begin{itemize}
\item Syntactically speaking, there are two varieties of North Sámi
\begin{itemize}
\item one ´´Scandinavian''
\begin{itemize}
\item with relative clauses, verb particles, and genitive pronouns
\end{itemize}
\item one ´´Finnish''
\begin{itemize}
\item with infinite particible/infinitive construction, verb derivation and possessive suffixes
\end{itemize}
\end{itemize}
\item Translating ``Scandinavian'' Sámi into Finnish will give the output a colloquial flavour
\end{itemize}
}

\frame{\frametitle{Agreement}

\begin{itemize}
\item Finnish adjectives agree with their head noun in case and number
\item  North Sámi has an invariant $attribute$ form for all but one adjective, the adjective \textit{buorre} `good'
\end{itemize}
}


\frame{\frametitle{Habitive constructions and E-sentences}

\begin{itemize}
\item Existential and habitive clauses have the same structure in the two languages
\begin{itemize}
\item  \textit{possessor.local-case copula possessed} and \textit{adverbial copula e-subject} (\textit{on me / in street is car} \\
`I have a car/There is a car in the street')
\end{itemize}
\item The e-subject / possessed behaves like...
\begin{itemize}
\item an object in Finnish (carrying object case)
\item a subject in North Sámi (agreeing with the verb)
\end{itemize}
\end{itemize}
}




\section{The MT system}


\frame{\frametitle{Use of the North Sámi - Finnish MT system}

\begin{itemize}
\item The MT system fulfils different functions within and outside Finland \pause
\begin{itemize}
\item In Finland
\begin{itemize}
\item it works as a gisting systems for Finns wanting to understand North Sámi
\item ... especially North Sámi written in Norway
\end{itemize}
\item In Norway and Sweden
\begin{itemize}
\item it may be used by North Sámi speakers to understand Finnish text. 
\end{itemize}
\end{itemize} \pause
\item In principle, the system may also be used for North Sámi to Finnish text production
\begin{itemize}
\item ... but the use cases are few, and the quality still way too low
\end{itemize}
\end{itemize}
}




\frame{\frametitle{The Apertium system}

\begin{itemize}
\item The Apertium machine translation platform
\item originally aimed at the Romance languages of the Iberian peninsula
\item has also been adapted for other, more distantly related, language pairs.
\item Licensed under General Public Licence (GPL) and all the software and data is available for download from the project website.
\end{itemize}
}

\subsection{Architecture of the system}

\frame{\frametitle{Architecture of the system}


The Apertium translation engine consists of a Unix-style pipeline or assembly line with the following modules
}

\frame{\frametitle{}

\begin{figure}
\includegraphics[height=.92\textheight]{Apertium-structure}
\caption{Apertium structure (Image from apertium wiki by user Rcrowther)
\url{http://wiki.apertium.org/wiki/Workflow_diagram}
\label{fig:apertium}}
\end{figure}
}

\frame{\frametitle{Apertium pipeline}
\begin{itemize}
\item \textbf{Deformatter:} encapsulates format information of the input \pause
\item \textbf{Morphological analyser:} gives lexical forms(s) (LF) consisting of lemma + analysis. \pause
\item \textbf{Morphological disambiguator (CG):} chooses the most adequate morphological analysis for an ambiguous sentence. \pause
\item \textbf{Lexical transfer module:} Looks up each LF in the bilingual dictionary and translates it \pause
\item \textbf{Lexical selection:} chooses the most adequate translation of ambiguous source language LFs.  \pause
\item \textbf{Structural transfer:} performs local syntactic operations on LF patterns. \pause
\item \textbf{Morphological generator:} generates a TL SF for each TL LF \pause
\item \textbf{Reformatter:} de-encapsulates any format information.
\end{itemize}
}


\frame{\frametitle{Example}


Translation process for the North Sámi phrase:

\textit{Sámegielat leat gielat maid sámit hállet} \\

(The Sámi languages are the languages that the Sámis speak)
}

\frame{\frametitle{}
\begin{table}
\begin{tiny}
\begin{tabular}{|l|}
\hline
\^{}Sámegielat/sámegielat<adj><attr>/sámegielat<adj><sg><nom>/ \\
sámegiella<n><pl><nom>/ sámegiella<n><sg><acc><px2sg>/ \\
sámegiella<n><sg><gen><px2sg>/ sámegiella<n> <sg><acc><px2sg>/ \\
sámegiella<n><sg><gen><px2sg>\$ \\ 
\\
\^{}leat/leat<vblex><iv><indic><pres><conneg>/ \\
leat<vblex><iv><indic><pres><p1><pl>/leat<vblex><iv><indic><pres><p2><sg>/ \\ 
leat<vblex><iv><indic><pres><p3><pl>/leat<vblex><iv><inf>\$ \\
\\
\^{}gielat/giella<n><pl><nom>/giella<n><sg><acc><px2sg>/ \\
giella<n><sg><gen><px2sg>/giella<n><sg><acc><px2sg>/\\ 
giella<n><sg><gen><px2sg>\$ \\
\\
\^{}maid/maid<adv>/mii<prn><itg><pl><acc>/ \\
mii<prn><itg><pl><gen>/mii<prn><itg><sg><acc>/ mii<prn><rel><pl><acc>/ \\
mii<prn><rel><pl><gen>/mii<prn><rel><sg><acc>\$ \\
\\
\^{}sámit/sápmi<n><pl><nom>/sápmi<n><pl><nom>\$ \\
\\
\^{}hállet/hállat<vblex><tv><imp><p2><pl>/ \\
hállat<vblex><tv><indic><pres><p3><pl>/hállat<vblex><tv><indic><pret><p2><sg>\$ \\
\^{}./.<sent>\$
\\
\hline 
\end{tabular}
%\caption{Translation process for the North Sámi phrase \\
%\textit{Sámegielat leat gielat maid sámit hállet} \\
%(The Sámi languages are the languages that the Sámis speak)
%\label{table:translation1}}
\end{tiny}
\end{table}
}


\frame{\frametitle{}
\begin{table}
\begin{small}
\begin{tabular}{|l|}

\hline
\^{}Sámegielat/sámegiella<n><pl><nom><@SUBJ$\rightarrow$>\$ \\
\\
\^{}leat/leat<vblex><iv><indic><pres><p3><pl> <@+FMAINV>\$ \\
\\
\^{}gielat/giella<n><pl><nom> <@$\leftarrow$SPRED>\$ \\
\\
\^{}maid/mii<prn><rel><pl><acc> <@OBJ$\rightarrow$>\$ \\
\\
\^{}sámit/sápmi<n><pl><nom><@SUBJ$\rightarrow$>\$ \\
\\
\^{}hállet/hállat<vblex><tv><indic><pres><p3><pl><@+FMAINV>\$ \\
\^{}./.<sent>\$
\\
\hline 
\end{tabular}
\caption{Morphological disambiguation
\label{table:translation2}}
\end{small}
\end{table}
}





\frame{\frametitle{}
\begin{table}
\begin{small}
\begin{tabular}{|l|}
\hline
\^{}Sámegiella<n><pl><nom><@SUBJ$\rightarrow$>/ \\
Saamekieli<n><pl><nom><@SUBJ$\rightarrow$>/Saame<n><pl><nom><@SUBJ$\rightarrow$>\$ \\
\\
\^{}leat<vblex><iv> <indic><pres><p3><pl><@+FMAINV>/ \\
olla<vblex> <actv><indic><pres><p3><pl><@+FMAINV>/ \\
sijaita<vblex> <actv><indic><pres><p3><pl><@+FMAINV>\$ \\
\\
\^{}giella<n> <pl><nom> <@$\leftarrow$SPRED>/kieli<n><pl><nom> <@$\leftarrow$SPRED>/ \\ 
ansa<n><pl><nom><@$\leftarrow$SPRED>\$ \\
\\
\^{}mii<prn><rel><pl><acc><@OBJ$\rightarrow$>/mikä<prn><rel><pl><par><@OBJ$\rightarrow$>\$ \\
\\
\^{}sápmi<n><pl><nom><@SUBJ$\rightarrow$>/saamelainen<n><pl><nom> <@SUBJ$\rightarrow$>\$ \\
\\
\^{}hállat<vblex><tv><indic><pres><p3><pl><@+FMAINV>/ \\
puhua<vblex><actv><indic><pres><p3><pl><@+FMAINV>/ \\
mekastaa<vblex><actv><indic><pres><p3><pl><@+FMAINV>\$\^{}.<sent>/.<sent>\$
\\
\hline
\end{tabular}
\caption{Lexical translation
\label{table:translation3}}
\end{small}
\end{table}
}



\frame{\frametitle{}
\begin{table}
\begin{small}
\begin{tabular}{|l|}
\hline
\bf North Sámi input: \\
Sámegielat leat gielat maid sámit hállet. \\
\hline
\hline
\^{}Saamekieli<n><pl><nom>\$ \\
\^{}olla<vblex><actv><indic><pres><p3><pl>\$ \\
\^{}kieli<n><pl><nom>\$ \\
\^{}mik\"{a}<prn><rel><pl><par>\$ \\
\^{}saamelainen<n><pl><nom>\$ \\
\^{}puhua<vblex><actv><indic><pres><p3><pl>\$\^{}.<sent>\$
\\
\hline 
\end{tabular}
\caption{Structural transfer
\label{table:translation4}}
\end{small}
\end{table}

\begin{table}
\begin{small}
\begin{tabular}{|l|}
\hline
Saamekielet ovat kielet mit\"{a} saamelaiset puhuvat \\
\hline 
\end{tabular}
\caption{Finnish translation
\label{table:translation}}
\end{small}
\end{table}
}

\subsection{Morphological transducers}

\frame{\frametitle{Morphological transducers}

\begin{itemize}
\item transducers are compiled with HFST -- a free/open-source  reimplementation of the Xerox finite-state tool-chain \pause
\item It implements lexc for stems and concatenative morphology, and the twolc and xfst scripting languages for modeling morphophonological rules. \pause
\item The morphologies of both languages are implemented in lexc, and the morphophonologies of both languages are implemented in twolc.
\end{itemize}
}

\frame{\frametitle{Analysis vs. generation}
\begin{itemize}
\item The same morphological description is used for both analysis and generation. \pause
\item To avoid overgeneration, any alternative forms are marked with one of two marks, LR (only analyser) or RL (only generator). \pause
\item Instead of the usual compile/invert to compile the transducers, we compile twice, once the generator, without the LR paths, and then again the analyser without the RL paths.
\end{itemize}
}

\subsection{Bilingual lexicon}

\frame{\frametitle{Bilingual lexicon}

\begin{itemize}
\item Bilingual lexicon: 19,415 stem-to-stem correspondences (half = proper nouns)
\item it was built upon
\begin{itemize}
\item an available North Sámi - Finnish dictionary, 
\item supplemented by manual work
\item The proper nouns were taken from existing lexical resources.
\end{itemize}
\end{itemize}
}


\subsection{Disambiguation rules}

\frame{\frametitle{Disambiguation rules}

\begin{itemize}
\item Morphological disambiguation module (Constraint Grammar)
\item Output of each morphological analyser is ambiguous
\begin{itemize}
\item 2.4 morphological analyses per form for Finnish
\item 2.6 morphological analyses per form for North Sámi
\end{itemize}
\item The goal of the CG rules is to select the correct analysis when there are multiple analyses.
\end{itemize}
}

\frame{\frametitle{Similarity between Finnish and North Sámi}

\begin{itemize}
\item ambiguity across parts of speech may often be passed from one SL to TL
\item Disambiguating between forms within the inflectional paradigms in case of homonymy, on the other hand, are crucial for choosing the correct form of the target language
\item CG rules to resolve ambiguity for North Sámi
\begin{itemize}
\item Ambiguity is down to 1.08 for North Sámi
\item ... and to 1.36 for Finnish
\end{itemize}
\end{itemize}
}

\section{Evaluation}

\frame{\frametitle{Examples}
{\tiny (from Yle Sapmi 2017-05-23)}\\
Sámediggi nuoraidráđđi lea ožžon 15000 euro sturrosaš ruhtadeami dán jahkái DigiÁrran-prošektii. Nuoraidčálli Kati Eriken muitala, ahte digitálalaš nuoraidbargu oaivvilda nuoraidbarggu mii geavaha ávkin digitálalaš reaidduid.\\
\vfill
\alert<3>{Saamelaiskäräjät} nuorisoneuvosto on saanut \alert<3>{15000:ta} euron suuruisen rahoituksen \alert<3>{tähän vuoteen} \alert<1>{*DigiÁrran}-projektiin. Nuorisosihteeri Kati \alert<1>{*Eriken} kertoo, että \alert<2>{*digitálalaš} \alert<4>{nuorienteos} tarkoittaa \alert<4>{nuorienteosta} mikä käyttää etuna \alert<2>{*digitálalaš} työkaluja.
\vfill
\only<1>{Missing Proper nouns (sometimes do not count as error)}
\only<2>{Missing other words}
\only<3>{Wrong inflections}
\only<4>{re-compounding oddities}
\only<5>{Word-Error Rate: 0.32}
}

\frame{\frametitle{Coverage}



\begin{table}
\begin{tabular}{|l|r|r|r|}
\hline
\bf Corpus & \bf Tokens & \bf Cov. & \bf std \\
\hline
se.wikipedia.org & 190,894 & 76,81 \%  & $\pm$10\\  % updated numberw for in toto
New Testament    & 162,718 & 92,45 \% & $\pm$0.06 \\  % updated numberw for in toto
\hline 
\end{tabular}
\caption{Na\"{i}ve coverage of sme-fin system
\label{table:coverage}}
\end{table}

}


\frame{\frametitle{OOV WER}

\begin{table}
\begin{tabular}{|l|r|r|r|}
\hline
\bf Corpus & \bf Tokens & \bf OOV & \bf WER \\
\hline
Redigering.se & 1,070 & 95 & 34.24 \\
Samediggi.fi & 570 & 33 & 36.32 \\
The Story & 361 & 0 & 19.94 \\
\hline
\end{tabular}
\caption{Word error rate over the corpora; OOV is the number of out-of-vocabulary (unknown) words.
\label{table:wer}}
\end{table}
}

\frame{\frametitle{Error Analysis}
Problematic errors:
\begin{itemize}
\item Adposition <-> case:
	\begin{itemize}
	\item (sme) ``birra''  -> (fin) ``ymp\"ari'' (around), should be:
    \item (sme) -> (fin) ``seassa'', should be:
	\end{itemize}
\item Possessive congruence:
	\begin{itemize}
	\item (sme) ``''  -> (fin) ``heidän äiti'' (around), should be: ``heidän äitinsä''
	\end{itemize}
\item Lexical:
	\begin{itemize}
	\item  ``hallintoalue'' (governmental area), ``seurantavastuu'' (responsibility of surveillance), ``itsehallinto'' (autonomy),
	\end{itemize}
\item Others: \begin{itemize}
	\item  ``mukana'' (with) was corrected to ``mukaan'' (according to). 
	\end{itemize}
\end{itemize}
can be fixed easily
}


\frame{\frametitle{Error Analysis}
Smaller problems:
\begin{itemize}
\item object case selection:
	\begin{itemize}
	\item (sme) `` Máret geahččá \textit{eatnis}''  -> (fin)``Mari katsoo \textit{äitinsä}'' should be ``...äitiänsä'' 
    \item (sme) ``''  -> (fin) ``hän näkee \textit{pientä kättäkin}'' should be ``...pienen kädenkin'' 
	\end{itemize}
\item Possessive congruence:
	\begin{itemize}
	\item (sme) ``''  -> (fin) ``heidän äiti'' (around), should be: ``heidän äitinsä''
	\end{itemize}
\item Lexical:
	\begin{itemize}
	\item  ``hallintoalue'' (governmental area), ``seurantavastuu'' (responsibility of surveillance), ``itsehallinto'' (autonomy),
	\end{itemize}
\item Others: \begin{itemize}
	\item  ``mukana'' (with) was corrected to ``mukaan'' (according to). 
	\end{itemize}
\end{itemize}

}


\section{Conclusion}

\frame{\frametitle{Conclusion}

\begin{itemize}
\item We have presented the first MT system from Finnish to North Sámi.
\item It is still a prototype
\begin{itemize}
\item With a WER of above 30\%
\item and few rules.
\end{itemize}
\item the outlook is promising 
\item a high quality translation between morphologically-rich agglutinative languages is possible.
\item We plan to continue development on the pair; 
\item the coverage of the system is already quite high, 
 we intend to increase it to 95~\% on the corpora we have
\item we estimate that this will mean adding around 5,000 new stems and take 1–2 months.
\item The remaining work will be improving the quality of translation by adding more rules, starting with the transfer component.
\item The long-term plan is to integrate the data created with other open-source data for Uralic languages in order to make transfer systems between all the Uralic language pairs.
\item Related work is currently ongoing from North Sámi to South, Lule and Inari Sámi, from North  Sámi to Norwegian, and between Finnish and Estonian.
\item The system presented here is available as free/open-source software under the GNU GPL and the whole system may be downloaded from Sourceforge and the open repository of Giellatekno.
\end{itemize}
}


\frame{\frametitle{Acknowledgements}

The work on this North Sámi-Finnish machine translation system was partially funded by the Google Summer of Code and Google Code-In programmes, and partly by a Norsk forskingsråd grant (234299) on machine translation between Sámi languages.
}
\end{document}


