\documentclass{beamer}

\usepackage{fontspec}
\usepackage{polyglossia}

\usepackage{graphicx}
\usepackage{color}
\usepackage{url}

\usepackage[normalem]{ulem}

\usepackage{pifont}
 \newcommand{\hand}{\ding{43}}

\mode<presentation>

\title{Workflows for Developing Finnish–Karelian Shallow RBMT,
\scriptsize{in LoResMT, Dublin, 2019-08-20\\
\url{https://github.com/flammie/apertium-fin-deu/}}}
\author{Tommi A Pirinen \scriptsize \guilsinglleft tommi.antero.pirinen@uni-hamburg.de \guilsinglright }
\institute{Universität Hamburg}
\date{\today}

\begin{document}

\selectlanguage{english}

\maketitle

\begin{frame}
    \frametitle{Introduction}
    \begin{itemize}
        \item Closely related languages in Uralic tree: Finnish---Karelian
        \item ``Low-Resource''?\begin{itemize}
                \item Finnish: all NLP tools, dictionaries, few Web-corpora and
                    crawls, treebanks, \ldots (gets called low-resource
                    sometimes)
                \item Karelian: 1000 lines in a Web-crawl corpus,
                    an XML dictionary (= \textbf{low}-resource)
                \item Probably no parallel or comparable texts\ldots
        \end{itemize}
        \item \hand RBMT+FSA+language documentation =
            cool new workflow for creating: corpora + mono- and bilingual
            lexicons, chunkers, treebanks etc.\textbf{!!}
        \item the workflow has been successfully tested with a number of
            Finnish-X language pairs
    \end{itemize}
\end{frame}

\begin{frame}
    \frametitle{Finnish and Karelian}
    \parbox{.6\textwidth}{%
    \begin{itemize}
        \item Agglutinative, more morphological complexity than most IE, less
            than Eskimo-Aleut
        \item Limited mutual intelligibility for non-experts: e.g., few more
            phonemes, mismatched semantic case suffixes, lexical differences
    \end{itemize}
    }\hfill\parbox{.4\textwidth}{%
        \includegraphics[width=.33\textwidth]{KarelianMapPlusJoensuu}
    }
\end{frame}


\begin{frame}
    \frametitle{RBMT workflow}
    \begin{enumerate}
        \item Machine-translate a text (analyse)
        \item Add OOVs to source language dictionary, repeat 1
        \item Add missing word-translations to bilingual dictionary, repeat 1
        \item Add ungenerateable words to target language dictionary, repeat 1
    \end{enumerate}
    Also:
    \begin{itemize}
        \item Translate by hand / post-edit (disamgibuate)
    \end{itemize}
    can be interleaved
\end{frame}

\begin{frame}
    \frametitle{RBMT workflow}
    Parts of workflow are feeding each other:
    \begin{enumerate}
        \item Existing analysers and translators are used to generate
            suggestions for new translations and annotations
        \item Manual translations and annotations form a gold corpora / parallel
            texts / treebanks
        \item Manually translated / annotated lexemes are used to
            generate entries for mono- and bilingual dictionaries
        \item in future we would like to record gold annotations to improve
            rule-based disambiguation
    \end{enumerate}
\end{frame}

\begin{frame}
    \frametitle{RBMT workflow demo}
    \only<1>{Source text}
    \only<2>{Analysis with source OOVs}
    \only<3>{Analysis (100 \% coverage)}
    \\
        \only<1>{III Eläkeläisien talvispartakiadi piettih Petroskoissa

Kaunehena pakkaispäivänä 6. kevätkuuta Petroskoin Kurgan-urheilukeškukšešša keräyty yli šata ijäkäštä hiihtourheilun harraštajua ympäri Karjalašta.
Kilpailuja piettih kahteh jakšoh – kilpahiihot naisien ta miehien kešen šekä šekaviestihiihto.
Urheilupruasniekan tulokšet jätettih tyytyväisiksi kuin tapahtuman järještäjie, šamoin ni ošallistujie
    }
    \only<2>{\alert<2>{*III *Eläkeläisien *talvispartakiadi}
    piteä.VERB Petroskoi.PROPN

    \alert<2>{*Kaunehena  *pakkaispäivänä} 6.ADJ kevätkuu.NOUN Petroskoi.PROPN
    \alert<2>{*Kurgan}-urheilu.NOUN#keškuš.NOUN *keräyty yli.ADP šata.NUM ijäkäštä.ADJ
    \alert<2>{*hiihtourheilun} \alert<2>{*harraštajua} ympäri.ADP Karjala.PROPN ..PUNCT

    }
\only<3>{
    Päälle \alert<2>{*S-asema} berliiniläinen portilla on se samoin kolme käytöt liittovaltionpoliisin annettu.
    torstai-ilta oli siellä yksi vahvasti \alert<2>{*angetrunkener} mies
    \alert<2>{@Gleis<n><pl><ine>} miellyttää ja täytyi virkailijoista pelastetaan.
    perjantai jotta puolittainen kahdeksan saapui myös \alert<2>{*angetrunkener} mies
    \alert<2>{@Gleis<n><pl><nom>} suunnassa \alert<2>{@Hauptbahnhof<n><sg><nom>}.
    Ja vastaan kymmenen pudotti mies \alert<2>{@Deckel<n><sg><gen>}
    \alert<2>{@Mülleimer<n><sg><gen>} \alert<2>{@Gleis<n><pl><ine>}.
    kaiken kolme tapauksien/tapausten täytyi reitti lyhytaikainen estetään.
}
\only<3>{
    \alert<3>{\sout{Päälle} @S-Bahnhof<n><sg><nom>} berliiniläiseltä portilla on se samoin kolme käytöt liittovaltionpoliisin annettu.
    torstai-ilta oli siellä yksi vahvasti
    \alert<3>{@angetrunken<adj><pos><sg><nom>} mies @Gleis<n><pl><ine> miellyttää ja täytyi virkailijoista pelastetaan.
    perjantai jotta puolittainen kahdeksan saapui myös
    \alert<3>{@angetrunken<adj><pos><sg><nom>} mies @Gleis<n><pl><nom> suunnassa @Hauptbahnhof<n><sg><nom>.
Ja vastaan kymmenen pudotti mies @Deckel<n><sg><gen> @Mülleimer<n><sg><gen> @Gleis<n><pl><ine>.
 kaiken kolme tapauksien/tapausten täytyi reitti lyhytaikainen estetään.
}
\only<4>{
    \alert<4>{\#lähijuna-asema<n><sg><nom>} berliiniläiseltä portilla on se samoin kolme käytöt liittovaltionpoliisin annettu.
torstai-ilta oli siellä yksi vahvasti humalainen mies raiteissa miellyttää ja täytyi virkailijoista pelastetaan.
perjantai jotta puolittainen kahdeksan saapui myös humalainen mies raiteet
    suunnassa \alert<4>{\#päärautatieasema<n><sg><nom>}.
Ja vastaan kymmenen pudotti mies kannen roskiksen raiteissa.

}
\only<5>{
        lähijuna-asema berliiniläiseltä portilla on se samoin kolme käytöt liittovaltionpoliisin annettu.
torstai-ilta oli siellä yksi vahvasti humalainen mies raiteissa miellyttää ja täytyi virkailijoista pelastetaan.
perjantai jotta puolittainen kahdeksan saapui myös humalainen mies raiteet suunnassa päärautatieasema.
Ja vastaan kymmenen pudotti mies kannen roskiksen raiteissa.
 kaiken kolme tapauksien/tapausten täytyi reitti lyhytaikainen estetään.
}
\end{frame}


\begin{frame}
    \frametitle{Experimental ``results''}
    In about a year I have created some resources:
    \begin{enumerate}
        \item Karelian morphological analyser
        \item Finnish-Karelian RBMT (limited vocabuliaries)
        \item Universal Dependency treebank for Karelian
        \item \ldots
    \end{enumerate}
    A demo that may or may not run correctly at the moment:
    \scriptsize{\url{https://flammie.name}}
\end{frame}

\end{document}

