\documentclass[10pt,xetex]{beamer} %[compress, blue]

\usepackage{fontspec}
\usepackage{polyglossia}

\usepackage{graphicx}
\usepackage{color}
\usepackage{url}

\usepackage{pifont}
\newcommand{\hand}{\ding{43}}


\title{Installing Apertium and tools\\
\scriptsize{ESU-DH in Leipzig, 2018-07}}
\author{Tommi A Pirinen \scriptsize \guilsinglleft
tommi.antero.pirinen@uni-hamburg.de
\guilsinglright}
\institute{Hamburger Zentrum für Sprachkorpora, CLARIN-D}
\date{\today}

\begin{document}

\begin{frame}
        \titlepage
\end{frame}



\begin{frame}
  \frametitle{Introduction to Apertium}

  This is a short intro on the system while we get it installed and
    projects started if possible

\begin{itemize}
    \item installation instructions
  \item The history of the project
  \item The architecture of the system
  \item How the community development process works
\end{itemize}

\end{frame}

%%%%%%%%%%%%%%%%%%%%%%%%%%%%%%%%%%%%%%%%%%%%%%%%%%%%%%%%%%%%%%%%%%%%%%%%%%%%%%%%%%%%%%%
%% history
%%%%%%%%%%%%%%%%%%%%%%%%%%%%%%%%%%%%%%%%%%%%%%%%%%%%%%%%%%%%%%%%%%%%%%%%%%%%%%%%%%%%%%%

\begin{frame}
    \begin{itemize}
        \item http://wiki.apertium.org/Installing
    \end{itemize}
\end{frame}


\begin{frame}
  \frametitle{History}

  \begin{itemize}
  \item October 2004: The Spanish Ministry of Industry, funds a consortium to
      build Free Open Source MT for the languages of Spain:
     \begin{itemize}
     \item Universities: EHU, UA, UPC, UVigo
     \item Companies: Eleka, Elhuyar, Imaxin Software
     \end{itemize}
   \item Project develops two systems:
     \begin{itemize}
     \item Apertium (Spanish---Catalan, Spanish---Galician)
     \item Matxin (Spanish$\rightarrow$Basque)
     \end{itemize}
 \end{itemize}

% apertium built from traductor universia / internostrum
\end{frame}

%%%%%%%%%%%%%%%%%%%%%%%%%%%%%%%%%%%%%%%%%%%%%%%%%%%%%%%%%%%%%%%%%%%%%%%%%%%%%%%%%%%%%%%
%% philosophy
%%%%%%%%%%%%%%%%%%%%%%%%%%%%%%%%%%%%%%%%%%%%%%%%%%%%%%%%%%%%%%%%%%%%%%%%%%%%%%%%%%%%%%%

\begin{frame} %% framesection
 \begin{center}
 {\Large {\bf Philosophy}}
 \end{center}
\end{frame}


\begin{frame}
 \frametitle{Philosophy} 

\begin{itemize} 
  \item Build on top of word-by-word translation
  \item Use a pipeline of independent programs 
  \item Separate engine and data
  \item Support regional, minority and lesser-used languages
\end{itemize}

\end{frame}

\begin{frame}
  \frametitle{Build on top of word-by-word translation}

\begin{itemize}
  \item With closely-related languages, word-for-word translation can often get you 80\% of the way
  \item We should be able to build on top of this with:
  \begin{itemize}
    \item Robust lexical processing (e.g.\ tokenisation and multiword units)
    \item Lexical category disambiguation (e.g.\ part-of-speech tagging)
    \item Local structural processing (e.g.\ for agreement, reordering)
  \end{itemize}
  \item to get post-editable quality translations
  \item For less-related pairs, we should be able to use a similar model to get
      translations suitable for assimilation/gisting
\end{itemize} 



\end{frame}

\begin{frame}
  \frametitle{Use a pipeline of independent programs}

Instead of building a monolithic application, write simple programs which
communicate using text interfaces (e.g. Unix pipes).

\begin{itemize}
  \item Simplifies debugging, 
  \item Makes extending the platform easier, just ``drop in'' a new module, or
      a replacement module.
  \item Modules developed for Apertium can be used for other natural language
      processing (NLP) tasks
\end{itemize}

\end{frame}

\begin{frame}
  \frametitle{Separate engine and data}

It should be possible to have a single translation engine, reading data for different
language pairs.

  \begin{itemize}
    \item No programming knowledge necessary
    \item A system should be able to be built just by writing/describing linguistic data
    \begin{itemize}
      \item Dictionaries
      \item Tagging rules
      \item Transfer rules
    \end{itemize}
    \item The data should be written in an interoperable format (= XML)
    \begin{itemize}
      \item \ldots although this may not be the case for external programs
    \end{itemize}
      
  \end{itemize}

\end{frame}

\begin{frame}
  \frametitle{Minority, marginalised and regional languages}

As a project, Apertium is particularly interested in supporting minority, marginalised
lesser-resourced and regional languages,

Why ?  For some altruistic reasons, 

  \begin{itemize}
    \item Larger, more well-developed/well-funded languages already have several commercial offerings 
    \item We are firm believers in linguistic diversity
    \item Most of the developers speak one or more marginalised languages to some degree
    \item Why shouldn't everyone have access to machine translation ?
  \end{itemize}

and for some less-than altruistic ones,

  \begin{itemize}
    \item Being the first to do something has some benefits from a research point of view --- publications, etc.
    \item Easy to be the best in the field, when you're the only ;D
  \end{itemize}

\end{frame}

%%%%%%%%%%%%%%%%%%%%%%%%%%%%%%%%%%%%%%%%%%%%%%%%%%%%%%%%%%%%%%%%%%%%%%%%%%%%%%%%%%%%%%%
%% growth by year 
%%%%%%%%%%%%%%%%%%%%%%%%%%%%%%%%%%%%%%%%%%%%%%%%%%%%%%%%%%%%%%%%%%%%%%%%%%%%%%%%%%%%%%%

\begin{frame} %% framesection
 \begin{center}
 {\Large {\bf Project growth}}
 \end{center}
\end{frame}

\begin{frame}
  \frametitle{Project growth: 2005}
\parbox{0.8\textwidth}{\includegraphics[width=0.8\textwidth]{growth/2005.pdf}}\parbox{0.2\textwidth}{{\bf 2005}}

\end{frame}
\begin{frame}
  \frametitle{Project growth: 2006}
\parbox{0.8\textwidth}{\includegraphics[width=0.8\textwidth]{growth/2006.pdf}}\parbox{0.2\textwidth}{{\bf 2006}}

\end{frame}
\begin{frame}
  \frametitle{Project growth: 2007}
\parbox{0.8\textwidth}{\includegraphics[width=0.8\textwidth]{growth/2007.pdf}}\parbox{0.2\textwidth}{{\bf 2007}}

\end{frame}
\begin{frame}
  \frametitle{Project growth: 2008}
\parbox{0.8\textwidth}{\includegraphics[width=0.8\textwidth]{growth/2008.pdf}}\parbox{0.2\textwidth}{{\bf 2008}}

\end{frame}
\begin{frame}
  \frametitle{Project growth: 2009}
\parbox{0.8\textwidth}{\includegraphics[width=0.8\textwidth]{growth/2009.pdf}}\parbox{0.2\textwidth}{{\bf 2009}}

\end{frame}
\begin{frame}
  \frametitle{Project growth: 2010}
\parbox{0.8\textwidth}{\includegraphics[width=0.8\textwidth]{growth/2010.pdf}}\parbox{0.2\textwidth}{{\bf 2010}}

\end{frame}
\begin{frame}
  \frametitle{Project growth: 2011}
\parbox{0.8\textwidth}{\includegraphics[width=0.8\textwidth]{growth/2011.pdf}}\parbox{0.2\textwidth}{{\bf 2011}}

\end{frame}

\begin{frame}
  \frametitle{Project growth: 2012}
\parbox{0.8\textwidth}{\includegraphics[width=0.8\textwidth]{growth/2012.pdf}}\parbox{0.2\textwidth}{{\bf 2012}}

\end{frame}

\begin{frame}
  \frametitle{Project growth: 2013}
\parbox{0.8\textwidth}{\includegraphics[width=0.8\textwidth]{growth/2013.pdf}}\parbox{0.2\textwidth}{{\bf 2013}}

\end{frame}

\begin{frame}
  \frametitle{Project growth: 2014}
\parbox{0.8\textwidth}{\includegraphics[width=0.8\textwidth]{growth/2014.pdf}}\parbox{0.2\textwidth}{{\bf 2014}}

\end{frame}

%%%%%%%%%%%%%%%%%%%%%%%%%%%%%%%%%%%%%%%%%%%%%%%%%%%%%%%%%%%%%%%%%%%%%%%%%%%%%%%%%%%%%%%
%% architecture
%%%%%%%%%%%%%%%%%%%%%%%%%%%%%%%%%%%%%%%%%%%%%%%%%%%%%%%%%%%%%%%%%%%%%%%%%%%%%%%%%%%%%%%

\begin{frame} %% framesection
 \begin{center}
 {\Large {\bf Usage}}
 \end{center}
\end{frame}

\begin{frame}
   \frametitle{Major users}

   \begin{itemize}
     \item \emph{La Voz de Galicia}
     \item \emph{La Generalitat de Catalunya}
     \item \emph{Ofis Publik ar Brezhoneg}
     \item \emph{WikiMedia}
     \item \emph{Oslo School District}
   \end{itemize}

% la voz de galicia (spa-glg)
% generalitat de catalunya (oci-cat, oci-spa)
% ofis publik ar brezhoneg (bre-fra)
% wikimedia (spa-cat)
% oslo school district (nno-nob) \smiley

\end{frame}


\begin{frame}
    \frametitle{Online users}
    \begin{center}
      \includegraphics[width=0.87\textwidth]{graphics/usage-stats-may-2014-map.png}
    \end{center}


\end{frame}

\begin{frame}
    \frametitle{Online translation statistics}
    \begin{center}
      \includegraphics[width=0.8\textwidth]{graphics/usage-stats-may-2014.png}
    \end{center}

\end{frame}


%%%%%%%%%%%%%%%%%%%%%%%%%%%%%%%%%%%%%%%%%%%%%%%%%%%%%%%%%%%%%%%%%%%%%%%%%%%%%%%%%%%%%%%
%% community
%%%%%%%%%%%%%%%%%%%%%%%%%%%%%%%%%%%%%%%%%%%%%%%%%%%%%%%%%%%%%%%%%%%%%%%%%%%%%%%%%%%%%%%

\begin{frame} %% framesection
 \begin{center}
 {\Large {\bf Development community}}%\cyrtext{Project growth}}} % Development community
 \end{center}
\end{frame}


\begin{frame}
  \frametitle{Traditional development models} % Traditional development models

Apertium has a wide range of development styles, but the one we try and encourage
is the community development style. But before explaining how that works, let's recap
the typical development types, 

\begin{itemize}
  \item {\bf Commercial}: A company makes a system, and sells licences for it in the usual fashion.
  \item {\bf Big research}: A big consortium or group develops a system, and charges for its use. Sometimes with public money, sometimes without.
  \item {\bf Small research}: A small group develops an MT system, and publishes it on their university web. They mark it as {\em research only/non commercial}
 \item {\bf Single person}: A lecturer or student --- or just an interested person outside of academia develops a system, and as above.
\end{itemize}

\end{frame}


\begin{frame}
  \frametitle{Community development model}

As opposed to the monolithic models outlined before, we prefer one which is,

\begin{itemize}
  \item {\bf User-driven}: Think of the needs of your users, and listen to them!
  \item {\bf Open}: Anyone can participate!
  \begin{itemize}
    \item Individual developers / linguists
    \item Companies
    \item Academic institutions
    \item Students
  \end{itemize}
  \item {\bf Release early, release often}
\end{itemize}

\end{frame}

\begin{frame}
  \frametitle{Participating} % Participating

\begin{itemize}
  \item {\bf Mailing lists}: % Mailing lists
  \begin{itemize}
    \item {\tt apertium-stuff}: For general discussion (over 100 members)
    \item {\tt apertium-turkic}: For discussion relating specifically to Turkic languages (over 30 members)
    \item {\tt apertium-celtic}: For discussion relating specifically to Celtic languages (6 members)
    \item {\tt apertium-uralic}: For discussion relating specifically to Uralic languages (5 members)
    \item \ldots all our mailing lists are multilingual, and new mailing lists can be made for specific development groups
  \end{itemize}
  \item {\bf IRC}: 
  \begin{itemize}
    \item {\tt \#apertium} on {\tt irc.freenode.net}
    \item \ldots multilingual, useful for resolving problems interactively and quickly
  \end{itemize}
  \item {\bf Wiki}:
  \begin{itemize}
    \item \url{http://wiki.apertium.org}
	    \item Collaboratively written documentation
    \item Has a wide range of information about Apertium and related tools
    \item \ldots mostly in English, but with some translations (some pages have \\been translated into more than 15 languages)
  \end{itemize}  
\end{itemize}

% mailing list

\end{frame}
%%%%%%%%%%%%%%%%%%%%%%%%%%%%%%%%%%%%%%%%%%%%%%%%%%%%%%%%%%%%%%%%%%%%%%%%%%%%%%%%%%%%%%%
%% architecture
%%%%%%%%%%%%%%%%%%%%%%%%%%%%%%%%%%%%%%%%%%%%%%%%%%%%%%%%%%%%%%%%%%%%%%%%%%%%%%%%%%%%%%%

\begin{frame} %% framesection
 \begin{center}
 {\Large {\bf Structure}}
 \end{center}
\end{frame}

\begin{frame}
  \frametitle{Structure}

    Hands-on session starts here...

\includegraphics[width=\textwidth]{architecture.pdf}


\end{frame}



\begin{frame}
  \frametitle{Morphological analysis}

\begin{onlyenv}<1>
\includegraphics[width=\textwidth]{architecture-2.pdf}
\end{onlyenv}
\begin{onlyenv}<2>

  \begin{block}{Morphological analysis}
 \begin{itemize}
   \item segments the source text in \emph{surface forms} (SFs),  
   \item assigns to each SF one or more \emph{lexical forms} (LFs)
 \end{itemize}
  \end{block}
  \begin{block}{Running the morphological analyser ({\tt hfst-proc})} % Running the morphological analyser
  \begin{small}
    \texttt{\$ echo "пахчара" | hfst-proc chv-tur.automorf.hfst}
    %\texttt{\$ echo "пахчара" | hfst-proc chv-tur.automorf.hfst}
    \texttt{ \^{}пахчара/пахча$<$n$><$loc$>$/пахча$<$n$><$px2pl$><$dat$>$\$}
  \end{small}
  \end{block}
  \begin{block}{Running the morphological analyser ({\tt lt-proc})}
  \begin{small}
    \texttt{\$ echo "саду" | lt-proc rus-hbs.automorf.bin}
    \texttt{ \^{}саду/сад$<$n$><$m$><$nn$><$sg$><$nom$>$/сад$<$n$><$m$><$nn$><$sg$><$acc$>$\$}
  \end{small}
  \end{block}
\end{onlyenv}

\end{frame}

\begin{frame}
  \frametitle{Morphological disambiguation}


\begin{onlyenv}<1>
\includegraphics[width=\textwidth]{architecture-3.pdf}
\end{onlyenv}
\begin{onlyenv}<2>

  \begin{block}{Morphological disambiguation}
 \begin{itemize}
  \item picks one of the LFs corresponding to each ambiguous SF according to context
  \item uses a combination of hand-written disambiguation rules and hidden Markov models
 \end{itemize}
  \end{block}
  \begin{block}{Running the disambiguator ({\tt cg-proc})} % Running the tagger
  \begin{small}
    \texttt{\$ echo "Все эти регионы являются частью Европейского союза." | lt-proc rus-hbs.automorf.bin |}
    \texttt{     cg-proc rus-hbs.rlx.bin}
    \texttt{\^{}Все/Весь$<$det$><$ind$><$mfn$><$pl$><$nom$>$\$ \^{}эти/этот$<$det$><$dem$><$mfn$><$pl$><$nom$>$\$}
    \texttt{\^{}регионы/регион$<$n$><$m$><$nn$><$pl$><$nom$>$\$ \ldots}
  \end{small}
  \end{block}

\end{onlyenv}
% Все эти регионы являются частью Европейского союза.
% ^Все/Весь<det><ind><mfn><pl><nom>$ ^эти/этот<det><dem><mfn><pl><nom>$ ^регионы/регион<n><m><nn><pl><nom>$ ^являются/являться<vblex><imperf><pres><p3><pl>$ ^частью/часть<n><f><nn><sg><ins>$ ^Европейского/Европейский<adj><m><sg><gen>$ ^союза/союз<n><m><nn><sg><gen>$^./.<sent>$

\end{frame}

\begin{frame}
  \frametitle{Lexical transfer} % Lexical transfer

\begin{onlyenv}<1>
\includegraphics[width=\textwidth]{architecture-4.pdf}
\end{onlyenv}
\begin{onlyenv}<2>

 \begin{block}{Lexical transfer} % Lexical transfer
  \begin{itemize}
 \item reads each SL LF and generates the corresponding TL LFs
 \item reads finite-state transducers generated from bilingual dictionaries in XML
  \end{itemize}
  \end{block}
  \begin{block}{Running the lexical transfer ({\tt lt-proc -b})} % Running the lexical transfer
  \begin{small}
     \texttt{\$ echo "Әхмәт һәм Гөлнара бакчада." | hfst-proc tat-bak.automorf.hfst | cg-proc tat-bak.rlx.bin | }
     \texttt{  apertium-tagger -g tat-bak.prob | lt-proc -b tat-bak.autobil.bin}
     \texttt{\^{}Әхмәт$<$np$><$ant$><$m$><$nom$>$/Әхмәт$<$np$><$ant$><$m$><$nom$>$\$ \^{}һәм$<$cnjcoo$>$/һәм$<$cnjcoo$>$\$ \^{}Гөлнара$<$np$><$ant$><$f$><$nom$>$/Гөлнара$<$np$><$ant$><$f$><$nom$>$\$ \^{}бакча$<$n$><$loc$>$/баҡса$<$n$><$loc$>$\$ }
  \end{small}
  \end{block}
\end{onlyenv}
\end{frame}

\begin{frame}
  \frametitle{Lexical selection} % Lexical selection}

\begin{onlyenv}<1>
\includegraphics[width=\textwidth]{architecture-5.pdf}
\end{onlyenv}
\begin{onlyenv}<2>

 \begin{block}{Lexical selection}
  \begin{itemize}
 \item for each pair of SL/TL translations, chooses the most appropriate translation in context
 \item rules are written in an XML format and compiled into a finite-state transducer
  \end{itemize}
  \end{block}
  \begin{block}{Running the lexical selection ({\tt lrx-proc})} % Running the lexical selection
  \begin{small}
    \texttt{\$ echo "Minä pidän sinusta." | hfst-proc fin-sme.automorf.hfst | cg-proc fin-sme.rlx.bin |}
    \texttt{  apertium-tagger -g fin-sme.prob | lt-proc -b fin-sme.autobil.bin | lrx-proc fin-sme.lrx.bin}\\
    \texttt{ \^{}Minä$<$Pron$><$Pers$><$Sg$><$Nom$>$/Mun$<$Pron$><$Pers$><$Sg$><$Nom$>$\$ \^{}pitää$<$V$><$Ind$><$Prs$><$Sg1$>$/liikot$<$V$><$Ind$><$Prs$><$Sg1$>$\$ \^{}sinä$<$Pron$><$Pers$><$Sg$><$Ela$>$/don$<$Pron$><$Pers$><$Sg$><$Ela$>$\$ }
  \end{small}
  \end{block}
\end{onlyenv}
\end{frame}

\begin{frame}
  \frametitle{Structural transfer} % Structural transfer


\begin{onlyenv}<1>
\includegraphics[width=\textwidth]{architecture-6.pdf}
\end{onlyenv}
\begin{onlyenv}<2>

  \begin{block}{Structural transfer}
 \begin{itemize}
 \item Rules have a \emph{pattern}--\emph{action} form.
 \item It detects LF patterns to be processed using a left-to-right, longest-match strategy.
 \item It executes the actions associated to each pattern in the rule file to generate the corresponding LF pattern for the TL.
 \end{itemize}
  \end{block}
  \begin{block}{Running the structural transfer}
  \begin{small}
  \texttt{ echo "Әхмәт тиз генә иске зур бер агачка йөгерә, аның артына Гөлнарадан яшеренә."  | \ldots | }
  \texttt{ apertium-transfer -b apertium-tat-kir.tat-kir.t1x tat-kir.t1x.bin | }
  \texttt{ apertium-interchunk apertium-tat-kir.tat-kir.t2x tat-kir.t2x.bin | apertium-postchunk apertium-tat-kir.tat-kir.t3x tat-kir.t3x.bin }
 
  \texttt{ \^{}Акмат$<$np$><$ant$><$m$><$nom$>$\$ \^{}эски$<$adj$>$\$ \^{}чоң$<$adj$>$\$ \^{}бир$<$det$><$ind$>$\$ \ldots }
%^дарак<n><acc>$ ^көздөй<post>$ ^катуу<adv>$ ^гана<postadv>$ ^чурка<v><iv><prt_perf>$ ^бар<v><iv><prt_impf>$ ^жат<vaux><aor><p3><sg>$^..<sent>$

  \end{small}
  \end{block}

\end{onlyenv}
\end{frame}

\begin{frame}
  \frametitle{Morphological generation} % Morphological generation

\begin{onlyenv}<1>
\includegraphics[width=\textwidth]{architecture-7.pdf}
\end{onlyenv}
\begin{onlyenv}<2>

 \begin{block}{Morphological generation}
 \begin{itemize}
 \item Generates a TL SF from each TL LF after adequately inflecting it
 \item It reads finite-state transducers generated from a morphological dictionary 
 \end{itemize}
 \end{block}
 \begin{block}{Running morphological generation ({\tt hfst-proc -g})} % Running morphological generation
     \texttt{\$ echo "Әхмәт тиз генә иске зур бер агачка йөгерә, аның артына Гөлнарадан яшеренә." | \ldots | hfst-proc -g tat-kir.autogen.hfst}\\
     \texttt{ Акмат эски чоң бир даракты көздөй катуу гана чуркап бара жатат.}
 \end{block}
% \begin{block}{Running morphological generation ({\tt lt-proc})}
%   \begin{small}
%     \texttt{\$ echo "\cyrtext{Әхмәт тиз генә иске зур бер агачка йөгерә, аның артына Гөлнарадан яшеренә.}" | \ldots | hfst-proc -g tat-kir.autogen.hfst}\\
%     \texttt{ \cyrtext{Акмат эски чоң бир даракты көздөй катуу гана чуркап бара жатат.}}
%   \end{small}
% \end{block}
\end{onlyenv}
\end{frame}

\begin{frame}
  \frametitle{Structure}

\includegraphics[width=\textwidth]{architecture.pdf}

\end{frame}



\end{document}
