\documentclass[10pt,xetex]{beamer} %[compress, blue]
\mode<presentation>

\usepackage{multicol}

\date{\today}
\title{{Morphological disambiguation}}
\begin{document}

\section{Introduction}

\begin{frame} %% framesection
 \begin{center}
 {\Large {\bf Introduction}}  % Introduction
 \end{center}
\end{frame}

\begin{frame}
  \frametitle{Introduction}

\begin{itemize}
  \item What is morphological ambiguity ?
  \begin{itemize}
    \item The ambiguity that comes from a surface form having more than one possible analysis.
  \end{itemize}
  \item Why is it important ?
  \begin{itemize}
    \item Sometimes different readings will lead to different translations, and we want
       to get the appropriate translation.
  \end{itemize}
\pause
  \item For example: « \emph{She looks under the table and under the chairs.} »
    \begin{itemize}
       \item \texttt{PRON VERB PREP DET NOUN CC PREP DET NOUN.}
       \item ++ Hän katsoo pöydän alle ja tuolien alle.
    \end{itemize}
    \begin{itemize}
       \item \texttt{PRON NOUN ADV DET VERB CC ADV DET VERB.}
       \item -- Hän katsomiset huonommin panee pöydälle ja huonommin johtaa puhetta.
    \end{itemize}
\end{itemize}

\end{frame}

\begin{frame}
   \frametitle{Solving ambiguity: Constraint Grammar}

\begin{itemize}
  \item Takes input consisting of surface forms and their readings
  \item Applies hand-written rules to remove inappropriate readings in context
  \item Originally developed in Finland (Fred Karlsson)
  \begin{itemize}
    \item see: \emph{Constraint Grammar: A Language-Independent System for Parsing Unrestricted Text}
    \item \ldots but we use the free/open-source VISLCG3 implementation from the U. Southern Denmark
  \end{itemize}
\end{itemize}

Free/open-source grammars available for: Finnish, North Sámi, Faroese

\end{frame}

\begin{frame}
  \frametitle{Concepts}

\begin{itemize}
  \item  {\bf  cohort} --- a surface form of a word, along with its analyses (possible lexical units), an ambiguous lexical unit.

   \begin{itemize}
         \item  Apertium equivalent: \texttt{\^{}words/word<n><pl>/word<vblex><pres><p3><sg>\$}
   \end{itemize}

    \item {\bf baseform} --- the lemma of a word.
    \item {\bf reading} --- a single analysis of a word.

   \begin{itemize}
        \item Apertium equivalent: \texttt{\^{}word<n><pl>\$}
   \end{itemize}

    \item {\bf wordform} --- a surface form of a word.
\end{itemize}

\end{frame}

\begin{frame}
  \frametitle{How the rules work}

{\bf Operations}:

\begin{itemize}

  \item \texttt{REMOVE} --- Remove a reading that matches in  context

  \item \texttt{SELECT} --- Remove all readings except the one that matches in context
\end{itemize}

{\bf Contexts}: Consist of a \alert<2>{position} and a \alert<3>{pattern}

\begin{itemize}
  \item \texttt{(\alert<2>{0} \alert<3>{noun})}
  \item \texttt{(\alert<2>{-1*} \alert<3>{noun})}
  \item \texttt{(\alert<2>{2*} \alert<3>{noun})}
  \item \texttt{(\alert<2>{1}\alert<3>{C noun})} --- \texttt{C} = careful
\end{itemize}


\end{frame}

% Morphological disambiguation
%%%%%%%%%%%%%%%%%%%%%%%%%%%%%%%%%%%%%%%%%%%%%%%%%%%%%%%%%%%%%%%%%%%%%%%%%%%%%%%

\begin{frame}


 \begin{center}
 {\Large {\bf Morphological disambiguation}}
 \end{center}

\end{frame}

\begin{frame}[fragile]
 %« Jean et Marie adorent jouer, ils jouent toujours ensemble dans le jardin devant la grande maison. »

\begin{multicols}{2}
\begin{verbatim}
"<Jean>"
        "Jean" np ant
"<et>"
        "et" cnjcoo
"<Marie>"
        "Marie" np ant
        "Marier" vblex pri p3 sg
        "Marier" vblex pri p1 sg
        "Marier" vblex prs p3 sg
        "Marier" vblex prs p1 sg
        "Marier" vblex imp p2 sg
"<adorent>"
        "adorer" vblex pri p3 pl
        "adorer" vblex prs p3 pl
"<jouer>"
        "jouer" vblex inf
"<,>"
        "," cm
"<ils>"
        "il" prn tn p3 m pl
"<jouent>"
        "jouer" vblex pri p3 pl
        "jouer" vblex prs p3 pl
"<toujours>"
        "toujours" adv
"<ensemble>"
        "ensemble" adv
        "ensemble" n m sg
"<dans>"
        "dans" pr
"<le>"
        "le" prn pro p3 nt
        "le" det def m sg
        "le" prn pro p3 m sg
"<jardin>"
        "jardin" n m sg
"<.>"
        "." sent

\end{verbatim}
\end{multicols}

\end{frame}

\begin{frame}
\frametitle{Formulating rules}

\begin{center}
 « \textit{Jean et \alert<4>{Marie} \alert<3>{adorent} jouer, ils
    \alert<3>{jouent} toujours ensemble dans le jardin devant \alert<2>{la
    grande maison}.} »
\end{center}

Looking at the previous sentence, we could come up with the following rules:

\begin{itemize}
  \item No proclitic pronoun without a following verb or other proclitic pronoun.
  \item No two finite verbs can appear in the same clause
  \item A verb in the present subjunctive \texttt{prs} is rare without a preceeding \emph{que}, \emph{qui}, \emph{quoique}.
\end{itemize}

\end{frame}

\begin{frame}[fragile]
\frametitle{Lists and sets}

% LISTS

\begin{verbatim}
LIST DET = det ;
LIST PRS = prs ;
LIST N = n ;
LIST QUI-QUE = "qui" "que" "quoique" ;
LIST V-FIN = prs pri pii imp ;  # prs OR pri OR pii OR imp
LIST PRO-PRON = (prn pro) ; # prn AND pro
LIST SUBJ-PRON = ("ils" prn) ("je" prn) ;
\end{verbatim}

\begin{verbatim}
SET CLB = SUBJ-PRON OR PRO-PRON OR V-FIN ; # Clause boundary
\end{verbatim}

\end{frame}

\begin{frame}
  \frametitle{\texttt{REMOVE} rules}

Remove a proclitic pronoun (\texttt{PRO-PRON}) reading if:

\begin{itemize}
  \item \texttt{(0 DET)}:  The current word can be a determiner
  \item \texttt{(NOT 1 V-FIN)}:  There is no following finite verb
  \item \texttt{(1 N)}: The following verb is a noun
\end{itemize}

\pause
\begin{center}
\texttt{
REMOVE PRO-PRON IF (0 DET) (NOT 1 V-FIN) (1 N);
}
\end{center}
\end{frame}

\begin{frame}[fragile]
 %« Jean et Marie adorent jouer, ils jouent toujours ensemble dans le jardin devant la grande maison. »

\begin{multicols}{2}
\begin{verbatim}
"<Jean>"
        "Jean" np ant
"<et>"
        "et" cnjcoo
"<Marie>"
        "Marie" np ant
        "Marier" vblex pri p3 sg
        "Marier" vblex pri p1 sg
        "Marier" vblex prs p3 sg
        "Marier" vblex prs p1 sg
        "Marier" vblex imp p2 sg
"<adorent>"
        "adorer" vblex pri p3 pl
        "adorer" vblex prs p3 pl
"<jouer>"
        "jouer" vblex inf
"<,>"
        "," cm
"<ils>"
        "il" prn tn p3 m pl
"<jouent>"
        "jouer" vblex pri p3 pl
        "jouer" vblex prs p3 pl
"<toujours>"
        "toujours" adv
"<ensemble>"
        "ensemble" adv
        "ensemble" n m sg
"<dans>"
        "dans" pr
"<le>"
        "le" det def m sg
;  "le" prn pro p3 nt REMOVE:12
;  "le" prn pro p3 m sg REMOVE:12
"<jardin>"
        "jardin" n m sg
"<.>"
        "." sent

\end{verbatim}
\end{multicols}

\end{frame}

\begin{frame}
  \frametitle{\texttt{REMOVE} rules}

Remove a finite-verb reading (\texttt{V-FIN}) if:

\begin{itemize}
  \item \texttt{(1C V-FIN)}: The following word can only be a finite verb
\end{itemize}

\pause
\begin{center}
\texttt{
REMOVE V-FIN IF (1C V-FIN) ;
}
\end{center}
\end{frame}


\begin{frame}[fragile]
 %« Jean et Marie adorent jouer, ils jouent toujours ensemble dans le jardin devant la grande maison. »

\begin{multicols}{2}
\begin{verbatim}
"<Jean>"
        "Jean" np ant
"<et>"
        "et" cnjcoo
"<Marie>"
        "Marie" np ant
;  "Marier" vblex imp p2 sg
;  "Marier" vblex pri p3 sg REMOVE:14
;  "Marier" vblex pri p1 sg REMOVE:14
;  "Marier" vblex prs p3 sg REMOVE:14
;  "Marier" vblex prs p1 sg REMOVE:14
"<adorent>"
        "adorer" vblex pri p3 pl
        "adorer" vblex prs p3 pl
"<jouer>"
        "jouer" vblex inf
"<,>"
        "," cm
"<ils>"
        "il" prn tn p3 m pl
"<jouent>"
        "jouer" vblex pri p3 pl
        "jouer" vblex prs p3 pl
"<toujours>"
        "toujours" adv
"<ensemble>"
        "ensemble" adv
        "ensemble" n m sg
"<dans>"
        "dans" pr
"<le>"
        "le" det def m sg
"<jardin>"
        "jardin" n m sg
"<.>"
        "." sent

\end{verbatim}
\end{multicols}

\end{frame}

\begin{frame}
  \frametitle{\texttt{REMOVE} rules}

Remove a present subjunctive reading if:

\begin{itemize}
   \item \texttt{(NOT -1* QUI-QUE BARRIER CLB)}: There is no `que', `qui' or `quoique' between the
      current word and a clause boundary to the left.
\end{itemize}

\pause
\begin{center}
\texttt{
REMOVE PRS IF (NOT -1* QUI-QUE BARRIER CLB);
}
\end{center}

\end{frame}



\begin{frame}[fragile]
 %« Jean et Marie adorent jouer, ils jouent toujours ensemble dans le jardin devant la grande maison. »

\begin{multicols}{2}
\begin{verbatim}
"<Jean>"
        "Jean" np ant
"<et>"
        "et" cnjcoo
"<Marie>"
        "Marie" np ant
"<adorent>"
        "adorer" vblex pri p3 pl
;  "adorer" vblex prs p3 pl REMOVE:16
"<jouer>"
        "jouer" vblex inf
"<,>"
        "," cm
"<ils>"
        "il" prn tn p3 m pl
"<jouent>"
        "jouer" vblex pri p3 pl
;  "jouer" vblex prs p3 pl REMOVE:16
"<toujours>"
        "toujours" adv
"<ensemble>"
        "ensemble" adv
        "ensemble" n m sg
"<dans>"
        "dans" pr
"<le>"
        "le" det def m sg
"<jardin>"
        "jardin" n m sg
"<.>"
        "." sent

\end{verbatim}
\end{multicols}

\end{frame}

\begin{frame}

\begin{center}
But what about the \emph{ensemble} `together' \\ and \emph{ensemble} `ensemble, group'?
\end{center}

Not so straightforward \ldots \\
~\\

\pause

But here are some ideas:

\begin{itemize}
  \item If the word is used in an NP (preceeded by an article/adjective: `l'ensemble', `un ensemble', `le grand ensemble'
  \item If the word is modified by a prepositional phrase with \emph{de}: `ensemble de les conducteurs'
  \begin{itemize}
    \item Caution: `\ldots qui liait ensemble des peuples aussi divers que \ldots'
  \end{itemize}
  \item If the word is followed by an NP without a preposition: `pour conduire ensemble un projet', `ensemble ils sont'
\end{itemize}

Looking at a concordance can often be enlightening :)

\end{frame}

\begin{frame}[fragile]

\begin{verbatim}
"<Лицо>"
        "Лицо" n nt nn sg acc
        "Лицо" n nt nn sg nom
"<города>"
        "город" n m nn sg gen
        "город" n m nn pl acc
        "город" n m nn pl nom
"<менялось>"
        "меняться" vblex impf iv past nt sg
"<вместе>"
        "вместе" adv
"<со>"
        "с" pr
"<сменой>"
        "смена" n f nn sg ins
"<политических>"
        "политический" adj mfn nn pl gen
        "политический" adj mfn nn pl prp
        "политический" adj mfn aa pl acc
"<эпох>"
        "эпоха" n f nn pl gen
"<.>"
        "." sent
\end{verbatim}

\end{frame}

\begin{frame}[fragile]
  \frametitle{\texttt{SELECT} rules}

Remove all readings of a word which can only be an adjective which do not agree in case with the head noun:

\begin{verbatim}
    LIST A = adj ;
    LIST N = n ;
    LIST NAGDIP = nom acc gen dat ins prp ;

    SECTION

    SELECT A + $$NAGDIP IF (0C A) (1C N + $$NAGDIP) ;
\end{verbatim}

The \texttt{\$\$} symbol before a list is \emph{unification} --- it is basically equivalent to writing
out all the combinations of the list:

\begin{itemize}
   \item \texttt{SELECT A + (nom) IF (0C A) (1C N + (nom));}
   \item \texttt{SELECT A + (gen) IF (0C A) (1C N + (gen));}
   \item \texttt{SELECT A + (dat) IF (0C A) (1C N + (dat));}
   \item \ldots
\end{itemize}

\end{frame}

\begin{frame}[fragile]

\begin{verbatim}
"<Лицо>"
        "Лицо" n nt nn sg acc
        "Лицо" n nt nn sg nom
"<города>"
        "город" n m nn sg gen
        "город" n m nn pl acc
        "город" n m nn pl nom
"<менялось>"
        "меняться" vblex impf iv past nt sg
"<вместе>"
        "вместе" adv
"<со>"
        "с" pr
"<сменой>"
        "смена" n f nn sg ins
"<политических>"
        "политический" adj mfn nn pl gen SELECT:8
;       "политический" adj mfn nn pl prp SELECT:8
;       "политический" adj mfn aa pl acc SELECT:8
"<эпох>"
        "эпоха" n f nn pl gen
"<.>"
        "." sent
\end{verbatim}

\end{frame}



\begin{frame}

For the rest of the ambiguities I'd use \texttt{REMOVE} rules:

\begin{itemize}
  \item No accusative without a preposition in a clause with an intransitive verb.
  \item A nominative should agree with a past tense verb in gender and number.
\end{itemize}

\end{frame}


\begin{frame}[fragile]
  \frametitle{Useful commands}

{\bf Get the tagger output}:

\begin{verbatim}
  cat filename.txt | apertium -d . xxx-yyy-tagger
\end{verbatim}

{\bf Get the tagger output in CG format}:

\begin{verbatim}
  cat filename.txt | apertium -d . xxx-yyy-disam
\end{verbatim}

\end{frame}

\end{document}
