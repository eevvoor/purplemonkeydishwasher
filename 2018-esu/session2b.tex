\documentclass[10pt,xetex]{beamer} %[compress, blue]
\mode<presentation>

\usepackage{colortbl}
\usepackage{tabularx}
%\usepackage[breaklinks]{hyperref}

\date{\today}
\title{Session 2: Morphophonology with \texttt{twol}}
\begin{document}



\begin{frame}
	\frametitle{Morphophonology: what is it?}
	\begin{definition}{Morphophonology}
		\begin{itemize}
			\item ``\textit{a branch of linguistics which studies the interaction between morphological and phonological or phonetic processes}'' -wikipedia
			\item or looking at ``the sound changes that take place in morphemes when they combine to form words''
			\item or, for us, just ``which changes occur in morphemes as a result of their combination''
		\end{itemize}
	\end{definition}
	\begin{block}{Morphophonology: why it matters}
		\begin{itemize}
			\item In many languages, many words and morphemes change form when they cooccur
			\item MT systems should recognise all forms of any word or morpheme
			\item ... and we have to tell the computer how it can deal with these processes
		\end{itemize}
	\end{block}
\end{frame}

\begin{frame}
	\frametitle{Morphophonology and \texttt{twol}}
	\begin{block}{What is \texttt{twol}?}
		\begin{itemize}
			\item A formalism for implementing morphophonological rules
		\end{itemize}
	\end{block}
	\begin{block}{What do I need to use \texttt{twol} for?}
		\begin{itemize}
			\item When the form of a word or morpheme changes as a result of phonological rules
		\end{itemize}
	\end{block}
	\begin{block}{How can I use \texttt{twol}?}
		\begin{itemize}
			\item Find the correct phonological rules\\
			(They aren't always in the grammars, but the grammars can help)
			\item Write the rules in the \texttt{twol} formalism!
		\end{itemize}
	\end{block}

\end{frame}

\begin{frame}
    \frametitle{Most important: When to use twol---or not}
    \begin{block}{When to use \texttt{twol}}
        \begin{itemize}
            \item Kimmo gives 2 * 2 reasons for decisions (see the
                book)\footnote{Koskenniemi, Kimmo, Two-level Morphology: A
                General Computational Model for Word-Form Recognition and
                Production, Publications, No. 11, 160 pages, University of
                Helsinki, Department of General Linguistics (1983).}:
            \item natural---suppletic variation (m : n or a : t)
            \item phonological---morphological conditioning, or orthographic
                (before all labials? before past tense?)
            \item I'll give you third: frequency (s : h) in Finnish is natural and
                morphologic but happens once.
        \end{itemize}
    \end{block}
\end{frame}

\begin{frame}
    \frametitle{Concrete example: Finnish and vowel harmony}
\texttt{LEXICON nouns\\
\textbf{xaxo} backvowelcasesfornouns ; \\
\textbf{xixe} frontvowelcasesfornouns ; \\
\textbf{xäxö} frontvowelcasesfornouns ;\\
LEXICON backvowelcasesfornouns\\
ssa backclits ; \\
sta backclits ; \\
lla backclits ;\\
lta backclits ;\\
LEXICON frontvowelcasesfornouns\\
ssä frontclits ;  \\
stä frontclits ; \\
llä frontclits ; \\
ltä frontclits ;\\
LEXICON backclits\\
ko \# ; pa \# ; han \# ;\\
LEXICON frontclits\\
kö \# ; pä \# ; hän \# ;\\
    }
\end{frame}

\begin{frame}
    \frametitle{Concrete example cont'd}
    \texttt{LEXICON nouns\\
        xaxo casesfornouns ; xixe casesfornouns ; \\
        xäxö casesfornouns ;\\
        LEXICON casesfornouns\\
        ss\{A\} clits ;  \\
	st\{A\} clits;  \\
	ll\{A\} clits ;  \\
	lta clits ; \\
        LEXICON clits\\
        k\{O\} \# ;  \\
	p\{A\} \# ;  \\
	h\{A\}n \# ;\\
    }
\end{frame}
%
\begin{frame}
    \frametitle{Twol workings: START from Alphabet!}
    \texttt{Alphabet\\
        a b c ... x y z \{A\}:a \{A\}:ä \{O\}:o \{O\}:ö\\
        ;\\
        ~\\
        Sets \\
        FRONT = ä ö y ;\\
        BACK = a o u ;\\
        NEUT = i e ;\\
        Rules\\
        "No rules!!"\\
        a <=> \_ ;
    }
~\\
~\\
    What happens now? Try it: you should get xaxossako xaxossakö xaxossäkö
    xaxossäko ... etc. This is where we start mentally!
~\\
~\\
\texttt{
\$ hfst-lexc myfile.lexc -o myfile.lexc.hfst \\
\$ hfst-twolc myfile.twol -o myfile.twol.hfst \\
\$ hfst-compose-intersect -1 myfile.lexc.hfst -2 myfile.twol.hfst \\-o myfile.hfst \\
}
\end{frame}
%
\begin{frame}
    \frametitle{Twol rules are not:}
\begin{itemize}
  \item  Twol rules are not for rewriting.
  \item  Twol rules do not make realisations.
  \item  Twol rules do not change anything into anything.
    \begin{itemize}
      \item Previous example did not have rules to begin with.
      \item \ldots but everything was changed into everything!
   \end{itemize}
\end{itemize}
Keep in mind: in twol you write rules to restrict wrong things out. See negative readings of Appendix TWOL of
    Beesley---Karttunen (2003)---but ignore most of the rest for now.
%    \url{http://fsmbook.com} $\rightarrow$ New Software.
\end{frame}
%
%\begin{frame}
%    \frametitle{Twol rules are not:}
%    Twol rules are not for rewriting. Twol rules do not make realisations.
%    Twol rules do not change anything into anything. (Previous example did not
%    have rules to begin with.) Keep in mind: in twol you write rules to restrict
%    wrong things out. See negative readings of Appendix TWOL of
%\end{frame}


\begin{frame}
    \frametitle{Rules and linguistic thinking}
    Ideally, in my opinion, rules should talk about linguistics. Vowel harmony
    is about linguistics and easy to formulate: \_\_\_
\end{frame}

\begin{frame}
    \frametitle{Twol rules are these many arrowy things}
    \begin{block}{Arrows}
    So, how encode the linguistic logic is quite mathematical logic in twol
    (just experiment with all of these):
    \begin{itemize}
        \item \texttt{\{A\}:ä <= ä \textbackslash a* \_ ;} whenever you see \textit{ä} left
            of \{A\} and no \textit{a}'s in between, go with \textit{ä}
        \item \texttt{\{A\}:a => a \textbackslash ä* \_ ;} \{A\} will not be \textit{a} elsewhere
            than when preceded with \textit{a} and no \textit{ä}'s in between
        \item \texttt{\{A\}:a <=> a \textbackslash ä* \_ ;} this \{A\}:a is restricted
            to go after \textit{a}'s and other \{A\}'s must be \textit{a} (these are majority of all
            rules in typical systems)
        \item \texttt{\{A\}:a /<= ä \textbackslash a \_ ;} \{A\} will not be \textit{a} if
            there's some \textit{ä} in leftwards vicinity (use these rules to
            get out from some tricky situations)
    \end{itemize}
    if you experiment with these and pay attention to all results in above
    cases you'll understand all implications
    \end{block}
\end{frame}



\end{document}
