\documentclass[10pt,xetex]{beamer} %[compress, blue]
\mode<presentation>

\usepackage{helsingfors}

\usepackage{alltt}

\date{2nd--13th November, 2015}
\title{Session 5: Structural transfer}
\begin{document}

\begin{frame}
        \titlepage
\MyLogoBottomCentred
\end{frame}

\logo{\includegraphics[height=1.6cm]{../logo/logo}}


%\begin{frame}
%
%\end{frame}

\begin{frame}
 \frametitle{Introduction}

\begin{itemize}
  \item {\bf Contrastive analysis}
~\\
~\\
  \item {\bf Rule writing}
~\\
~\\
  \item {\bf Structural transfer in Apertium}
\end{itemize}

\end{frame}

\begin{frame} %% framesection
 \begin{center}
 {\Large {\bf Contrastive analysis}} 
 \end{center}
\end{frame}

\begin{frame}
  \frametitle{Contrastive analysis}

Contrastive analysis is the process of examining two or more languages together to find out what kind of features they share, and how they are distinguished. 

\end{frame}

\begin{frame}
  \frametitle{Morphological contrasts}

We call morphological contrasts those differences between languages which are expressed at the level of a single word. Some examples might be:

\begin{itemize}
  \item How possession is expressed: suffix, or separate word
  \item Case inventories
  \item Gender (or word class) inventories
  \item Number (in all words, or just in nouns/pronouns)
  \item ``Synthetic'' verb tenses
  \item Person inventory 
  \item Formality
  \item Morpheme order (e.g. possessive before/after number)
\end{itemize}

\end{frame}

\begin{frame}
  \frametitle{Syntactic contrasts}

Syntactic contrasts on the other hand are differences between words

\begin{itemize}
  \item Existence or not of separate articles
  \item Existence or not of compulsory subject pronouns
  \item Word order (both within phrases, and between phrases)
  \item ``Analytic'' verb tenses
\end{itemize}

\end{frame}



%%%%%%%%%%%%%%%%%%%%%%%%%%%%%%%%%%%%%%%%%%%%%%%%%%%%%%%%%%%%%%%%%%%%%%%%%%%%%%%
%%
%%%%%%%%%%%%%%%%%%%%%%%%%%%%%%%%%%%%%%%%%%%%%%%%%%%%%%%%%%%%%%%%%%%%%%%%%%%%%%%

\begin{frame} %% framesection
 \begin{center}
 {\Large {\bf Formalising rules}} % Formalising rules
 \end{center}
\end{frame}


\begin{frame}
  \frametitle{Creating rules}

\begin{itemize}
\item In Apertium, our intermediate representation is based on lemmas, parts-of-speech and morphological tags. Syntactic tags may also be used if available.
\item So, when writing rules, they should be based on this level of intermediate representation
\end{itemize}

\end{frame}

\begin{frame}
  \frametitle{Operations}

When thinking about implementing transfer rules, it is often worth trying to think in terms of some very basic operations. 

The most basic are:

\begin{itemize}
  \item {\bf Insertion}: Add a new word or tag, 
  \begin{itemize}
    \item e.g., (\texttt{tur}) {\em Yeme!} $\rightarrow$ (\texttt{chv}) {\em {\bf Ан} çи!}. 
  \end{itemize}
  \item {\bf Deletion}: Removing a word or a tag, 
  \begin{itemize}
    \item e.g., (\texttt{rus}) {\tt <n>{\bf <m><aa>}<nom>} $\rightarrow$ (\texttt{chv}) {\tt <n><nom>}
  \end{itemize}
\end{itemize}

The following operations are also basic operations, but can be considered to be permutations of the first two:

\begin{itemize}
  \item {\bf Substitution}: Replacing one tag or word, with another tag or word
  \begin{itemize}
    \item e.g. the vocative case in Ukranian has to be replaced with nominative in Russian 
  \end{itemize}
  \item {\bf Reordering}: Changing the order of words, or tags
  \begin{itemize}
    \item e.g. the order of plural and possessive morphemes in Chuvash and Turkish: {\em kitap·{\bf lar·ım}} $\rightarrow$ {\em кӗнеке·{\bf м·сем}}
  \end{itemize}
\end{itemize}

\end{frame}

\begin{frame}
  \frametitle{Combining operations} % Combining operations

Most actual transfer rules will require more than a single operation.  Consider the following example of translating the abilitative from Turkish into Kyrgyz. The rule:

\begin{itemize}
  \item The abilitative in Turkish is formed with a suffix, {\em -AbIl-}, which then takes verb endings
  \item In Kyrgyz, it is formed with the imperfect participle and the auxiliary {\em ал}, which takes the verb endings
\end{itemize}

Formalised:

\begin{itemize}
  \item {\bf Input}: {\em gelebilirim} = {\tt gel<v><abil><aor><p1><sg>} % Input
  \begin{itemize}
    \item ({\sc ins}) Add auxiliary verb {\em ал} after the main verb
    \item ({\sc del}) Remove {\tt <abil>} tag from the main verb
    \item ({\sc ins}) Set the person, number and tense of the auxiliary verb to be the same as the main verb
    \item ({\sc del}) Remove the person, number from the main verb and set the tense to be imperfect participle
  \end{itemize}
  \item {\bf Output}: {\tt кел<v><iv>\alert<4>{<prt\_{}impf>} \alert<1>{ал<v><tv>\alert<3>{<aor><p1><sg>}}} = {\em келе алам} % Output
\end{itemize}


\end{frame}

%\begin{frame}
%  \frametitle{Some advice}
%
%
%
%\end{frame}
%

%%%%%%%%%%%%%%%%%%%%%%%%%%%%%%%%%%%%%%%%%%%%%%%%%%%%%%%%%%%%%%%%%%%%%%%%%%%%%%%
%%
%%%%%%%%%%%%%%%%%%%%%%%%%%%%%%%%%%%%%%%%%%%%%%%%%%%%%%%%%%%%%%%%%%%%%%%%%%%%%%%


\begin{frame} %% framesection
 \begin{center}
 {\Large {\bf Structural transfer in Apertium}} 
 \end{center}
\end{frame}

\begin{frame}
\begin{center}
\includegraphics[width=\textwidth]{architecture-6.pdf}
\end{center}
\end{frame}


\begin{frame}
  \frametitle{Structural transfer}

There are several types of transfer arrangements in Apertium language pairs,

\begin{itemize}
  \item Single-level transfer:
\includegraphics[width=0.8\textwidth]{apertium1.pdf}

  \item Multiple-level transfer (without chunking):
\includegraphics[width=0.8\textwidth]{apertium2.pdf}

  \item Multiple-level transfer:

\includegraphics[width=0.8\textwidth]{apertium3.pdf}
  
\end{itemize}

\end{frame}

\begin{frame}
\frametitle{How the transfer works}

\begin{itemize}
\item Rules detect patterns of words (generally specified as parts of speech):

  \begin{itemize}
  \item DET NOUN\\
    \emph{The table}
  \item DET ADJ NOUN\\ 
    \emph{The black table}
   \end{itemize}
\item Transformations are applied to these patterns:


\texttt{EN $\rightarrow$ ES}
\newline
\newline
 \begin{tabular}{ll}
   DET ADJ NOUN & $\rightarrow$ DET NOUN ADJ\\
   \emph{The black table} &  $\rightarrow$\emph{La ~~mesa ~~negra}\\
          sp  ~~sp ~~~~  sg     & ~~~f.sg ~f.sg ~~f.sg\\
\hline
   \emph{The black book} &  $\rightarrow$\emph{El ~~libro ~~negro}\\
          sp  ~~sp ~~~~  sg     & ~~~m.sg ~m.sg ~m.sg\\
  \end{tabular}

   \end{itemize}
\end{frame}


\begin{frame}
\frametitle{How the rules work}
\begin{itemize}
\item words can only be processed by ONE rule: rules can not be overlapped
\item patterns are matched following the LRLM principle (left-to-right-longest match)
\newline Example:
\begin{itemize}
  \item Rule 1: DET ADJ
  \item Rule 2: ADJ CONJ ADJ NOUN
   \item Input text (ru): \emph{Этот большой и шумный город}
\newline 
\newline 
\textbf{Which rule(s) will be applied?}
\end{itemize}
\end{itemize}
\end{frame}

\begin{frame}
\frametitle{How the rules work}
\begin{itemize}
\item words can only be processed by ONE rule: rules can not be overlapped
\item patterns are matched following the LRLM principle (left-to-right-longest match)
\newline Example:
\begin{itemize}
  \item Rule 1: DET ADJ
  \item Rule 2: ADJ CONJ ADJ NOUN
   \item Input text (ru): \emph{Этот большой и шумный город}
\newline 
\newline \textbf{Which rule(s) will be applied?}
\newline
\newline
\end{itemize}
\end{itemize}
\begin{block}{}
\begin{center}
\begin{tabular}{llllll}
 & Этот   & большой   &   и &    шумный   &  город\\
 & Det  &  Adj &   Conj  &   Adj &  Noun\\
\end{tabular}
\end{center}
\end{block}

\end{frame}

\begin{frame}
\frametitle{How the rules work}
\begin{itemize}
\item words can only be processed by ONE rule: rules can not be overlapped
\item patterns are matched following the LRLM principle (left-to-right-longest match)
\newline Example:
\begin{itemize}
  \item Rule 1: DET ADJ
  \item Rule 2: ADJ CONJ ADJ NOUN
   \item Input text (ru): \emph{Этот большой и шумный город}
\newline 
\newline \textbf{Which rule(s) will be applied?}
\newline
\newline
\end{itemize}
\end{itemize}
\begin{block}{}
\begin{tabular}{llllll}

 & \emph{Этот} & \emph{большой} &  \emph{и} & \emph{шумный} & \emph{город} \\
 & Det  &  Adj &   Conj  &   Adj &  Noun\\
\emph{left-to-right} & DET & ADJ &&&\\
\end{tabular}
\end{block}

\end{frame}

\begin{frame}
\frametitle{Structure of the rule file}
Structure of the rule file:
\begin{itemize}
\item \texttt{<section-def-cats>}
\item \texttt{<section-def-attrs>}
\item \texttt{<section-def-vars>}
\item \texttt{<section-def-lists>}
\item \texttt{<section-def-macros>}
\item \texttt{<section-def-rules>}

\end{itemize}
\end{frame}

\begin{frame}
\frametitle{Structure of the rule file /2}

\begin{itemize}
 \item \texttt{<section-def-cats>} $\rightarrow$ definition of categories. Categories will be used to define the pattern detected by a rule.
\begin{exampleblock}{}
\begin{small}
\begin{alltt}
  <def-cat n="noun">\\
  ~~<cat-item tags="n.*">\\
  </def-cat>\\
  <def-cat n="nounplural">\\
  ~~  <cat-item tags="n.pl"/>\\
  </def-cat>\\
\end{alltt}
\end{small}
\end{exampleblock}
Categories can be lexicalised:\\
\begin{exampleblock}{}
\begin{small}
\begin{alltt}
    <def-cat n="всегда">\\
     ~~ <cat-item lemma="всегда" tags="adv"/>\\
    </def-cat>\\
\end{alltt}
\end{small}
\end{exampleblock}
\end{itemize}
\end{frame}

\begin{frame}
\frametitle{Structure of the rule file /3}
\begin{itemize}
\item \texttt{<section-def-attrs>} $\rightarrow$ definition of attributes. Attributes are possible values of a certain feature.
\begin{exampleblock}{}
\begin{small}
\begin{alltt}
  <def-attr n="nbr">\\
  ~~<attr-item tags="sg">\\
  ~~<attr-item tags="pl">\\
  ~~<attr-item tags="sp"/>\\
  ~~<attr-item tags="ND"/>\\
  </def-attr>\\

<def-attr n="gen">\\
~~<attr-item tags="m">\\
~~<attr-item tags="f">\\
~~<attr-item tags="mf">\\
~~<attr-item tags="GD">\\
</def-attr>\\
\end{alltt}
\end{small}
\end{exampleblock}

 \item \texttt{<section-def-vars>} $\rightarrow$ definition of variables.

\end{itemize}
\end{frame}

\begin{frame}
\frametitle{Structure of the rule file /4}
\begin{itemize}
 \item \texttt{<section-def-lists>} $\rightarrow$ definition of lists of lemmas which will be used inside of the rules.
\begin{exampleblock}{}
\begin{small}
\begin{alltt}
  <def-list n="egunak">\\
  ~~ <list-item v="понедельник"/>\\
  ~~ <list-item v="вторник"/> \\
  ~~ <list-item v="среда"/> \\
  (...)
\end{alltt}
\end{small}
\end{exampleblock}
 \item \texttt{<section-def-macros>} $\rightarrow$ definition of macro instructions (parts of code to be reused in the rules).
 \item \texttt{<section-def-rules>} $\rightarrow$ definition of the transfer rules.

\end{itemize}

\end{frame}

%\begin{frame}
%\frametitle{Structure of the rule file /2}
%Minimal layout of a rule file:
%\begin{block}{}
% \begin{scriptsize}
% \begin{alltt}
%   <transfer>\\
%   ~<section-def-cats>\\
%   ~~ <def-cat n="adj\_or\_pp">\\
%   ~~~ <cat-item tags="adj.*">\\
%   ~~~ <cat-item tags="vblex.pp.*">\\
%   ~~ </def-cat>\\
%   ~~...\\
%   ~</section-def-cats>\\
%   ~<section-def-attrs>\\
%   ~~ <def-attr n="nbr">\\
%   ~~~ <attr-item n="sg"/>\\
%   ~~~ <attr-item n="pl"/>\\
%   ~~...\\
%   ~</section-def-attrs>\\
%   ~<section-def-vars>\\
%   ~~  <def-var n="number"/>\\
%   ~~ ...\\
%   ~</section-def-vars>\\
%   ~<section-rules>\\
%   ~~<rule comment="NOUN ADJ">\\
%   ~~ ...\\
%   ~</section-rules>\\
%   </transfer>\\
%    \end{alltt}
%    \end{scriptsize}
%\end{block}

%\end{frame}

\begin{frame}
\frametitle{Structure of a rule} % Structure of a rule

Rule = <pattern> + <action>
\begin{itemize}
\item Pattern: defined using the categories of the <section-def-cats>
\begin{exampleblock}{}
\begin{small}
 \begin{alltt}
   <rule comment="REGLA: DET ADJ">\\
   ~ <pattern>\\
   ~~	<pattern-item n="det"/>\\
   ~~	<pattern-item n="adj"/>\\
   ~     </pattern>\\
(...)
 \end{alltt}
\end{small}
\end{exampleblock}
\item Action: 
\begin{itemize}
\item specifies the tests and transformations to be made on the detected pattern
\item outputs the resulting lexical forms in the desired order
\end{itemize}
\end{itemize}
\end{frame}

\begin{frame}
\frametitle{Description of some tags}
Some tags used in the \texttt{<action>} part of the rules:
\begin{itemize}
  \item \texttt{<choose>}: start a test
  \begin{itemize}
  \item \texttt{<when>}: if the conditions specified are met...
    \begin{itemize}
    \item \texttt{<test>}: the condition to be tested
    \item \texttt{<otherwise>}: if the condition is not met...
    \end{itemize}
  \end{itemize}
\item \texttt{<let>}: assign a value
\item \texttt{<call-macro>}: call a macroinstruction defined in \texttt{<section-def-macros>}
\item \texttt{<out>}: output the lexical units
\begin{itemize}
\item \texttt{<lu>}: lexical unit, specified by means of one or more <clip>
\end{itemize}
\item \texttt{<clip>}: the basic unit of a lexical form.\\
\texttt{<clip pos="2" side="tl" part="lem"/>}\\
\texttt{<clip pos="2" side="tl" part="temps"/>}\\
\texttt{<clip pos="2" side="tl" part="whole"/>}\\

\end{itemize}

\end{frame}


\begin{frame}
\frametitle{The clip element}

Given the lexical form:
\begin{block}{}
\^{} врач\# общей практики<n><m><pl>\$
\end{block}

\begin{itemize}
\item \texttt{<clip pos="1" side="sl" part="whole"/>} : whole content of lexical unit
\item \texttt{<clip pos="1" side="sl" part="lem"/>} : lemma: \emph{врач\# общей практики}
\item \texttt{<clip pos="1" side="sl" part="tags"/>} : \texttt{<n><m><pl>}
\item \texttt{<clip pos="1" side="sl" part="lemh"/>} : lemma head : \emph{врач}
\item \texttt{<clip pos="1" side="sl" part="lemq"/>} : lemma queue : \emph{\# общей практики}

\end{itemize}

\end{frame}

\begin{frame}[fragile]
  \frametitle{Example rule}
\vspace{-6pt}
\begin{small}
\begin{alltt}
    <rule comment="regla: v\_{}abil">
      <pattern>
        <pattern-item n="v\_{}abil"/>
      </pattern>
      <action>
        <let><clip pos="1" side="tl" part="abil"/><lit v=""/></let>
        <out>
            <lu>
              <clip pos="1" side="tl" part="lem"/>
              <clip pos="1" side="tl" part="a\_{}verb"/>
              <lit-tag v="prt\_{}impf"/>
            </lu>
            <b/> 
            <lu>
              <lit v="ал"/><lit-tag v="v.tv"/>
              <clip pos="1" side="tl" part="neg"/>
              <clip pos="1" side="tl" part="a\_{}pers"/>
              <clip pos="1" side="tl" part="a\_{}num"/>
            </lu>
        </out>
      </action>
    </rule>

\end{alltt}
\end{small}

\end{frame}

\begin{frame}
  \frametitle{Практическая часть}

Now we'll try describing some rules, formalising them and then implementing them in 
Apertium's XML format.

\end{frame}

\end{document}
