\documentclass[free]{flammie}

\usepackage{polyglossia}
\usepackage{fontspec}
\usepackage{xunicode}
\usepackage{xltxtra}
\usepackage{url}
\usepackage{hyperref}
\usepackage{expex}

\setmainlanguage{finnish}

\begin{document}

\parskip 2mm
\parindent 0mm

\begin{titlepage}
  \setlength{\parindent}{0mm}
  \sloppy
  \large \textsc{Helsingin Yliopisto \\
                 Yleisen kielitieteen laitos \\
                 Kieliteknologia}
  \vspace{5mm}

  \hrule height3pt
  \vspace{20mm}

  \begin{center}
    \large Pro gradu -tutkielma
    \linebreak \vfill

    \huge \textbf{Suomen kielen äärellistilainen morfologinen jäsennin avoimen
lähdekoodin resurssein}
    \vspace{20mm}

    \Large Tommi A Pirinen \linebreak
    \normalsize 013160681  % opiskelijanumero
    \vfill

  \end{center}
  \hrule height2pt
  \vspace{15mm}

  Ohjaaja: Kimmo Koskenniemi ja Krister Lindén
  \hfill
  26.4.2008 (korjattu viimeksi 13.10.2017, generoitu
    \today)    % päivämäärä (esim. 2.2.2002)
\end{titlepage}

\tableofcontents

\pagebreak

\section*{Esisanat}

Tämä on pro graduni modernisoitu versio: kadotin alkuperäisen tex-lähdekoodin
jossain vaiheessa, joten olen kirjoittanut sisällön uusiksi lähinnä
automaattisen konversion avulla. Olen myös korjannut alkuperäisversiosta
lähinnä ''painovirheitä'', kuten eräs vahingossa IRC:stä copypastettu teksti.
Korjaukset olen merkinnyt tavalla tai toisella alaviitteisiin. Sivunumerointi
ei ole enää alkuperäinen, lähes kaikki asettelut, fontit ja muut on päivitetty
modernimpaan suuntaan ja myös avoimiksi, siis esimerkiksi Times New Romanin
asemesta käytetään täysin vapaata Liberation Serif -fonttia.

Tämä dokumentti muutoin on lähinnä historiallisesti mielenkiintoinen. Jotkin
siinä esitetyistä ideoista näkynevät vielä tämän päivän kieliteknologiassa,
kuten omorfin tällä hetkellä
githubissa\footnote{\url{https://github.com/flammie/omorfi/}} saattaa huomata.
Monet esitetyistä ideoista, kuten kotikutoinen XML-muoto ja SFST-skriptikieli,
ovat jo jääneet historiaan vähemmän kestävinä ratkaisuina.

---Tommi A Pirinen, Hampurissa 7.11.2016


\newpage

\footnote{Tämä sivu ei näytä lainkaan alkuperäiselt Word-liitteeltä, joka oli
erikseen lisätty graduun.}

\begin{tabular}{|l|l|}
\hline
Tiedekunta/Osasto Fakultet/Sektion – Faculty &
Laitos Institution – Department \\
\bf Humanistinen tiedekunta & \bf Yleinen kielitiede \\
\hline
\end{tabular}
\begin{tabular}{|l|}
\hline
\bf Tekijä - Författare – Author \\
Tommi A Pirinen \\
\hline
\bf Työn nimi - Arbetets titel – Title \\
Suomen kielen äärellistilainen morfologinen jäsennin avoimen lähdekoodin
resurssein \\
\hline
\bf Oppiaine - Läroämne – Subject \\
Kieli-, puhe- ja käännösteknologian maisteriohjelma \\
\hline
\end{tabular}
\begin{tabular}{|l|l|l|}
    \hline
    \bf Työn laji - Arbetets art – Level &
    \bf Aika - Datum – Month and year &
    \bf Sivumäärä - Sidoantal – Number of pages \\
    Pro gradu – Master’s thesis &
    29.4.2008 – April 2008 &
    64 \\
    \hline
\end{tabular}
\begin{tabular}{|p{0.98\linewidth}|}
    \hline
    \bf Tiivistelmä – Referat – Abstract \\
    Tutkielma on kuvaus suomen kielen automaattisen äärellistilaisen morfologisen
    jäsentimen toteutuksesta avoimen lähdekoodin menetelmin ja resurssein.
    Tutkielmassa kuvataan Kotimaisten kielten tutkimuskeskuksen julkaiseman
    Nykysuomen sanalistan sisältöä morfologisen jäsentimen toteuttamisessa.
    Äärellistilaisista menetelmistä tutkielma tarkastelee SFST-nimisen ohjelman
    käyttämistä jäsennintransduktorin rakentamisessa. Lisäksi tutkielmassa
    esitetään toteutetun järjestelmän testausmenetelmä valmiin morfologisesti
    jäsennetyn korpuksen avulla. \\
    \hline
\end{tabular}
\begin{tabular}{|l|}
    \hline
    \bf Avainsanat – Nyckelord – Keywords \\
    Äärellistilaiset menetelmät, morfologia, suomen kieli \\
    \hline
    \bf Säilytyspaikka – Förvaringställe – Where deposited \\
    Humanistisen tiedekunnan kirjasto / Yleisen kielitieteen laitos \\
    \hline
    Muita tietoja – Övriga uppgifter – Additional information \\
    \hline
\end{tabular}

\section{Johdanto}

Kieliteknologia on monitieteinen ala, joka käsitelee kielitieteellisten
teorioiden soveltamista tietojenkäsittelytieteiden menetelmin. Tässä
tutkielmassa keskitytään kieliteknologian osa-alueeseen, joka käsittelee sanojen
automaattista jäsentymistä ja käsittelyä. Tutkielmassa kuvaan kieliteknologisen
suomen kielen sanoja jäsentävän sovelluksen toteutusta äärellistilaisin
menetelmin. Sovelluksen erääksi kehitystavoitteeksi on valittu
kokonaisjärjestelmän lisenssiehtojen avoimuus ja vapaus. Tutkielmassa kuvaan
avointen ja vapaasti saatavilla olevien resurssien käyttöä äärellistilaisen
morfologisen jäsentimen rakentamisessa.

Morfologia viittaa tässä tutkielmassa
laveasti luonnollisen kielen sanojen muotoopilliseen rakenteeseen ja sen
mallintamiseen. Muoto-opillisen rakenteen osayksiköitä nimitetään konkreettisina
instansseina morfeiksi, ja näiden perusteella johdettuina abstrakteina malleina
morfeemeiksi. Esimerkiksi jos puhutaan suomen kielen sanamuodosta taloissa1 , on
siinä juurimorfi talo, monikon tunnusmorfi i ja inessiivin tunnusmorfi ssa.
Morfeemeina voisi pitää vaikkapa esitystä talo, I ja ssA, jossa I ja A2 kuvaavat
abstrakteja morfofoneemeja, jotka toisissa tilanteissa voivat olla vaikkapa i ja
a yhtä hyvin kuin j ja ä vastaavasti vokaalisoinnun tai vastaavan
äännemuunnoksen vaikutuksesta. Tämä luokittelu ja kuvaus vastaa läheisesti sitä,
joka on esitetty suomen kielestä mm. teoksessa Karlsson (1982).

Yleensä
kielitieteissä sanan jako morfeemeiksi tapahtuu jakamalla sana pienimmiksi
semanttista sisältöä kuvastaviksi yksiköiksi (Karlsson, 1998). Jokaisella
morfeemilla on oma merkityksensä sanamuodon taloissa kokonaismerkityksen
kannalta: talo merkitsee reaalimaailman talo-tyyppistä oliota, i merkitsee
monikkoa eli sitä että olioita on monta ja ssa merkitsee viittausta talojen
sisässä olemiseen. Tutkielmassa kuvatussa järjestelmässä relevanttia on siis se,
että taloissa, on talo-sanan monikon inessiivi. Muihin asioihin kuten sanan
rakenteen merkitykseen tai äännerajaukseen lopullinen järjestelmä ei ota kantaa.

Gradun terminologiassa sanan kokoisiin yksiköihin joudun viittamaan monesta eri
näkökulmasta, joten olen yrittänyt tehdä tarkan jaon sanan eri merkityksille.
Sanalla ’sana’3 viittaan yhteen sanakirjasanaan sen kaikissa muodoissa,
esimerkiksi sanaan paperi. Yhdellä sanalla voi olla useita sanamuotoja, kuten
sanalla paperi taivutusmuodot papereiden ja papereitten. Sanamuodot papereiden
ja papereitten vastaavat samaa morfologista sanamuotoa, monikon genetiiviä, ja
niitä kutsutaan tämän muodon allomorfeiksi. Juoksevia tekstejä käsitellessäni
jokaista sanamuodon esiintymää kutsun saneeksi.

Tutkielmassa kuvaamani toteutus
on suomen kielen muoto-opillinen jäsennin, tarkalleen ottaen sen eräs varhainen
kehitysversio. Suomen kielen äänne- ja muoto-opin kannalta sanojen taivutuksessa
esiintyy monia rinnakkaisia vaihteluja sekä sanavartalossa että tunnuksissa.
Esimerkiksi sanan yö monikkovartaloksi sanamuodossa öissä voisi laskea morfin ö,
joka poikkeaa yksikön taivutusvartalosta yö. Vastaavasti esiintyvän inessiivin
tunnuksen ssä-muoto johtuu sanavartalon vokaalien etisyydestä, muussa
tapauksessa se voisi olla ssa. Tällä tarkoitan, että suomen kielen sanojen
taivutuksessa esiintyy ainakin sekä juurimorfeemin äännevaihtelua taivutuksen
yhteydessä että taivutusmorfeemin äännevaihtelua juuren äännerakenteen
vaikutuksesta.

Äärellistilaisilla menetelmillä (finite-state methods) tarkoitan
tämän tutkielman piirissä kieliteknologisia sovelluksia, jotka käsittelevät
syötteenään merkkijonoja kappaleessa 3.1 tarkemmin kuvatun mekaniikan
mukaisesti. Äärellistilaisista olioista puhun lähinnä transduktorista
(transducer), joka on käytännön tasolla sellainen mekanismi, joka ottaa
syötteekseen merkkijonon, tunnistaa sen ja muuntaa toisiksi vastineiksi.
Äärellinen transduktori on nimenomaan morfologian kannalta käyttökelpoinen,
sillä niillä voi kuvata sanan sanakirjamuotojen ja taivutusmuotojen välisen
suhteen suoraan, ja tämä kuvaus on valmiiksi kaksisuuntainen siten, että se osaa
muuntaa sanan ja taivutusmuotojen tunnisteet sanamuodoksi sekä päinvastoin.
Käytännössä siis eräs yksinkertainen transduktori olisi sellainen, jossa on
kuvaus talo SG INE ↔ talossa4 . Tässä tutkielmassa esitellään transduktori, joka
kuvaa tutkielman sanalähteenä olevan sanalistan kaikki sanat kaikille suomen
kielen säännöllisen taivutusopin sanamuodoille.

Avoin lähdekoodi (open source)
on valittu keskeiseksi perusteeksi tutkielmassa käytettävien aineistojen
lisensointiin, sillä on pidetty ongelmana, että saatavilla olevat suositut
resurssit, kuten äärellistilaisia järjestelmiä käsittelevät koodit tai
sanakirjalähteet, ovat tarkoin määritellyin lisenssein kaupallisia tai
suljettuja ohjelmia. Tämä hankaloittaa niiden yleistä kehitettävyyttä sekä
niiden avulla käsitellyn materiaalin vapaata jakelua (ks. esim. Yli-Jyrä
(2005)). Avoimella lähdekoodilla tässä viittaan ensi sijassa sellaisiin
resursseihin, joiden sisältö on vapaasti saatavilla, mukautettavissa ja
edelleenlevitettävissä lisenssinsä perusteella.

Järjestelmän toteuttamiseen tarvittavia äärellistilaisia ohjelmistoja on
saatavilla runsaasti. Kuvaan tässä lyhyesti ominaisuuksien valintaa ja niiden
relevanssia tutkielman sekä siinä toteutetun järjestelmän kannalta, minkä
jälkeen tarkennan asiaa valitun järjestelmän kannalta. Liitteessä A on taulukko,
joka sisältää tietoja kaikista kirjoitushetkellä löydetyistä ohjelmistoista
(Yli-Jyrä, 2007), sikäli kun ne ovat löydettävissä, tarkastettavissa tai muuten
tunnetut.

Koska tutkielman tavoitteena oli alusta loppuun rakentaa avoin ja vapaa
kokonaisjärjestelmä, ohjelmistojen lisenssiehdoista on otettu huomioon ensi
sijassa avoimuus. Tällä perusteella myös voi sinänsä sivuuttaa kaikki suljetut
ohjelmistot irrelevantteina. Tyypillisimpiä avoimen lähdekoodin lisenssejä ovat
GNU-projektin (GNU is Not Unix) GPL- (GNU General Public Licence) ja
LGPL-lisenssit (GNU Lesser General Public Licence), joita käsiteltäessä on
pantava merkille avoimuuden lisäksi muitakin rajoituksia tai mahdollisuuksia5 .
GNU-lisenssejä pidän ensisijaisina avoimina ja vapaina lisensseinä tässä
selvityksessä.

Sovellusten toiminnallisista ominaisuuksista on valitussa järjestelmässä pidetty
tärkeänä toisinkirjoitus- eli replace-sääntöjen kirjoittamisen mahdollisuutta,
ja replace-sääntöjä on käytetty toteutuksen perustana. Replace-säännöstä kerron
tarkemmin kappaleessa 3.1. Käytännössä kyseessä on operaatio, jolla voi
yksinkertaisesti ja havainnollisesti mallintaa yleisiä suomenkin kielen
morfofonologisia vaihteluita. Esimerkiksi muutamalla tällaisella säännöllä voi
mallintaa taivutuspäätteen sointuvokaalin valintaa suhteessa sitä edeltävään
lähimpään sointuvokaaliin. Ne mahdollistavat tavan ilmaista helpohkosti, että
jos vasemmasta kontekstista löytyy takainen sointuvokaali ennen lähintä
sananrajaa, käytetään takaista sointua, muutoin etistä. Yksi tällainen sääntö
voisi olla yksinkertaistaen vaikkapa \verb|A:ä → ([äöy] __)|: sointuvokaalista
tulee ä jos vasemalla on ä, ö tai y.

Tutkielmassa toteutustyökaluksi on valittu SFST-ohjelmisto, joka on GNU GPL
-lisensoitu ja sisältää replace-säännön sellaisen toteutuksen, joka on
suomenkielisen morfologisen jäsentimen toteuttamiseen riittävä. SFST:tä ei
aiemmin ole käytetty kuin muutamissa saksan kielen taivutuksen ja johto-opin
morfologiaa kuvaavissa järjestelmissä (Schmid, 2005), mutta hyötypuolena
järjestelmää ja toteutusta verrattaessa muihin saatavilla oleviin, oli mukana
edes jonkinlaiset ohjeet ja kuvaukset morfologian toteuttamisesta (Schmid,
2007a,b). Joitakin sekä lisenssiltään että toiminnoiltaan soveltuvia
ohjelmistoja, kuten PC-KIMMOa KGENillä tai mmorphia ei ole tarkasteltu, sillä
niiden saaminen toimimaan modernilla kääntäjillä ja järjestelmäkokoonpanoilla
osoittautui hankalaksi tai mahdottomaksi ilman merkittäviä muutoksia koodiin.
Mitä tulee OpenFST:hen, joka on mm.  AT\&T:n FSM:n tekijöiden julkaisema avoin
ja vapaa FST-toteutus, tiedon siitä sain vasta syksyllä 2007, jolloin oma
morfologiani SFST:llä oli jo lähes valmis, ja se on jätetty siitä syystä pois
laskuista. Jatkotutkimusta varten kuitenkin OpenFST lienee varsin
kehityskelpoinen aihe.

Vertailukohdaksi suorituskykytestauksessa on valittu kaupallinen ja
tutkimuskäyttöön rajattu AT\&T:n FSM-järjestelmä, joka lienee ainakin tunnettu
ja käytetty. Se oli myös saatavilla suorituskykytestauksia varten, ja muunnos
käyttämästäni järjestelmästä sille sopivaksi testaustarkoituksiin ei ollut kovin
hankala.

Toinen merkittävä resurssi, jonka tässä automaattisen morfologisen jäsentimen
toteutuksessa tarvitsin, oli leksikko. Tätä tarkoitusta varten suomenkielinen
sanalista, Kotimaisten kielten tutkimuskeskuksen GNU LGPL -lisenssillä
julkaisema Nykysuomen sanalista (Kotimaisten Kielten Tutkimuskeskus, 2006), oli
juuri julkaistu tutkielman teon vaiheessa ja sopi loistavasti tutkittavaksi
materiaaliksi.  Nykysuomen sanalistan muoto, joka on seikkaperäisemmin kuvattu
kappaleessa 2, määritteli myös luontevasti rajat ja tyylin tekemälleni
järjestelmälle, sillä siinä sanat on luokiltaan jaettu perinteisen
sanakirjaluokituksen mukaisesti, joka jotakuinkin kuvaa morfofonologisia
piirteitä sanoissa, ja jota olen jokseenkin tarkasti seurannut järjestelmässäni.
Toteutus toimii siis sillä perusolettamuksella, että tietyllä numerolla olevassa
taivutusluokassa sana taipuu aina saman kaavan mukaisesti, ja tällainen kaava
väistämättä pätee kaikkiin luokan sanoihin. Sanalistassa ei toisaalta ole
eroteltu esimerkiksi adjektiiveja substantiiveista, jonka perusteella rajoitin
myös adjektiivitaivutuksen järjestelmän ulkopuolelle. Toteutin myös
testaustarkoituksiin yligeneroivan yhdyssanamuodostuksen, joka yhdistelee
taivutettuja nomineja melko mielivaltaisesti. Sanojen johtamista ei myöskään ole
toteutettu verbien yleisempiä infiniittimuotoja enempää.

Kolmantena resurssina järjestelmän testaukseen tarvitaan saneita oikeine
morfologisine tulkintoineen. Tällaisia valmiiksi jäsennettyjä tekstiaineistoja
kutsutaan myös korpuksiksi. Sopivia korpuksia ei ole vapain lisenssein, joten
tutkin tutkielmassani rajoitetummin lisensoidun automaattisesti jäsennetyn
tarkistamattoman korpusmateriaalin käyttöä kertaluontoisen testauksen
toteuttamisessa. Lisensoinnin takia testaus jouduttiin suorittamaan sivullisella
palvelimella eikä tuloksia ole mahdollista hyödyntää vertailua laajemmin niiden
lisenssiehtojen puitteissa.  Kokonaisuutena graduni on selvitystyö, johon yritin
koota tietoa seuraavista:

\begin{enumerate}
    \item  morfologiajärjestelmän luonnosteleminen, jos saatavilla on jo luokiteltu sanalista, sekä tunnettu säännöllinen taivutusjärjestelmä
    \item Nykysuomen sanalistan datan käyttö koneellisen suomen kielen morfologian toteutukseen
    \item Nykysuomen sanalistan taivutusluokkien käyttö koneellisessa morfologiajärjestelmässä
    \item Nykysuomen sanalistassa käytetyn XML-formaatin käyttö morfologisen järjestelmän osana
    \item SFST:n käyttö suomen kielen morfologisen järjestelmän teossa
    \item SFST:n suorituskyky verrattuna AT\&T:n FSMlib-järjestelmään
    \item morfologisen järjestelmän testaaminen korpusaineistolla, joka on morfologisesti jäsennetty.
\end{enumerate}

\footnote{1
    Käytän tutkielmassa kursivointia viittamaan sanoihin, sanamuotoihin, saneisiin,
morfeihin ja morfeemeihin lingvistisinä olioina niiden merkityksen asemesta}
\footnote{
    2
Käytän tutkielmassa suuraakkosia kuvaamaan abstrakteja morfofoneemeja.
Tavallisesti kuvauksissa vastaavat A, O ja U vokaalipareja a ja ä, o ja ö sekä u
ja y, eli sointuvokaalien pareja, koska ne käyttäytyvät lähes poikkeuksetta
samoin vokaalisointua lukuunottamatta ja merkintätapa on melko vakiintunut.
Vastaavasti merkitään myös V:llä mitä tahansa vokaalia ja C:llä konsonanttia.
Tapauksittain myös kirjaimia K, P, T, G, B ja D käytetään kuvaamaan
astevaihtelun alaisia klusiileja sekä kirjainta L likvidojen joukkoa}
\footnote{3
muualla
käytetään myös esim. nimitystä lekseemi tai hakusana
}

\footnote{
4

Merkitsen tutkielmassa koodijärjestelmien merkintöjä, kuten tunnisteita, nimiä ja varattuja
sanoja tasavälisellä fontilla.
}
\footnote{
5

GNU-lisensseistä ja käytöstä sekä periaatteista tarkemmin ks. http://www.gnu.org/
philosophy/philosophy.html.
}


Tutkielma jakautuu selvitykseltään kahteen pääteemaan. Ensin käydään läpi
käytössä olevien resurssien ja menetelmien sisältö ja käyttötarkoitukset.
Toiseksi tutkitaan näin resurssein valmistamani järjestelmän toimintaa
testaamalla sitä erilaisin aineistoin ja suorituskykytestein ja analysoimalla
havaitut puutteet ja ongelmat. Ensimmäinen osa rakentuu tutkielmassani siten,
että kappaleessa 2 selvitetään Nykysuomen sanalistan sisältöä siltä kannalta,
miten sitä käytetään morfologisessa jäsentimessä. Selvityksessä käydään läpi
yleinen morfofonologinen tieto suomen kielen taivutuksesta ja se, miten se
suhtautuu sanalistassa olevaan luokitukseen. Tämän jälkeen kappaleessa 3
kerrotaan äärellistilaisista menetelmistä morfologisen jäsentämisen käytössä.
Kappaleessa selvitetään miten kappaleessa 2 selvitetty tieto sanalistan
taivutusluokituksesta on muunnettavissa äärellistilaisen järjestelmän käyttöön.
Toinen osa koostuu testiaineistojen ja -asetelmien kuvauksista sekä tuloksista
kappaleessa 4, jonka jälkeen kappaleessa 5 analysoidaan koko järjestelmää ja sen
testituloksia sekä tältä pohjalta asetetaan mahdollisia tulevaisuuden
kehityslinjoja ja tutkimusideoita.

\section{Aineistot}

Automaattinen morfologinen jäsennin, jollaista tutkielmassa rakennetaan,
perustuu sanalistaan. Tällaiseksi sanalistaksi on valittu Kotimaisten kielten
tutkimuskeskuksen, eli Kotuksen, julkaisema Nykysuomen sanalista, jonka
tutkielmassa käytetty julkaistu ensiversio sisältää 94 110 sanaa. Sanoista 44
348:aan on merkitty tieto taivutusluokista sekä mahdollisesta astevaihtelusta,
jota sana suomen kielen perussanakirjan taivutusluokittelussa käyttää
(Kotimaisten Kielten Tutkimuskeskus, 2006). Loput luokittelemattomat sanat ovat
yhdyssanoja, joiden taipuva osa on luokiteltu sanalistassa; nämä sanat olen
jättänyt tutkielmassani toistaiseksi huomiotta.

Nykysuomen sanalista on
tallennettu XML 1.06 -määrittelyn mukaisen merkkauskielen sovelluksella.
XML-kielen käyttö tarkoittaa että tiedosto on pohjimmiltaan tekstipohjainen
puuta kuvaava datarakenne, jonka osaset on merkattu kulmasulkeisiin rajatuin
tunnistein eli merkkauksin. Sanalistassa se tarkoittaa pelkistäen, että jokaista
sanaa kohti on eroteltu vähintään tiedot sanan sanakirjamuodosta,
taivutusluokasta ja mahdollisesta astevaihteluluokasta. Tyypillinen sanatietue
sisältää siis osat <s>sanakirjamuoto</s>, <tn>taivutusluokka</tn> ja
<av>astevaihtelukirjain</av>. Kokonaisia esimerkkejä on listauksessa 1.

Ensimmäinen toteutuksen kannalta olennainen asia Nykysuomen sanalistan
muuttamisessa kohti morfologista jäsennintä on selvittää mitä tietoja
sanalistasta jäsennin tarvitsee voidakseen kuvata sanakirjamuodot
taivutusmuodoiksi. Lähtökohta toteutuksessani on, että sanalistassa käytetty
sanaluokitus on itsenäisesti riittävä kuvaamaan kaikki tai lähes kaikki
morfofonologinen vaihtelu, joka sanojen taivutuksen kuvauksessa on tarpeellista.
Seuraavaksi selvitetäänkin sanalistassa käytetyn luokituksen periaatteet ja
miten ne muuntuvat äärellistilaisen jäsentimen käytettäväksi.

Nykysuomen
sanalistan data — sanat ja niiden luokitukset — suurimmilta osin soveltuvat
sellaisenaan lopullisen järjestelmän käyttöön, eikä niitä tarvinnut jakaa,
yhdistellä tai muuttaa, joten kappaleessa 2.1 kuvaan tämän alkuperäisen
XMLrakenteen osineen. Muutamat muutokset ja päivitykset, jotka olen tehnyt
sanalistaan on myös selostettu. Tämän jälkeen kappaleessa 2.2 kuvaan mitä
sanalistan sisältämä data tarkoittaa käytännössä lingvistiseltä kannalta, eli
mitä kaikkea sellaista käytetyn sanaluokituksen luokat kuvaavat, mikä
morfologisen jäsentimen tulee ottaa huomioon.

\footnote{
6

http://www.w3.org/TR/2006/REC-xml-20060816/
}
\footnote{8

Käytän tässä Ison suomen kieliopin (Hakulinen et al., 2004) termejä, sillä erotuksella, että kirjoitan tunnusten nimet pienaakkosin, eli nut- pro NUT-partisiippi. Vanhastaan kieliopeissa käytetty
muoto on I partisiippi tai menneen ajan partisiippi}
\footnote{
9
Tässä tutkielmassa produktiivisuuden määritelmänä käytetään vain natiivin kielenpuhujan intuitiota morfeemin esiintymän tuoreudesta. Todellisuudessa produktiivisuuden määrittäminen on
hankalampi kysymys, jota on tutkittu jonkin verran. Tutkielman kannalta tärkeää on huomata, että
produktiivisiksi merkityt morfologiset prosessit ovat sellaisia, joita esiintyy paljon, ja epäproduktiiviset sellaisia, joita ei ole löydettävissä kuin muutama laajoistakin aineistoista.
}




\subsection{Nykysuomen sanalistan XML-muoto}

Nykysuomen sanalista on yksinkertainen XML-muotoinen tiedosto. Se sisältää yhden
monialkioisen listan, jonka jokaisen alkion sisältönä on datarakenne
sanatietue-alkiossa st, johon kuuluu sana sanakirjamuodossaan alkiona nimeltä s.
Toinen osa sanan sisältävää datarakennetta on taivutustiedot alkiossa t, johon
kuuluu sanan taivutusluokka alkiona tn sekä mahdollinen astevaihtelutieto
alkiona av. Taivutustietoja per sana voi olla useampia kuin yksi, mutta vain
erikoistapauksissa (harvinainen vanhempi taivutus tms., jolloin t-alkiolla myös
on selittävä vakioitu tekstimuotoinen attribuutti), sillä homonyymit on merkitty
eri sanoiksi, ja niille on datarakenteessa erillinen alkio hn, joka on juokseva
homonyyminumero.

Kuvasta 1 näemme esimerkkejä miten erilaiset sanatyypit
kuvataan tässä XMLmuodossa. Kuvaan tässä XML-rakenteeseen kuuluvat alkiot siinä
järjestyksessä, jossa ne tulevat esille. Juuren muodostaa XML-alkio
kotus-sanalista, jolla
ei ole attribuutteja, ja sisältönä on nolla tai useampia sanatietue-alkioita, eli se on
yksinkertainen listan sisällyttävä rakenne.

Sanatietueita ovat st-alkiot, eikä niilläkään ole mahdollisia attribuutteja. Sisältönään jokaiseen sanatietueeseen kuuluu tasan yksi sana-alkio, nolla tai yksi homonyymialkiota ja miten monta tahansa taivutustietoalkiota.

Sana-alkiolla, jonka nimi on s, ei ole mahdollisia attribuutteja, ja sen sisältönä on
pelkkää tekstiä. Tekstisisällöksi kuuluu sana sanakirjamuodossaan. Tämä tarkoittaa tyypillisesti nomineilla yksikön nominatiivia ja verbeillä a-infinitiivin latiivia.
Poikkeuksia muodostovat esim. nomineista monikkosanat, joiden sanakirjamuotona käytetään monikon nominatiivia (esim. häät, sakset), ja verbeistä sanat, joilla
a-infinitiivin latiivia ei käytetä (esim. erkanee pro ? erata).

Homonyymialkiolla hn ei ole attribuutteja. Sen sisältönä on positiivinen kokonaislukuarvo, joka ei saa olla sama kuin yhdelläkään toisella saman sanan (so.
sellaisen, jonka s-alkion sisältö on sama) sanatietueella. Käytännössä siis sanalistassa homograafit numeroidaan juoksevalla tunnisteluvulla.

Taivutusalkiolla t on valinnainen attribuutti @taivutus, jonka arvoina
on sanallinen selitys taivutusalkion kuvaaman taivutuksen poikkeusluonteesta. Nykysuomen sanalistan versiossa 1 attribuutissa käytetyt arvot olivat
”harvinainen” ja ”mahdollinen” kuvaamaan vaihtoehtoisten taivutusten
yleisyyttä sekä ”yksikössä” ja ”monikossa”, jolla kuvattiin sanan kolme
taivutustapaa. Omassa koeversiossani laajensin attribuuttiarvoa monikossa koskemaan monikkosanojen t-alkiota. Taivutusalkion sisältönä on tasan yksi taivutusnumeroalkio, sekä nollasta yhteen astevaihtelualkioita.

Taivutusnumeroalkio tn:llä ei ole attribuutteja, ja sen sisältönä on positiivinen
kokonaisluku, joka on yksi kappaleessa 2.2 kuvatuista taivutusluokista.

Astevaihtelualkiolla av on attribuutti astevaihtelu, jonka ainoa arvo sanalistassa on valinnainen, niille sanoille joita voi taivuttaa sekä astevaihtelullisena että -vaihteluttomana. Alkion sisältönä on suuraakkonen alueelta A—M, joka
kertoo astevaihtelun alaisena olevan äänteen ja sen heikon asteen vastineen.

Taivutusnumero ja astevaihtelukirjain yhdessä sanan sanakirjamuodon kanssa
muodostavat sen keskeisen datan, johon nojaten olen jäsentimeni rakentanut.

Näiden lisäksi olen muokannut hieman formaattia siten, että on mahdollista kuvata sanan sanakirjamuodon on monikollisuus (esim. sakset tai muut pluratiivit),
sanan vokaalisoinnun tai vartalovokaalin odotuksenvastaisuus (esim. bordeaux ja
monet vierassanat), ja sanan taivutuksen odotuksenvastaisuus (esim. olla). Näistä
ja muista kehitysideoista tarkemmin 5. kappaleessa.


\begin{figure}
    \label{Otos Nykysuomen sanalistan XML-datasta (Kotimaisten Kielten
    Tutkimuskeskus, 2006)}
    \begin{verbatim}
<st><s>aloitteikas</s><t><tn>41</tn><av>A</av></t></st>
<st><s>-aloitteinen</s><t><tn>38</tn></t></st>
<st><s>aloittelija</s><t><tn>12</tn></t></st>
<st><s>aloitus</s><t><tn>39</tn></t></st>
<st><s>aloituskorkeus</s></st>
<st><s>aloitusmerkki</s></st>
<st><s>aloituspaikka</s></st>
<st><s>aloitussyöttö</s></st>
<st><s>aloitusviisikko</s></st>
<st><s>alokas</s><t><tn>41</tn><av>A</av></t></st>
<st><s>alokasaika</s><t><tn>9</tn><av>D</av></t></st>
<st><s>alokasaste</s></st>
<st><s>alokasmainen</s><t><tn>38</tn></t></st>
<st><s>aloke</s><t><tn>48</tn><av>A</av></t></st>
<st><s>alpakka</s><hn>1</hn>
<t><tn>14</tn><av>A</av></t></st>
<st><s>alpakka</s><hn>2</hn>
<t><tn>14</tn><av>A</av></t></st>
<st><s>alpakkainen</s><hn>1</hn><t><tn>38</tn></t></st>
<st><s>alpakkainen</s><hn>2</hn><t><tn>38</tn></t></st>
<st><s>alpakkalusikka</s></st>
<st><s>alpi</s><t><tn>7</tn><av>E</av></t>
<t taivutus="harvinainen"><tn>5</tn></t></st>
    \end{verbatim}
\end{figure}


\subsection{Nykysuomen sanalistan sisältö: sanaluokitus käytännössä}

Tässä kappaleessa pyrin selvittämään Nykysuomen sanalistassa käytetystä taivutusluokituksesta kaiken sen tiedon, jota lopullisessa morfologisessa jäsentimessä
on käytetty. Kyseessä on käytännöllinen kuvaus siitä, miten taivutusluokittelua
lopulta päädyin käsittelemään järjestelmässä, ei niinkään lingvistinen kuvaus taivutusluokituksen sisällöstä. Paikoin kuvauksissa on annettu viitteitä relevantteihin lähteisiin suomen kielen äänne- ja muotohistoriasta, mutta lähinnä niiltä osin
kuin annettu lisätieto riittää motivoimaan järjestelmässä esiintyvän äännevaihtelun. Systemaattiset selvitykset luokituksesta kattavat vain näkyvissä olevat konkreettiset muutokset sanojen kirjoitusmuodossa sikäli kuin ne tämän järjestelmän
toteutuksessa ovat olleet tarpeellisia.

Nykysuomen sanalistan sanat on luokiteltu saman luokittelun mukaisesti kuin mitä on käytetty Suomen kielen perussanakirjassa ja Kielitoimiston sanakirjassa.
Nomineille tarkoitettuja luokkia on 49 (1—49) ja verbeille 27 (52—78). Lisäksi sanalistassa on yksi luokka kaikille adverbeille, adpositioille, lyhenteille ja partikkeleille (99), yksi pronomineille (101) ja kaksi yhdysnomineille (50 ja 51), joita ei tämän tutkielman morfologian toteutuksessa lainkaan käsitellä. Sanaluokitus
perustuu vanhempaan Nykysuomen sanakirjan luokitukseen siten, että luokkien
määrää on vähennetty fonologisin perustein (Eronen, 1994).

Sanaluokitus vaikuttaa enimmäkseen käytännössä toteutetun siten, että numerosta käy ilmi sekä sanalle kuuluvat potentiaaliset allomorfit että vartalossa esiintyvät äännevaihtelut, toisin sanoen jos yhdenlaisista perusmuodon vartalon äännevaihtelusta koostuvalla luokalla esiintyy jonkin taivutuspäätteen kohdalla erilaista
allomorfien jakaumaa sen pitäisi aina saada oma taivutusluokkansa. Poikkeuksena päätevaihtelun tuomista taivutusluokista on vokaalisoinnun aiheuttamat muutokset, joita ei ole merkitty taivutusluokiksi, ja joka pitää tässä huomioida morfologian toteutuksessa sääntöpohjaisesti. Vastaavasti vartalon äännemuutoksista
astevaihtelu ei tuo uusia taivutusluokkanumeroita, sillä se on kuvattu erillisenä
komponenttina taivutusluokitusta astevaihtelukirjaimella. Tästä seuraten siis sanat hyytö ja huuto ovat molemmat luokassa 1, vaikka yksikön inessiivin tunnukset
ssä ja ssa poikkeavat, kun taas sanat valo ja valtio taas luovat eri luokat 1 ja 2,
koska monikon genetiivin tunnusten jen ja iden poikkeukset eivät selity vokaaliharmoniasäännöllä.

Taivutusluokitus on toteutetussa järjestelmässä analysoitu eli eroteltu piirteiksi
niin, että taivutusluokkanumeron oletetaan kuvaavan:

\begin{itemize}
\item taivutuksessa vartaloon kohdistuvat äännemuutokset
\item taivutustunnusten allomorfien valinnat niissä taivutuspäätteissä, joissa esiintyy taivutusluokittaista vaihtelua
\item astevaihtelun tyypin (suora vai käänteinen astevaihtelu)
\item perusmuodon äännerakenteen.
\end{itemize}

Astevaihtelukirjain taas kuvaa suoraan astevaihtelun alaisen äänneparin, josta yhdessä astevaihteluluokan kanssa on selvitettävissä kirjain, joka vastaa astevaihtelun käyvää äännettä sanan sanakirjamuodon kirjoitusasussa. Astevaihtelun kuvauksesta erillisenä yhdistelmällisenä piirteenä luokituksessa seuraa myös, että
astevaihtelun vaikutusta äännerakenteeseen ja vartalon äännemuutoksiin ei oteta
lukuun niitä kuvatessa.

Seuraavassa käyn läpi taivutusluokkia yksi kerrallaan siten, että selvitän jokaisesta perusmuodon äännerakenteesta luokalle yhteisen osan, joka tässä tarkoittaa sanan perusmuodon loppua, jokaisen allomorfista vaihtelua sisältävän päätetyypin
sallitut allomorfit, sekä taivutuksessa sanan vartalossa esiintyvät äännemuutokset.
Näitä ei ole tietääkseni missään erikseen kuvattu tämän taivutusluokituksen suhteen, joten kuvaukseni perustuu toisaalta tietoihini suomen kielen äänne- ja muotorakenteen kehityksestä, joista lähteinä järjestelmää työstäessäni olen käyttänyt
alkeisteoksia Remes (2004); Setälä (1930); Karlsson (1982), toisaalta puhtaasti
kokeiluun ja kielitajuuni. Alkeisteoksista Setälän kielioppi on yksi selkeimmistä
ja systemaattisimmista näkemistäni esityksistä suomen sanojen muoto-opin järjestelmästä, vaikkakin paikoin vanhentunut, joten olen käyttänyt myös Remeksen
laatimaa luentomonistepakettia faktojen tarkistuksessa. Samat tiedot löytynevät
fennistiikan peruskirjallisuudesta muutoinkin. Karlssonin äänne- ja muoto-oppia
taas on käytetty Nykysuomen sanakirjan luokkien vähentämiseksi Suomen kielen
perussanakirja, joten olen siitä poiminut myös joitain kuvauksia7 .

Nykysuomen sanalistan luokittelu lienee motivoitu siten, että sanaluokkien ensimmäiset luokat 1 ja 52 ovat perusluokkia, joissa äännevaihtelu on vähäisintä, ja
uudet luokat on tuotu jokaiselle havaitulle uudelle äännevaihtelukokonaisuudelle
tai pääteallomorfijoukolle. Samalla perusteella itse tutkielman toteutettu morfologiakin on kappaleessa 3 kuvattu. Kaikkien taivutusluokkien erittely löytyy myös
kuvausten jäljestä taulukoista 1, 2 ja 3.

Nominit jakautuvat tavallaan kahteen pääluokkaan: ne, joiden perusmuoto on vokaalivartalo, ja joiden astevaihtelu on suora, muodostavat luokat 1—32, ja ne, joiden perusmuoto on konsonanttivartalo, ja joiden astevaihtelu on käänteinen, muodostavat luokat 33—49. Verbit jakautuvat samoin astevaihtelun suhteen suoraan
(luokat 52—65) ja käänteiseen (luokat 66—78). Verbeistä lisäksi huomionarvoista on, että sanakirjamuoto a-infinitiivissä on taivutusvartalon lisäksi A-, dA- tai
tA-tyyppinen a-infinitiivin tunnus.

\footnote{
7

Tiedon Karlssonin kirjan käytöstä sanakirjatyössä sain vasta laadittuani seuraavat kuvaukset
ja taulukot, joten sitä olen käyttänyt viitteenä jälkikäteisesti.
}


Ensimmäinen taivutusluokka (valo) on nominien yksinkertaisin luokka, johon
ei liity ollenkaan vartalon äännemuutoksia poislukien sanaluokkakoodiin erikseen
merkityt astevaihtelumuutokset, ja vaihtelevia allomorfeja ei ole vaan monikon
genetiivin, partitiivin ja illatiivin tunnuksissa on vain yksi kieliopillinen vaihtoehto, morfeemit jen (valojen), jA (valoja) ja ihin (valoihin) vastaavasti. Luokkaan
kuuluvien nominien vartalon äännerakennetta rajoittaa vain vartalovokaalina O tai
U, jotka ovat vaihteluttomia. Luokka on suurehko ja avoin, sillä siihen on merkitty muun muassa nut-partisiipin passiivin8 kiteytyneet muodot sekä stO-johdokset,
jotka molemmat ovat hyvin produktiivisia9 .

Luokat 2—4 ovat muutoin kun luokka 1, mutta ne merkitsevät erilaisia pääteallomorfivaihtoehtoja.

Luokan 2 (palvelu) päätteiden tunnusten allomorfeihin kuuluu luokan 1 morfeemien lisäksi monikon partitiivin tunnus itA (palveluita), sekä genetiivin tunnukset
iden (palveluiden) ja itten (palveluitten). Luokkaan kuuluvien sanojen vartalo on
poikkeuksetta vähintään kolmitavuinen, ja lisäksi O- tai U-loppuinen kuten luokassa 1.

Luokka 3 (valtio) on äännerakenteeltaan monitavuisten vokaaliyhtymäpäätteisten
sanojen luokka, joiden monikon genetiivin tunnuksen allomorfit ovat iden-ittentyyppiä (valtioiden ∼ valtioitten), mutta partitiivin vain itA-tyyppiä (valtioita).

Luokkaan 4 (laatikko) kuuluu useita monikon tunnusten allomorfeja, jotka liittyvät sekä heikkoon että vahvaan vokaalivartaloon siten, että muodot jen (laatikkojen) ja jA (laatikkoja) liittyvät vahvaan, iden (laatikoiden), itten (laatikoitten) ja ita (laatikoita) heikkoon ja
ihin kumpaankin vartaloon (laatikoihin ∼ laatikkoihin), kuten on morfofonotaktisesti tyypillistä. Tämä vaihtelu esiintyy vain äännerakenteeltaan kolmi- tai useampitavuisissa sanoissa joiden loppu on kkO, kuten tässä luokassa, tai kkA, kuten
luokassa 14.

Luokat 5 ja 6 (risti ja paperi) sisältävät vartalovokaalin i vaihtelusta e:ksi monikon tunnuksen i:n edeltä, eli niihin kuuluu i:hin loppuvia sanoja, joiden monikkomuodoissa monikon tunnuksen i:tä tai j:tä edeltää useimmissa muodoissa e.
Kuitenkin niin että monikon nominatiivilla tai akkusatiivilla on yksikkövartalon
mukainen i.
Luokalla 5 monikon genetiivi on ien-tyyppiä (ristien) ja partitiivi jA-tyyppiä (ristejä). Luokkaan 6 kuuluvat sanat saavat useampia monikon tunnusten allomorfeja
(papereiden : papereita).

Luokka 7 (ovi) sisältää myös vaihtelua i : e, tai tarkemmin i : e : ∅, tässä tapauksessa i reaalistuu vain yksikön nominatiivissa ja muissa yksikkömuodoissa
esiintyy e (ovena), joka teoriassa on vartalovokaali. Monikon tunnuksen i:n edeltä
(ovina) vartalovokaali katoaa.

Luokka 8 (nalle) on sama vaihteluton taivutusluokka kuin 1, sillä poikkeuksella että siihen kuuluu tyypillisesti vaihteleva vartalovokaali e, joka ei tässä vaihtele eikä aiheuta astevaihtelua. Luokkaan kuuluvat sanat ovat lähinnä verrattain
uusia, ja monet myös sitaattilainoja, joiden taivutus on muutoinkin epäsäännöllistä, esimerkiksi astevaihtelutonta pääsäännön vastaisesti. Nykysuomen sanalistan
mallitaivutustaulukko merkitsee tähän taivutusluokkaan myös harvinaiseksi monikon genetiivin in-tunnuksen (nallein), joka esiintynee pelkästään tyylivärisenä
vanhastavana muotona.
Numeraali kolme on merkitty taivutettavaksi sekä luokan 8 että luokan 9 kuvauksen mukaisesti riippuen siitä taivutetaanko yksikössä vai monikossa.

Luokat 9—15 ovat vartalovokaalin a vaihteluiden a : ∅ ja a : o : ∅ sekä pääteallomorfivarianttien eri kombinaatioita.
Luokassa 9 (kala) vartalovokaali A reaalistuu O:na monikon tunnuksen i:n (kaloissa) edellä. Monikon taivutustunnuksilla ei ole vaihtelevia allomorfeja.

Luokkaan 9 on sanalistassa merkitty myös sana aika (yhdyssanamuotoineen), joka ei siihen säännölisesti sovellu, vaan sisältää myös i : j -vaihtelun astevaihtelussa kadonneen koon takia vokaaliväliin joutuneen i:n asteittaisen muuttumisen
seurauksena. Lisäksi luokassa on numeraali sata, joka taipuu kuten odotettua.

Luokka 10 (koira) on vartalovokaalin vaihtelun A : ∅ luokka, jossa monikon tunnuksen edellä A:ta vastaa kato. Luokkaan 10 kuuluu myös sana poika, jossa esiintyy luokan muista sanoista poiketen yllämainittu i : j -vaihtelu. Lisäksi luokkaan
on merkitty numeraalit neljä ja nolla, jotka taipuvat luokan kuvaamalla tavalla,
mutta myös numeraalit seitsemän, kahdeksan ja yhdeksän, joiden taivuttaminen
luokan kuvaamalla tavalla olettaa, että sanan taivutusvartalona käytetään muotoa,
josta loppu-n on poistettu.

Luokkaan 11 (omena) sisältää vartalovokaalin vaihtelusarjan A : O : ∅ ja siihen
liittyy koko joukko monikon tunnusten allomorfeja. Monikon genetiivissä tunnukset ovat ien (omenien), iden (omenoiden) ja itten (omenoitten) sekä harvinaisina
jen (omenojen) ja in (omenain), joista ensi mainittu liittyy vokaalittomaan vartaloon ja jäljemmät O:lliseen, sekä harvakäyttöinen in yksikkövartaloon. Vastaavasti monikon partitiivilla vokaalittomaan vartaloon liittyy allomorfi iA ja O:lliseen
itA sekä harvemmin jA. Myös illatiivilla on tässä luokassa kaksi allomorfia, iin
vokaalittomaan ja ihin O:lliseen vartaloon.

Luokkaan 12 (kulkija) pätee vain A : O -vaihtelu ja ainoastaan monikon genetiivi
saa allomorfit iden (kulkijoiden) ja itten (kulkijoitten), sekä harvinaisena yksikkövartaloon in-muodon.

Luokka 13 (katiska) on kuten 12, mutta allomorfivalikoimaan lisätään monikko
genetiivin jen-tunnus (katiskojen) ja partitiivin jA-tunnus (katiskoja), kun 12:ssa
oli vain itA.

Luokka 14 (solakka) on kuten 13, mutta siihen liittyy sama monikon illatiivin asteen vapaa vaihtelu kuin luokkaan 4 (solakoihin ∼ solakkoihin). Äännerakenteltaan luokka sisältää vähintään kolmitavuisia sanoja, jotka päättyvät enimmikseen
kkA, mutta myös ttA ja ppA.

Luokka 15 (korkea) sisältää A-loppuisten vokaaliyhtymäloppuisten adjektiivien
a : ∅ -vaihtelun.

Luokkaan 16 (vanhempi) on kerätty adjektiivien komparatiivien kiteytymiä, joiden lopun mpi-aines taipuu normaalien adjektiivien komparatiivien tavoin mPA,
ja monikon allomorfit mp-vartalon perään tulevat partitiivin iA, illatiivin iin ja genetiivin ien. Myös komparatiiviperäiset pronominit, kuten kumpi ja kumpikin on
sijoitettu tähän luokkaan.

Luokat 17 ja 18 sisältävät pitkän vartalovokaalin vaihtelun lyhyeksi monikon tunnuksen iin edeltä. Lisäksi 17 (vapaa) ottaa yksikön illatiivin allomorfin seen (vapaaseen), joka liittyy vain pitkän vokaaliaineksen perään, ja 18 (maa) allomorfin
hVn (maahan).

Luokka 19 (suo) on yksitavuisten sanojen väljenevien diftongien suppea ja ilmeisesti suljettu luokka, johon kuuluu 6 sanaa. Luokkaan kuuluvat lyhyet UO- tai ieloppuiset sanat, joiden monikkomuodoissa diftongin ensikomponentti katoaa tunnuksen iin (soihin) edeltä. Myös nämä sanat ottavat yksikön illatiivin allomorfin
hVn.

Luokat 20 ja 21 (filee, rosé) sisältävät niinikään vartalon pitkän vokaaliaineksen
lyhenemät ja ovat samoja sillä erotuksella. Näistä 20 sisältää ensi sijassa mukautuslainoja, joista pitkä vokaaliaines käy ilmi, 21 sitaattilainoja, josta pitkä vokaaliaines ei käy ilmi. Allomorfeilta nämä eroavat sillä, että 20-luokkaan kuuluvat
myös illatiivit seen (fileeseen) ja siin (fileisiin) sekä monikon genetiivi iden (fileiden), mutta luokkaan 21 vain hVn (roséhen), ihin (roséihin) ja itten (roséitten).

Luokkaan 22 (parfait) on sijoitettu toteutuksen kannalta hankala joukko vierassanoja, joiden kirjoitusasu päättyy konsonanttiin ja ääntöasu vokaaliin, ja jotka taivutetaan puolilainausmerkin ja ääntöasun mukaisen päätteen kanssa (parfait’na :
parfait’ina). Sanojen käsittelyn hankaluus seuraa siitä, että kirjoitusasu ei sinänsä
suoraan paljasta ääntöasua, eikä Nykysuomen sanalistassa olevista tiedoista sitä selviä. Tässä luokassa taivutuksen allomorfit ovat kuten luokassa 21:kin, joten
yksikön illatiivin vokaalia (parfait’hen) ei voida tietää, eikä tietenkään myöskään
vokaalisointua.

Luokat 23—26 sisältävät konsonanttivartaloon liittyvät yksikön partitiivin ja monikon genetiivin allomorfit tA ja ten. Lisäksi luokassa 25 esiintyy nasaaliassimilaatiosta johtuva m : n(t) vaihtelu näiden muotojen kanssa. Vartalovokaalivaihtelun osalta kyse on luokan 7 i : e -vaihtelusta. Luokkien 24 ja 26 välinen ero on
ainoastaan niiden monikon genetiivin allomorfien keskinäinen järjestys.

Luokkaan 24 kuuluu myös sana meri ja luokkaan 26 sana veri, jotka taipuvat
muutoin odotuksenmukaisella tavalla, mutta niiden yksikön partitiivin tunnuksessa käytetään odotuksenvastaista sointuvokaalia (merta ja verta vastaavasti).

Luokat 27—31 sisältävät ti-si-muunnokseen liittyvät monimutkaiset vaihtelut
(ks. tarkemmin esim. Karlsson (1982)), joihin sisältyy myös vartalovokaalin i :
e -vaihtelu sekä konsonanttiklusterien yksinkertaistumat.

Luokan 27 (käsi) yksikön nominatiivin vartalo päättyy si, muiden yksikkömuotojen vartalo on astevaihtelullinen te : de (kätenä : kädessä), yksikön genetiivi
ttA (kättä) ja monikkomuotojen s, josta vokaali on kadonnut monikon tunnuksen
i:n edeltä kuten luokissa 5 ja 6. Luokassa 27 ovat myös viisi ja kuusi, jotka ovat
numeraaleja.

Luokassa 28 (kynsi) si-sarjaa edeltää l, r tai n, joka aiheuttaa astevaihdellun devartalotyypin assimiloitumisen le-, re- ja ne-tyyppisiksi (kynnessä) vastaavasti.

Luokat 29 ja 30 (lapsi ja veitsi) ovat muutoin kuin luokat 27 ja 28, mutta psija tsi-päätteet muuttuvat konsonanttivartaloisessa yksikön partitiivissa ja monikon
genetiivissä stA- (lasta, veistä) ja sten-muotoihin (lasten, veisten), vastaavasti, historiallisen konsonanttiklusterin yksinkertaistuman seurauksena (Remes, 2004).

Luokka 31 (kaksi) sisältää äännevaihteluina ti-si-muunnoksen lisäksi k : h -vaihtelun (kahden), joka on seurausta vanhasta ks : ht -muutoksesta (vrt. lahti < vanh.
laksi, « ruots. lax) (Remes, 2004). Muutoin luokka on kuin 27. Luokka sisältää
vain sanat yksi ja kaksi, joka kattaa sen äännerakennekuvauksen sinänsä.

Luokka 32 (sisar) sisältää r-, l- ja n-loppuiset konsonanttivartalot, jotka vaihtelevat e-loppuisen vokaalivartalon (eli re, le ja ne vastaavasti) kanssa yksikkömuodoissa. Taivutustunnusten allomorfeiksi kuuluvat konsonanttivartalojen yksikön partitiivi tA (sisarta) sekä monikon genetiivit ten (sisarten) ja ien (sisarien).
Myös numeraali kymmenen on sijoitettu tähän luokkaan, kuitenkin siten, että oletettu taivutusvartalo olisi kymmen.

Luokka 33 (kytkin) sisältää n-loppuiset konsonanttivartalot, jotka vaihtelevat vokaalivartalossa m:n kanssa (kytkimen). Taivutustunnusten allomorfit ovat kuten
luokassa 32.

Luokka 34 (onneton) koostuu tOn-loppuisista adjektiivin abessiivin karitiivijohdosten kiteytyneistä muodoista, jotka ovat taas sanakirjan kannalta juurestaan niin
etäisiä että ne on merkitty omiksi muodoikseen. Muutoin taivutus kuuluu luokan 35 kanssa (lämmin) n : mA -vaihteluun. Luokat 34 ja 35 eroavat allomorfeiltaan
vain monikon genetiivin harvakäyttöisten varianttien osalta siten, että 34 voi ottaa
päätteen ten (onnetonten) ja 35 in (lämpimäin).
Luokka 36 (sisin) sisältää kiteytyneet adjektiivien superlatiivimuodot, ja taipuu
samoin kuin adjektiivin superlatiivi normaalisti n : mPA-vaihtelullisena. Myös
luokka 37 (vasen) taipuu samoin ja käytännössä samoilla allomorfeilla, eroksi
jää kantasana, ja teoriassa harvinaisen yksikön partitiivin allomorfin esiintyminen, sillä sanakirjan luokiutuksen mukaan superlatiivin kiteytymiin ei saa liittää
pääteallomorfia (mp)AA (? sisimpää, vasempaa).
Luokka 38 (nainen) sisältää nen-loppuiset, alkujaan johtimelliset sanat, joiden
taivutus perustuu johtimettomaan muotoon, eli vartalona on se, jonka vartalopääte se toimii kuten luokassa 29—31. Merkillepantavaa tässä luokassa on, että nominatiivin sijasta yhdyssanan määriteosan muotona käytetään nominatiivitapauksissa aina erillistä s-loppuista yhdyssanamuotoa (nais-). Muissa luokissa erillinen
yhdyssanamuoto on harvinainen.

Luokat 39—42 sisältävät s-loppuisten äännevaihtelut. Luokan 39 (vastaus) s
vaihtelee vokaalivartaloon kse-muodossa, jonka e käyttäytyy kuten vartalovokaali e luokassa 5. Lisäksi konsonanttivartalot saavat yksikön partitiivin allomorfinsa
tA ja monikon genetiivin ten. Luokka 40 (kalleus) käyttäytyy monikon osalta kuin
39, mutta yksikön vokaalivartalo kuten te : de luokassa 27.

Luokka 41 (vieras) sisältää vaihtelun s : V, missä s siis esiintyy yksikön vokaalivartalossa edeltävän vokaalin pidentymänä (vieraana). Lisäksi monikkomuodon
allomorfit ovat vokaalivartalopohjaisesti genetiivin iden-itten ja partitiivin itA sekä illatiivin isiin. Luokka 42 (mies) sisältää poikkeuksellisen s : he vaihtelun,
muutoin vaihtelu on kuten luokissa 32 ja 33.

Luokat 43—44 ovat t-loppuisten sanojen vaihteluluokat, näistä 43 (ohut) vaihtelee t : (∅)e, mutta muutoin kuten luokat 32, 33 ja 42. Monikon allomorfeina vain
vokaalivartaloiset iden, itten, itA ja isiin. Luokka 44 (kevät) vaihtelee vokaalivartalossa vokaalin pitenemän kanssa kuten luokka 41:kin, ja näiden välisenä erona on vain loppukonsonantti.

Luokka 45 (kahdeksas) sisältää järjestysluvut, joiden taivutus noudattaa epäsäännöllistä s : nte -vaihtelua, ja jonka t monikkomuodoissa vaihtelee s:n kanssa, koska
joutuu i:n edelle vartalovokaalin e:n kadon myötä.

Luokka 46 (tuhat) sisältää lukusanan tuhat, jonka taivutus on noudattaa epäsäännöllistä t : nte -vaihtelua, joka luokan 45 sisältää ti : si -vaihtelun monikossa.

Luokka 47 (kuollut) koostuu nut-partisiippien aktiivien muodoista, jotka ovat tarpeeksi kiteytyneitä kuuluakseen sanakirjaan. Vartalon vaihtelut koostuvat siis ut :
eet : e -muodoista, joita edeltää nut-partisiipin tunnuksen assimiloituva n.

Luokka 48 (hame) sisältää e-loppuiset sanat, joihin on kielihistoriallisesti kuulunut loppukonsonantti, ja jotka taipuvat kuin niissä olisi loppukonsonantti paikalla. Myös sanan ääntöasussa on loppukonsonanttia merkitsemässä vielä jäännöslopukepiirre. Yksikön vokaalivartalossa jäännöslopuke vaihtelee pitkän vokaalin
kanssa kuten luokissa 44 ja 41. Sanan yksikön partitiivin allomorfi on ttA kuin
konsonanttivartalossa, mutta monikkoallomorfit ovat vokaalivartaloiset iden, itten, itA, isiin ja ihin. Luokkaan kuuluu myös uusia sanoja, jotka on muodostettu
ex -johtimella, vaikka niihin ei muutoin jäännöslopukepiirrettä kielihistorian kautta tulisikaan.

Luokkaan 49 (askel ∼ askele) kuuluvat sanat, joilla on kaksi erillistä taivutusvartaloa, toinen luokkaan 48 kuuluvan kaltainen le-, re- tai ne-loppuinen, toinen
luokkaan 32 kuuluvan kaltainen l-, r tai n-loppuinen. Nykysuomen sanalistassa
sanoista on yleensä molemmat versiot, mutta mahdollinen astevaihtelu on kuvattu
vain konsonanttivartaloisille nominatiiveille (esim. manner’49−J ∼ mantere49 ).

Yhdysnominien luokat 50 ja 51 sisältävät kaikki yhdyssanat, joita ei tämän tutkielman aiheena käsitellä tarkemmin, mutta käytännössä on niin, että luokan 50
sanoja taivutetaan kuten yhdyssanan jälkiosaa sen omassa luokassaan taivutettaisiin, ja luokan 51 sanoja taivutetaan kuten sen molempia osia omissa luokissaan. Käytännössä näitä luokkia ei suoraan itse yhdyssanojen yhteydessä sanalistasta yleensä löydy, eikä yhdyssanarajan jakopaikkaa —, joka on kieliteknologiselle järjestelmälle osin ongelmallista. Varsinaisesti sanalistassa luokan 50 sanoja,
joiden osat sanalistasta löytyvät itsenäisinä tietueina ei ole luokiteltu mihinkään
luokkaan, ja ne luokan 50 yhdyssanat, joiden perusosa puuttuu sanalistasta itsenäisenä tietueena, on pyritty sijoittamaan siihen luokkaan, jonka mukaan varsinainen taivutus on. Luokkaan 51 kuuluu perinteisesti adjektiivialkuisia yhdyssanoja,
ja adjektiivialkuisista sanoista kaikki onkin luokiteltu joko luokkaan 50 tai 51,
poikkeuksena nominialkuisista. (Eronen, 1994, sanakirjan osalta)

Luokka 52 (sanoa) on verbien yksinkertaisin ja vaihteluttomin luokka vastaavasti
kuin luokka 1 on nomineiden kohdalla. Se sisältää vartalovokaalit O ja U, jotka
siis verbien tapauksessa sanakirjamuodossa esiintyvät ennen a-infinitiivin tunnusta, tässä A:ta.

Luokat 53—55 sisältävät vartalon A:n kadon menneen ajan tunnuksen i:n edeltä,
ja kuvaavat siitä syntyvän ti-sarjan ti-si-äännemuunnoksen eri vaiheita.

Luokassa 53 (muistaa) muunnosta ei tule, vaan menneen ajan muodossa on aina
t (muisti). Luokka 54 (huutaa) sisältää myös pakollisen muunnoksen menneen
ajan muodossa (huusi). Luokassa 55 (soutaa) ti- ja si-muodot ovat vapaassa vaihtelussa (souti ∼ sousi).

Luokka 56 (kaivaa) sisältää vartalovokaalivaihtelun A : O menneen ajan muodoissa (kaivoi). Luokassa 57 (saartaa) luokan 56 A : O -vaihtelullinen sekä luokan 54 ti : si -vaihtelullinen A : ∅ -muoto ovat vapaassa vaihtelussa menneen ajan
muodoissa (saarsi ∼ saartoi).

Luokat 58—60 ovat e : ∅ -vaihtelun sisältäviä luokkia, näistä luokka 58 (laskea) on muutoin vaihteluton. Luokka 59 (tuntea) sisältää ti : si -vaihtelun menneen ajan muodossa (tunsi) — sekä t:n heikon asteen assimilaation n:ään (tunnen). Luokka 60 (lähteä) on myös kuin 58, mutta sisältää harvinaisena ks : ht
-vaihtelun menneen ajan muodoissa (lähti ∼ läksi) kuten nominiluokassa 31.

Luokka 61 (sallia) sisältää vartalovokaalin i kadon menneen ajan (sallin) sekä
konditionaalin (sallisin) tunnuksen i:n edeltä.

Luokissa 62—65 on pitkä vokaalivartalo, jonka jäljessä esiintyy a-infinitiivin tunnus dA. Vartalon pitkä vokaaliaines esiintyy lyhyenä menneen ajan ja konditionaalin tunnuksen i:n edellä.

Luokat 62 (voida) ja 63 (saada) eroavat nähtävästi vain sillä, että vartalon vokaaliaines on i-loppuinen diftongi ja pitkä vokaali vastaavasti. Luokka 64 (juoda) sisältää vartalovokaalinaan väljenevän diftongin, joka käyttäytyy kuten nominiluokassa 19, diftongin ensikomponentti katoaa taivutustunnuksen i:n vaikutuksesta.

Luokka 65 (käydä) sisältää poikkeuksellisen y:n vaihtelun v:ksi menneen ajan
(kävin) tai konditionaalin (kävisin) tunnuksen i:n edeltä.

Luokka 66 (rohkaista) sisältää s-konsonanttivartalot ja luokka 67 (tulla) l- ja
r-konsonanttivartalot. Konsonanttivartaloihin liittyy potentiaali ja nut-partisiippi
tunnuksen n assimiloituneena. Luokkaan 67 on pantu myös olla-verbi, johon sisältyy sekä poikkeavaa taivutusta kolmansien persoonien muodoissa että suppletiivinen paradigma potentiaalimuodoissa.

Luokkaan 68 (tupakoi ∼ tupakoitsee) kuuluvat luokan 62 sanat, joilla on harvinaisena vaihtoehtoisena taivutusvartalona tyypin 69 vartalo.

Luokka 69 ja 70 (valita, juosta) sisältävät e : ∅ -vaihtelun sekä kompleksisemman
konsonanttiklusterin yksinkertaistuman siten, luokkaan 69 kuuluu tse : ts -vartalot ja 70 kse : st : s -vartalot.

Luokkaan 71 (nähdä) kuuluu ke : h-vaihtelu.

Luokka 72 (vanheta) sisältää vartalot, joissa ne-osa on ilmeisesti perua kokonaan
toisesta paradigmasta, eli vartalon vaihtelu on ∅ : ne.

Luokka 73—75 sisältävät tavallisen vokaalivartalon ja lisäksi menneen ajan
tunnusta edeltää vaihtelun johdosta s. Luokat eroavat toisistaan siten, että avartaloinen 73 (salata) ottaa menneen ajan tai konditionaalin tunnuksen vokaalivartalon lyhyen vokaalin jälkeen (salasin). E-, O- tai U-vartaloinen 74 (katketa)
säilyttää pitkän vokaaliaineksen ennen konditionaalin tunnusta (katkeaisin), mutta konditionaalissa tapauksittain myös luokan 73 mukainen kato on mahdollinen
(katkeisin). I-vartaloinen 75 (selvitä) taipuu kuten 74, mutta luokan 73 mukaista
konditionaalia ei esiinny, ja se menisikin yhteen konditionaalimuodon kanssa.

Luokka 76 (taitaa) on 58, mutta lisäksi se sallii n-konsonanttivartaloiset muodot
potentiaalissa (tainnee) ja nut-partisiipissa (tainnut).

Luokkaan 77 (väräjää) on koottu AjAA-tyypin verbit, joista esiintyy käytössä
vain yksikön kolmannen indikatiiveja (väräjää : väräji) ja konditionaaleja (väräjäisi).

Luokassa 78 (kaikaa) on verbit, joista esiintyy vain yksikön kolmannen indikatiivin menemättömän ajan muoto (kaikaa) ja konditionaali (kaikaisi).

Luokkiin 99 ja 101 on pantu adverbejä, partikkeleja ja pronominejä, joita en tässä
tutkielmassa käsittele.

Astevaihteluluokkia aineistossa on 13, jotka kuvaavat astevaihtelun alaisena olevan äänneparin, mutteivät astevaihtelun suuntaa, joka käy ilmi taivutusluokasta
sinänsä. Astevaihtelut on numeroiden sijasta annettu kirjaimin A—M, ja ne muodostavat kombinaatioita numeroluokkien kanssa. Luokat on järjestetty siten, että kirjaimet A—C kuvaavat kvantitatiivisen astevaihtelun eri klusiileille, ja loput
D—M kuvaavat kvalitatiivisen astevaihtelun eri toteutumavariantteja.

Astevaihteluluokista lienee huomionarvoista, että uudehkolla vierasperäisten konsonanttien g, b ja d kvantitatiivisella astevaihtelulla (dubata : dubbaan ja digata :
diggaan) ei ole omaa astevaihteluluokkaa, eikä sitä ole Nykysuomen sanalistaan
merkitty ja painetussa sanakirjassa se löytyy vain sana-artikkelin proosasta (Eronen, 1994). Myös sanassa auer esiintyvä t : ∅ -astevaihtelu puuttuu luokituksesta.
Astevaihtelun kannalta on merkittävää, että astevaihtelun toteutuminen sanakirjamuodossa riippuu siitä, onko astevaihtelun alainen tavu sanakirjamuodossa lyhyt avotavu vai ei. Perinteisesti kieliopissa sanakirjamuodoltaan avotavuisia eli
vahva-asteisia muotoja on kutsuttu astevaihtelun pääsäännön suoran astevaihtelun
alaisiksi, ja niitä, joissa on umpitavu ja heikko aste sanakirjamuodossa käänteisen
astevaihtelun mukaisiksi (Hakulinen et al., 2004). Nomineilla suoraan astevaihteluun kuuluvat yksinkertaisesti taivutusluokat 1—31, jotka ovat vokaaliloppuisia
nomineja. Luokat 32—47 ovat konsonanttiloppuisia käänteisen astevaihtelun sisältäviä luokkia. Verbien kohdalla suoraan astevaihteluun kuuluvat 52—65, käänteiseen 66—75. Kun astevaihtelun alainen äänne selviää sanaluokkanumeron ja
astevaihtelukirjaimen avulla, sen paikantaminen selviää lähes aina triviaalisti sanakirjamuodon kirjoitusasusta poimimalla viimeinen eli oikeanpuoleisin kyseeseen tuleva kirjain. Poikkeuksia luovat muutamat verbiluokat, joissa a-infinitiivin
tunnuksen t pitää sivuuttaa ensin sekä käänteisessä astevaihtelussa luokan D selvittäminen, kun siinä sanakirjamuodossa k:ta vastaa kato.

Taivutusluokituksen jäsentimessä käyttämistäni piirteistä olen laatinut kuvaukset
taulukkoihin 1 nomineille ja 2 verbeille, sekä astevaihtelun osalta yhteenvetona
taulukkoon 3. Samat tiedot löytyvät teoriassa Nykysuomen sanalistan (Kotimaisten Kielten Tutkimuskeskus, 2006) mallitaivutustaulukoista hieman eri muodossa.
Nominitaulukon äännevaihtelusarakkeessa kuvataan vartalossa esiintyvät äännemuutokset poislukien astevaihtelusta johtuvat —, jotka kuitenkin saattavat toisinaan olla luokalle ominaisia. Sarakkeessa äännerakenne on esitetty yleistäen se
äännerakenne, jota kaikki luokan sanat (yksikön nominatiivissa) noudattavat, sanan lopusta lukien, eli äännerakennekuvaus ”u” tarkoittaisi vartalovokaalia u ja
”psi” vain psi-loppuisia vartaloita. Monissa luokissa on muutamia sanoja, jotka poikkeavat odotuksenmukaisesta muodosta, esimerkiksi osittain mukautuneita
vierassanoja. Tällaiset tapaukset on usein merkitty sulkeilla. Esimerkiksi luokkaan 3 näyttävät kuuluvan ensi sijassa io- ja iö-loppuiset omaperäiset sanat, mutta
myös muut vokaaliyhtymään päättyvät uudet vierasperäiset sanat (esim. zombie3 ,
sikäli kun se lausutaan suunnilleen [0 tsom.bi.e]).

Huomattavaa on myös, että jotkin pronominit ja numeraalit on luokiteltu mukaan
luokkiin siten, etteivät ne sinänsä täsmää kuvaukseeni, vaikka etymologisesti sopivatkin, esimerkiksi sanan kumpi kieliopillistuneet kliitilliset muodot, kuten sanaksi merkitty kumpikaan, ovat luokassa 16 muiden komparatiivien kanssa, vaikka ne
taipuvat kliittinsä edestä (kumpikaan : kummatkaan : kummillekaan).

Taivutusmuotosarakkeissa yks. ptv., yks. ill., mon. ptv., mon. gen. ja mon. ill. on
kuvattu monta eri allomorfia sisältävien taivutuspäätteiden vapaassa vaihtelussa
olevat vaihtoehdot kuten ne on lähdeaineistossa annettu. Tästä lienee huomattava,
että aineiston mukaan annetulla järjestykselläkin on väliä, ja tosiaan esimerkiksi
luokkien 24 ja 26 ainoa morfologinen ero on kahden taivutustunnuksen järjestyksen ero.
Allomorfien ja vartalotyyppien suhdetta en ole kuvannut, sen yksityiskohdat löytyvät vain toteutuksesta tarkemmin. Vartalotyyppeihin viittaavat nimet nominatiivivartalo, heikko vokaalivartalo, vahva vokaalivartalo, konsonanttivartalo ja monikkovartalo ovat kuvauksissa kuten perinteisissä koulukieliopeissa, eli yksikön
nominatiivi sellaisenaan, yksikön essiivin vokaalivartalo, monikon nominatiivin
vokaalivartalo, yksikön partitiivin konsonanttivartalo ja i-tunnuksisten monikkomuotojen i:tä edeltävä vokaalivartalo vastaavasti.

Taulukon arvaukset perustuvat morfologian kehityksessä käytettyihin arvioihin,
joiden lähteenä ovat olleet aiemmin mainitut kieliopit (Remes, 2004; Setälä,
1930), Nykysuomen ja Perussanakirjan kuvaukset (Eronen, 1994, 1997), aiemmin toteutetut kieliteknologiset morfologiajärjestelmät (Koskenniemi, 1983) sekä
Nykysuomen sanalistan silmämääräinen ja koneellinen tarkastelu. Eräänä työkaluna vartalon äännerakennevaihteluita arvaillessani olen käyttänyt työkalua, joka
hakee vapain GNU GPL -sovelluksin luettelosta tietyn luokan sanoja, tietyn vartaloisia sanoja, sekä tietyn luokan sanoja, jotka eivät ole kirjoitusasultaan tiettyä
hakulauseketta vastaavia. Työkalua käytetään hakemalla kaikki luokan sanat listasta, arvaamalla äännerakenne, ja pyytämällä kaikkia arvatusta äännerakenteesta
poikkeavia luokan sanoja, ja jos tuloksena on tyhjä joukko on ainakin selvää että hakulauseke kattaa koko listassa olevien luokan sanojen äännerakenteen. Tästä
seuraa, että monet tutkielmassa käytetyistä arvioista sanojen luokkien äännerakenteesta ovat vailla varsinaista lingvististä perustetta, eli silmämääräisiä. Osa arvauksista on toki myös pyritty liittämään em. kielioppien alkeiskuvauksiin kielen
äännekehityksestä, kuten on aiemmissa luokkakuvauksissa huomattavissa, mutta
systemaattisesti tätä ei ole tehty.

\begin{table}
        \caption{Nominiluokkien äänne- ja muotorakenne sekä
        muut erottavat piirteet}
    \begin{tabular}{|c||l|l|l|lll|l|l|}
        \hline
        \bf Piirre→ &
        \bf Vartalo- & Äänne- & Yks. & Yks. & Mon. & Mon. & Mon. \\
        \bf ↓Luokka &
        \bf vaihtelu & rakenne & ptv. & ill. & ptv.& gen. & ill. \\
        \hline
%        \multicol{8}{|c|}{Vokaalivartaloiset, suoralla astevaihtelulla}
        \hline
    \end{tabular}
    \bf TODO:
1
2
3
4
5
6

—
O tai U
A
Vn
jA
—
O tai U
A
Vn
jA,
≥ 3 tavua
itA
—
iO (oe, yo tA
Vn
itA
ie, eo, . . . )
—
kkO
A
Vn
jA,
≥ 3 tavua
itA
†
i :e:∅
i
A
Vn
jA
i† : e : ∅
i
A
Vn
jA,
(jatkuu seuraavalla sivulla)

Mon.
gen.

Mon.
ill.

jen
jen,
iden, itten
iden,
itten
jen,
iden, itten
ien
ien,

ihin
ihin
ihin
ihin,
ihin‡
ihin
ihin

2

AINEISTOT

21

Piirre→
↓Luokka

Vartalovaihtelu

Äännerakenne

7
8
9
10
11

i:e:∅
e
A:O
A:∅
A:O:∅

i
e
A
A (An)
A,
≥ 3 tavua

12

A:O

13

A:O

A,
≥ 3 tavua
A

14

A:O

15
16
17
18
19
20
21
22
23
24
25
26
27

Yks. Yks. Mon.
ptv. ill.
ptv.
itA
A
Vn
iA
A
Vn
jA
A
Vn
jA
A
Vn
iA
A
Vn
iA,
itA
(jA)
A
Vn
itA
A

Vn

itA,
jA

Mon.
gen.
iden, itten
ien
jen (in)
jen (in)
ien (in)
ien, iden,
itten
(jen, in)
iden,
itten (in)
jen,
itten,
iden (in)
iden,
itten,
jen
iden,
itten (in)
ien (in)
iden,
itten
iden,
itten
iden,
itten
iden,
itten

Mon.
ill.
iin
ihin
ihin
iin
iin,
ihin
ihin
ihin

kkA, (ttA, A
Vn
itA,
ihin,
ppa)
jA
ihin‡
≥ 3 tavua
A:∅
eA
A,
Vn
itA
isiin
(UA, OA) tA
ihin
mpi : mPA
mpi
A
Vn
iA
iin
V1 V1 : V1
pitkä
tA
seen itA
isiin
vokaali
(ihin)
VV : V
pitkä votA
hVn itA
ihin
kaaliaines
V1 V2 : V2
laveneva
tA
hVn itA
ihin
diftongi
V1 V1 : V1
pitkä
tA
hen, itA
ihin,
vokaali
seen
isiin
vierasperäiset pitkään vokaaliainekseen loppuvat
vierasperäiset kirjoituksessa konsonanttiin, ääntäessä vokaaliin loppuvat
i:e
i
tA
Vn
iA
ien
iin
i:e
i
tA
Vn
iA
ien,
iin
1
ten
mi : me : n
mi
tA,
Vn
iA
ien,
iin
A
ten
i:e
i
tA
Vn
iA
ten,
iin
ien1
si : te : t
si
tA
Vn
iA
ien
iin
(jatkuu seuraavalla sivulla)

2

AINEISTOT

22

Piirre→
↓Luokka

Vartalovaihtelu

Äännerakenne

28

si : te : t

29
30

psi : pse : s
(ksi : kse : s)
tsi : tse : s

nsi,
Lsi
psi
(ksi)
tsi

31

ksi : hte : h

Yks. Yks. Mon. Mon.
ptv. ill.
ptv.
gen.
(ten)
tA
Vn
iA
ien
(ten)
tA
Vn
iA
ten,
ien
tA
Vn
iA
ien
(ten)
tA
Vn
iA
ien

34

yksi,
kaksi
konsonanttivartalot, käänteisastevaihtelulla
L : Le
L
tA
Vn
iA
n : ne
n
n : me
in
tA
Vn
iA
(n)
n : mA
tOn
tA
Vn
iA

35

mmin : mpimä lämmin

tA

Vn

iA

36

n : mpA

in,

tA

Vn

iA

37

n : mpA

vasen

Vn

iA

38

nen : se

nen

tA
(A)
tA

Vn

iA

39

s : kse

s

tA

Vn

iA

40

s

tA

Vn

iA

41

s : te :
*t : kse
s:V

s

tA

seen

itA

42
43

s : he
t:e

mies
Ut

tA
tA

Vn
Vn

iA
itA

44

t:V

ät

tA

seen

itA

45

s : nte :
t : ns

s

tA

Vn

iA

32
33

(jatkuu seuraavalla sivulla)

Mon.
ill.
iin
iin
iin
iin

ien,
ten
ien,
ten
ien,
(ten)
ien
(in)
ien,
ten (in)
ien,
ten (in)
ten,
ien
ten,
ien
ien

iin

iden,
itten
ien
iden,
itten
iden
itten
ien

isiin
(ihin)
iin
isiin,
ihin
isiin
(ihin)
iin

iin
iin
iin
iin
iin
iin
iin
iin

2

AINEISTOT

Piirre→
↓Luokka
46
47

Vartalovaihtelu
t : nte :
ns
Nut : Nee

48

e : ee

49

l ∼ le,
r ∼ re,
n ∼ ne

23
Äännerakenne
tuhat
nUt, LUt,
sUt
ex
(ori,
kiiru)
l ∼ le
r ∼ re
n ∼ ne

Yks. Yks. Mon. Mon.
ptv. ill.
ptv.
gen.
tA
Vn
iA
ien
(ten)
tA
seen itA
iden,
itten
ttA
seen itA
iden,
itten

Mon.
ill.
iin

tA

iin,
ihin,
isiin

(Vn)
seen

iA,
itA

ien, ten
iden,
itten

isiin,
ihin
isiin,
ihin

† myös konsonanttiloppuiset vierassanat, joissa sidevokaali i, mm. kaikki
konsonanttiloppuiset sitaattilainat
‡ ihin käy sekä vahvaan että heikkoon vartaloon

1

luokat 24 ja 26 eroavat toisistaan vain näiden allomorfien järjestyksen perusteella
\end{table}

Taulukko verbien vaihtelulle on hieman yksinkertaisemman näköinen, sillä verbeissä ei esiinny taivutuspäätteiden allomorfien vapaata vaihtelua siten kuin nomineissa. Sen sijaan verbeillä esiintyy vain vartalon vaihtelua, joka liittyy joko
vain menneen ajan muodon i-tunnukseen tai sekä siihen että konditionaaliin, riippuen kielihistoriallisista seikoista, joihin ei tässä tarkemmin puututa. Lisäksi merkillepantavaa on sanakirjamuodon suhde varsinaiseen verbivartaloon, sillä sanakirjamuodon eli a-infinitiivin tunnuksena esiintyy variantit A, tA, dA ja (tA)A, jotka eivät kuulu varsinaiseen taivutusvartaloon. A-infinitiivin allomorfi selviää kuitenkin sarakkeesta äännerakenne, sillä se kuvaa nimenomaan sanakirjamuodon
äännerakennetta. Sarakkeessa pret. eli preteriti on menneen ajan muodosta kuvattu sen vaikutus vartaloon, eli esim. mahdollinen ti>si-muunnos t-loppuisissa
vartaloissa. A-infinitiivin allomorfin mukaan määräytyvät infinitiivien tunnuksessa mahdollisesti esiintyvä d, joka on kuvattu sarakkeessa inf. Kyseessä on sinänsä
vain tavan astevaihtelukuvio, sillä poikkeuksella, että heikon asteen tunnuksesta
ð on tietyissä konteksteissa jäänyt jäljelle nykysuomessa kato (ks. tarkemmin Hakulinen (1979)). Vastaavasti s:n jälkeisessä asemassa aste on vahva. Sarakkeessa
pass. on kuvattu se vaihtelu, joka määrittää onko t-alkuisen passiivin alussa yksi
t (nähty, syötiin) vai kaksi t:tä (tapettu, haudattiin). Nämä vaihtelut ovat käytännössä samat kuin väitöskirjassa Koskenniemi (1983) kuvatun järjestelmän D- ja
Z-morfofoneemit.

\begin{table}
    \label{Verbiluokkien äänne- ja muotorakenne sekä
    muut erottavat piirteet}
    \begin{tabular}{|c||l|l|l|l|l|}
        \hline
        \bf Piirre→ &
        \bf Vartalo- & Äänne- & Pret. & Inf. & Pass. \\
        \bf ↓Luokka &
        vaihtelu & rakenne & & & \\
        \hline
    \end{tabular}
    \bf TODO:
52
53
54
55
56
57
58
59
60
61
62
63
64
65
66
67
68
69
70
71
72
73
74
75
76
77
78

Vartalovaihtelu
—
A:∅
tA : s∅(i)
tA : t∅(i) ∼ si
a:o
taa : s∅(i) ∼ : o
e:∅
nte : ns∅(i)
hte : ks∅(i)
i:∅
i:∅
V1 V1 : V1
V1 V2 : V2 i
y:v
s : se
L : Le
n : ne
i : ∅ ∼ ts
t : tse
s : kse
h : ke
t : ne
t:s
t:s
t:s
t:s
a:∅
—

Äännerakenne
OA, UA (iA)
tAA (AA)
tAA (sAA)
tAA
aa
Caa(r)taa
eA
tuntea
lähteä
iA, yä
idA
pitkä vokaali+dA, 1 tavu
UOdA, iedA, 1 tavu
käydä
stA
LLA
nnA
OidA ∼ OitseA
itA
UOstA, iestA
hdA
tA
AtA
OtA, UtA, etA
itA
taa
AjAA, ≥ 3 tavua
AA

Pret. Inf. Pass.
i
i
si
si
i
si
i
si
si
i
i
i
i
i
i
i

—
—
—
—
—
—
—
—
—
—
d
d
d
d
t
l

t
t
t
t
t
t
t
t
t
t
t
t
t
t
t
t

i
si
si
i
i
si
si
si
si
i
—

d
t
t
d
t
t
t
t
—
—
—

t
tt
t
t
tt
tt
tt
tt
tt
—
—
\end{table}

Astevaihtelun osalta taulukko on yksinkertainen, ja löytyy liki sellaisenaankin
aineistosta Kotimaisten Kielten Tutkimuskeskus (2006). Astevaihtelukirjain
sinänsä on morfofonologian kannalta redundanttia, sillä kun tiedetään vaihtelu
ja taivutusluokasta vaihtelun suunta, kyseeseen tuleva äänne selviää lopputavun
alusta ja sen toisen asteen vastine kontekstista säännönmukaisesti.

\begin{table}
    \label{Astevaihteluluokkien vaihteluparit}
    \begin{tabular}{|c||l|l|}
        \hline
        \bf Aste→ & \bf Vahva & \bf Heikko \\
        \bf v Luokka & & \\
        \hline
    \end{tabular}
    \bf TODO:
A
B
C
D
E
F
G
H
I
J
K
L
M

kk
pp
tt
k
p
t
k
p
t
t
t
k
k

k
p
t
∅
v
d
g
m
l
n
r
j
v
\end{table}

\subsection{Korpukset}

Lopullisen järjestelmän vertailutesteissä käytin aineistona korpusmateriaalia, joka
on saatavilla Tieteellinen laskenta CSC Oy:n ylläpitämällä palvelimella. Aineisto koostuu suomen kielen tekstipankin ei-vapaasti lisensoiduista B-korpuksista,
joita sinänsä on lupa käyttää kaupallisen tai open source -sovelluksen kehittämiseen, mutta esimerkiksi aineiston siirtäminen pois CSC:n omalta palvelimelta
ei tule kyseeseen, joten korpus-aineistoa ei voi esimerkiksi liittää valmiin kokonaissovelluksen osaksi esimerkiksi regressiotestausta varten. Regressiotestauksella tarkoitetaan tässä sellaista kieliteknologisen morfologiani kehitykseen liittyvää
toimivuustestausta, joka kertoo toimiiko uusi versio samoin kuin aiemmat.
Sisällöllisesti näissä korpuksissa teksti on sanomalehdistä otettua, joten sen voi
olettaa olevan suurelta osin laadullisesti ehjää yleiskieltä. Lisäksi korpukset on
valmiiksi jäsennetty käyttäen toista, kaupallista jäsennintä, joten tulokset antavat
myös karkeaa viitettä, mihin kehittämäni järjestelmä laadullisesti sijoittuu.

\section{Menetelmät}

Äärellistilainen morfologinen jäsennin on sellainen äärellistilainen transduktori,
joka sisältää kuvaukset sanojen taivutusmuotojen ja niiden sanakirjamuotojen sekä taivutusmuotojen tunnisteiden välillä (esim. kissasta ↔ kissa yksikön elatiivi) tai päinvastoin. Kuvauksen toteuttaminen lähtee siitä, että muunnetaan XMLmuotoinen sanalista sellaiseksi, jonka SFST osaa lukea transduktoriksi, ja tätä
transduktoria säännöin ja äärellistilaisin operaatioin käsittelemällä saadaan aikaiseksi sellainen äärellistilainen transduktori, joka kuvaa sanakirjamuodot taivutusmuodoiksi.

Kappaleessa 2 hahmotellun perusteella äärellistilainen järjestelmä, joka kuvaa
sanojen vartaloissa esiintyvät äännemuutokset, liittää sopivat taivutuspäätteet
sanojen sopiviin vartaloihin ja niin edelleen on perustettavissa pelkästään sanakirjamuotoihin, taivutusluokkanumeroihin ja astevaihtelukirjaimiin, joten
XML-sanalista pitää muuntaa äärellistilaisen järjestelmän sanalistaksi, jossa
jokainen sana muodostuu kolmikosta sanakirjamuoto, taivutusnumero ja astevaihtelukirjain. Tämä muunnos on tehty XSLT 2.010 -kielellä ja toteutusta on kuvattu kappaleessa 3.2.1. Tässä vaiheessa lähdetään siis liikkeelle tietueista tyyppiä
<st><s>pata</s><hn>1</hn><t><tn>9</tn><av>F</av></t></st>
ja päädytään tyyppiin pata<9><f>.

Itse äärellistilainen kuvaus, joka perustuu tuotettuun sanalistaan, on joukko
äärellistilaisten transduktorien operaatioita ja sääntöjä, joilla järjestelmällisesti tuotetaan sanan sanakirjamuodosta (tyyppiä pata<9><f>) taivutusvartalot
(tyyppiä pado<monikko>) ja näistä taivutustunnukset liittämällä (tyypiksi
padoiss<A><pl><ine> ja morfofonologiset (tyypiksi padoissa) vaihtelut
säännöin toteuttamalla päädytään transduktoriin, joka sisältää halutun morfologisen kuvauksen (tyyppiä pata<9><f><pl><ine> ↔ padoissa). Tämän
toteutuksen kuvauksen olen jakanut kahteen osaan: Aluksi kuvaan lyhyesti äärellistilaisten menetelmien pohjateoriaa ja esittelen käyttämiäni notaatioita kappaleessa 3.1. Kappaleessa kuvatuin merkinnöin ja käytännöin itse järjestelmä on
kuvattu kappaleessa 3.2.

Jäsentimen kattavuuden osalta on mainittava, että toteutus kattaa vain sanojen taivutuksen, eli sisältää nomineilta vain sijapäätteet, omistusliitteet ja kliitit sekä verbeiltä aika- ja tapamuodot, infinitiivit ja partisiipit, sekä persoonamuodot ja pääluokkataivutuksen. Taivutuksen oikeellisuuden perusteeksi on enimmäkseen oletettu, että Nykysuomen sanalistan luokat ovat oikein, ja kun käytetyt mallit kuvaavat osan sanoista per luokka oikein, muidenkin pitäisi kuvautua. Toteutus sisältää testaustarkoituksiin käytettävän naiivin yhdyssanamuodostussäännöstön, joka
muodostaa mielivaltaisen paljon perusyhdyssanoja, eli nomineista muodostuvia
yhdyssanoja, joiden määriteosa on joko yksikön nominatiivissa tai genetiivissä ja
vain jälkiosa taipuu.
Testitarkoituksiin valitut ei-avoimet, epävapaat korpukset ovat myös XMLaineistoa, joten niitäkin on käsitelty XSLT-pohjaisin menetelmin, lisäksi niistä
XSLT-menetelmillä irroitettua dataa on verrattu SFST:n morfologisiin jäsennyksiin Python-pohjaisilla menetelmillä, josta lyhyt kuvaus kappaleessa 3.3.

\footnote{10

http://www.w3.org/TR/2007/REC-xslt20-20070123/
}


\subsection{Äärellistilaisten menetelmien teoriasta}

Tässä kappaleessa kuvataan lyhyesti äärellistilaisten menetelmien teoriaa siltä
osin kuin se on tarpeellista suomen morfologian toteutuksessa ja samalla esitellään SFST-toteutuksen tapaa merkitä äärellistilaisia operaattoreita ja ilmauksia.
Laajempia ja kokonaisempia kuvauksia aiheesta löytyy vaikka teoksista Aho et al.
(2007); Beesley ja Karttunen (2004).

Äärellistilainen transduktori on teoreettinen malli, joka kuvaa äärellistilaisten
kielten relaatiota kuusikolla (Σ, Γ, S, s0 , F, δ), jossa Σ on ylemmän kielen aakkosto, Γ alemman kielen aakkosto, S on äärellinen joukko tiloja, s0 alkutila joukosta
S, F S:n osajoukko lopputiloja ja δ : S × (Σ ∪ 0) × (Γ ∪ 0) → S siirtymärelaatio
joka kuvaa tilalta toiselle aakkospareilla. (Jurafsky ja Martin, 2000)

Äärellistilaisten menetelmien havainnollistavassa kuvaamisessa käytetään yleensä
sellaista grafiikkaa, että tiloja merkitään ympyröillä tai ellipseillä, ja tilojen välisiä
siirtymärelaatioita merkitään nuolilla. Sekä tilat, että siirtymät ovat nimettyjä ja
siirtymien nimet muodostuvat aakkoston olioista. Tilojen nimet kirjoitetaan tässä
ympyrän sisälle, ja siirtymien nimet kaaren päälle tai sen läheisyyteen.

Säännölliset ilmaukset ovat nimitys menetelmille kuvata äärellistä automaattia
kuvaamansa kielen aakkosilla, jotka ovat mielivaltaisia kielen symboleja, kuten
esimerkiksi luonnollisessa kielessä konkreettisia latinalaisia aakkosia, sekä operaattoreilla, joilla mallinnetaan aakkosten tai aakkosjonojen mielivaltaisia joukkoopillisia yhdistelmiä ja toistoja. Morfologiaa käsitellessä relevantteja yksiköitä
ovat tietenkin aakkoset, joilla kirjoitetaan kielen sanoja, ja lisäksi on tyypillistä
käyttää joitain erikoismerkintöjä kuvaamaan sanojen taivutusmuotoja, morfofonologian abstrakteja lisämerkintöjä käsittelyn aikana ja muita erikoismerkkejä. Tyypillisimmät säännölliset ilmaukset käsittelevät yksitasoisia säännöllisiä automaatteja, tässä tutkielmassa on kyse useimmiten kaksitasoisista äärellisistä transduktoreista, joten kuvauksetkin on annettu hieman poikkeavalla SFST-PL-kielisellä
murteella säännöllisistä ilmauksista. SFST-PL poikkeaa myös osin muissa transduktorijärjestelmissä käytetyistä murteista, joten kuvaan tässä tarkemmin perusoperaatioita ensiksi SFST-PL-kielisesti. Muihin merkintöihin tottuneet voivat löytää liitteestä B taulukon, jossa on verrattu SFST:n notaatiota yleisempiin variantteihin.

Atomisimmat peruselementit äärellisissä järjestelmissä ovat aakkoston symboleista muodostuvien jonojen joukkoja, eli formaaleja kieliä, joita tässä merkitään symboleilla tyyppiä $L$. Aakkosten yksittäiset symbolit merkitään yksinkertaisesti a
ja symboliparit a : b.

Perusolioita äärellisissä verkoissa ovat universaalit ja tyhjät kielet (merkitään vastaavasti symboleilla .∗ ja <>) sekä useimmat niiden ja mielivaltaisten kielten
kombinaatio-operaattoreista, kuten ehdollisuus (so. nollasta yhteen toistoa merkitään postfiksaalisella kysymysmerkillä) ja valinnaisuus (nollasta äärettömään toistoa merkitään postfiksaalisella tähdellä).

Yhdiste (union) on loogisesti disjunktiivinen, jos siis meillä on kieli $L1 $, jossa on ’kissa’ ja kieli $L2 $, jossa on ’koira’, niiden yhdiste on kieli $L1 $|$L2 $,
jossa on ’kissa’ tai ’koira’. Konkatenaatio on merkkijonokäsittelyssä usein käytetty termi yksinkertaisesta peräkkäin asettamisesta, eläinkieltemme $L1 $ ja $L2 $
konkatenaatio tässä järjestyksessä olisi siis kieli $L1 $$L2 $, jossa on
’kissakoira’olio. Leikkaus (intersection) on taas loogisesti konjunktiivinen,
siis vaikkapa kielen L1,2 , jossa on ’kissa’ tai ’koira’ ja kielen L2,3 , jossa
on ’koira’ tai ’banaaniovi’ leikkaus on kieli \verb|$L1,2 $&$L2,3 $|, jossa on vain ’koira’. Erotus (difference)
on operaatio, jossa kielletään tai poistetaan asioita toisesta kielestä, esimerkiksi
$L1,2 $ − $L2 $ olisi $L1 $.

Monipuolisemmista operaatioista äärellisille relaatioille eli transduktoreille tyypillinen kompositio (composition) on tässä yhteydessä kuvaus ensimmäiseltä syötteeltä toiselle tulosteelle, jos ja vain jos ensimmäinen syöte kuvautuu ensimmäiselle tulosteelle, joka kuvautuu toisena syötteenä toiselle tulosteelle mielivaltaisen (kohdistetun) polun kautta. Esimerkiksi jos meillä on relaatio, jossa on pari
a : b ja komposition toisena osapuolena relaatio, jossa on pari b : c, komposition a : b||b : c tuloksena on relaatio, jossa a ja c ovat nyt pari a : c. Käytännön tasolla siis kompositiolla voidaan kuvata peräkkäisinä toteutumina yksittäiset säännöt siten, että välivaiheet jäävät aina pois näkyvistä. Esimerkiksi
a : b||b : c||c : d||d : e == a : e, joka morfofonologisessa maailmassa tarkoittaa, että voidaan soveltaa järjestyksessä peräkkäin sääntöjä siten, että edellisen lopputuloksella voi olla kontekstuaalinen vaikutus seuraavan sisältöön.

Sivuutuksella (ignoring) lisätään transduktorin joka väliin mielivaltaisen monta sivuutettavaa symbolia siten, että sääntönä se voisi sivuuttaa symboleita, jos ne ovat
käsittelyn kannalta irrelevantteja. Jos esimerkiksi haluaisimme tarkastella sanassa
esiintyviä vokaaleja, voisimme sivuuttaa kaikki konsonantit.

Kielten fonologisia relaatioita kuvatessa säännöllisten ilmausten lisäksi käytetään
sääntöjä, joista osa tunnetaan kaksitasosääntöinä (Koskenniemi, 1983). Näistä
kontekstirajoitusta (context restriction) määrää relaatiossa tai kielessä esiintyvän
symbolijonon mahdolliset esiintymäpaikat, eli jos symbolijono esiintyy, sillä on
oltava operaattorin määräämä konteksti. Esimerkiksi jos kielessä on morfofonologinen vaihtelu, joka muuttaa t:n d:ksi vain, kun se on umpitavun alussa vokaalien
välissä, voisimme rajoittaa parin t ja d tällaiseen ympäristöön V t => dV C. Kontekstirajoitus toimii relaationa siten, että se sallii parin kuvauksen jos se esiintyy
kontekstissa (kuten edellä).

Pintamuodon pakotus (surface coercion) toimii siten, että se määrää parin, jos
konteksti toteutuu, eli jos vaikkapa kielessä on nasaaliassimilaatio pakollinen, voitaisiin kuvata nasaalit pareina ääntöpaikaltaan sopiviin nasaaleihin kontekstin ääntöpaikan mukaan, esim. n <= mp. Näiden yhdistelmä loogisesti sekä sallii parin
esiintyä vain annetussa kontekstissa että pakottaa parin olemaan juuri määrätty
pari kun se esiintyy annetussa kontekstissa, eli määrää parin esiintymään jos ja
ainoastaan jos se esiintyy annetussa kontekstissa.

Toisinkirjoitus (replace) on operaatio, jolla kuvataan kieliä toisille ja relaatioita
toisille (Karttunen, 1995). Se on esimerkiksi kaksitasosääntöjä käytännöllisempi
operaatio tilanteessa, jossa relaatio, joka kuvataan, ei ole pari 1:1 vaan erimuotoinen (Schmid, 2007b), kuten esimerkiksi lavenevan diftongin ensikomponentin
kato tai säilyminen vanhoissa sanoissa voitaisiin kuvata \verb|uo : o ˆ→ __i.|

Toteuttamani järjestelmä toimii lopulta replace-sääntöjen järjestyksessä soveltamisella. Replace-sääntöjä yhdistetään sanalistasta tehtyyn transduktoriin sääntö
kerrallaan komposition avulla. Replace-sääntö rakentaa aina sellaisen transduktorin, joka kompositoidessa kuvaa kaiken säännön kannalta irrelevantin identiteetikseen, ja säännön alaiset merkkiparit säännön konteksteissaan väistämättä toteutettavaksi.

Tässä sanaluokitellun listan kanssa voimme aina valita kontekstiksi taivutusluokkanumeron kerrallaan, silloin kun se on tarpeen. Tästä esimerkiksi kun tiedämme, että vesi ja käsi ovat taivutusluokan 27 sanoja, voimme kuvata vain luokalle
27 kontekstilliset toisinkirjoitussäännöt, jolla perusmuodon lopun si tulee yksikön vokaalivartaloissa taivutusvartaloissa olemaan te tai de. Koska kompositoidessa sääntötransduktoria oikealle ylempi taso jää paikalleen, ja alemman tason
muutokset toteutuvat, säännöstön kompositoinnin jälkeen analyysitasolla on esimerkiksi käsi kun vartalon teon jälkeen generointitasolla voi olla käte.

Itse morfologian kuvauksen kannalta vähemmän relevantteja ovat
äärrellistilaisten verkkojen ylläpidolliset ja käytännölliset operaatiot, kuten
0-poisto, determinisointi, minimointi, vertailu jne., jotka sovelluksissa ovat sinänsä tarpeellisia,
ja joiden toteutuksen tehokkuus ja toiminta lopulta vaikuttavat järjestelmän suorituskykyyn. Tämän tutkielman vertailussa lienee merkillepantavaa, että SFSTjärjestelmässä nämä transduktorin ylläpidolliset toimenpiteet suorittaa SFST itse,
eikä niitä voi sääntöjärjestelmän kirjoittaja vahvasti hallita.


\subsection{Suomen kielen äärellistilaisen automaattisen morfologisen
jäsentimen toteutuksesta}

Tässä morfologian toteutuksessa lähdetään liikkeelle luettelosta sanakirjamuotoja taivutus- ja astevaihteluluokkineen ja päädytään transduktoriin, joka kuvaa jokaisen sanan jokaisen taivutusmuodon sanakirjamuotoonsa. Tämä prosessi koostuu käytännössä useasta pienestä muunnoksesta, jotka kuitenkin suurpiirteissään voi yleistää seuraaviin pääkohtiin: Alussa yksinkertainen XML-datana oleva sanalista, jossa yhtä sanaa voisi vastata vaikkapa tietue <st><s>valo</s><t><tn>1</t></st>. Tämä muunnetaan SFSTleksikkotiedostossa olevaksi sanalistaksi, joka kuvaa identiteettirelaation sanojen
perusmuodoista ja taivutusluokituksista, sekä muusta lisädatasta, joka järjestelmälle saattaa olla tarpeellista, kuten epäsäännöllisen vartalovokaalin olemassaolosta. Tämä tekstitiedosto koostuu riveistä, joista kullakin on yksi sana sanakirjamuodossaan, ja muu data kulmasulkein eroteltuna, esimerkiksi sana valo tulee
muotoon valo<N1>

Seuraavaksi sanoista muodostetaan replace-sääntöjen ja konkatenoitavien täydennysten avulla sellaiset taivutusvartalot joita voidaan käyttää taivutuksen lähtökohtana. Nämä taivutusvartalot ovat suunnilleen sellaisia, joihin voi konkatenoida
taivutuspäätteitä tekemättä suurempia muutoksia, ja käytännössä lähellä koulukielioppien käyttämiä suomen sanojen taivutusvartaloita (konsonanttivartalo, monikkovartalo jne.) (Setälä, 1930). Esimerkiksi sanalle joka on leksikossa muodossa käsi<N27> saadaan taivutusvartalot suunnilleen muotoon kä<~T>e<N27>.
Lopuksi konkatenoidaan taivutuspäätteet oikeisiin vartaloihin ja toteutetaan äännemuunnokset, jotka kuvaukseen kuuluvat, jolloin esimerkiksi päästään muotoon
kädessä, joka liittyy toisella tasolla olevaan muotoon käsi<N27><sg><ine>.
Lopullinen transduktori on kokonaisuudessaan muotoa, jossa on kaikkien sanojen
ja niiden taivutusmuotojen vastaavat relaatiot tällaisina yhdistelminä.

Malli, jolla olen tämän prosessin päätynyt tekemään (ks. kuva 2), on lineaarisesti
etenevä kuvaus, joka ottaa syötteenään edellisen vaiheen tulosteen, ja muodostaa uuden version äärellistilaisten operaatioiden avulla. Syöte ja tuloste ovat tässä
sanojen tai sanamuotojen lista transduktorina, joista periaatteessa joka vaiheessa
jokaista sanamuotoa ikään kuin operoidaan. Käytännön tasolla lista tietenkin äärellistilaisessa järjestelmässä on sanamuotolistan disjunktion kanssa ekvivalentti
kuvaus, mutta kuvauksen selventämiseksi esitän operaatiot kuin ne operoitaisiin
yksittäisille sanamuodoille kerrallaan. Lisäksi on tietysti niin, että operoinnin sijasta järjestelmä on vain sanalistatransduktorin, sääntötransduktorien ja operaatioiden yhdistelmä, mutta sen hahmottaminen sanamuotojen käsittelynä käsittelynä on mielestäni havainnollista ja perusteltua.

\begin{figure}
    \label{Morfologia-järjestelmän kaaviotaulukko}
    \begin{tabular}{|l|l|l|l|}
        \hline
        \bf Syöte & \bf Operaatio & \bf Operandi & \bf Tuloste \\
        \hline
        Nykysuomen & XSLT & XSLT-kohentimet & Parannettu \\
        XML-sanalista &   &                 & XML-sanalista \\
        \hline
        Paranneltu & XSLT & Leksikkomuunnin & SFST-leksikko \\
        XML-sanalista & & & \\
        \hline
        \hline
        SFST-leksikko & luku & Sanaston luku & Sanalistatransduktori\\
        \hline
        Sanalista- & kompositio & erillissiistintä, & \\
        transduktori & & typistyssäännöstö, & taivutusvartalo- \\
         &konkatenaatio & variantit vartalo-osat & listatransduktori \\
        \hline
        Taivutusvartalot & konkatenaatio & taivutuspäätteet &
        taivutusmuotolistatransduktori \\
        \hline
        Taivutusmuodot & kompositio & fonologia- & valmis sana- \\
        & & toteutussäännöt & listatransduktori \\
    \end{tabular}
\end{figure}

\begin{figure}
    \label{Äärellistilainen järjestelmän yksinkertaistettu hahmo koodina}
    \begin{verbatim}
(($sanalista$ || $taivutusvartaloiksi$)\
$taivutuspaatteet$)\
|| $fonologia$
    \end{verbatim}
\end{figure}


Kuvan 2 taulukkoa luetaan siten, että vasemmassa sarakkeessa on olio, jota operoidaan toisen sarakkeen operaattorilla kolmannen sarakkeen oliota vasten jotta
saadaan neljännen sarakkeen olio. Taulukosta on tarkoituksella jätetty pois toteutuksen detaljit, kuten siistinnät, minimoinnit ja filtteröinnit, jotka ovat melko triviaaleja ja muuttuvat järjestelmän eri versioiden välillä paljon; käytännössä turhia
merkkejä ja yligeneroivia konkatenaatioita filtteröidään aina kun se on järkevää.

Seikkakohtaisempi kuvaus koko järjestelmän toteutuksesta osa osalta on kappaleessa 3.2.2, ja vielä tarkemman kuvan saa tietenkin tarkastelemalla järjestelmän
lähdekoodia, joka on saatavilla osoitteesta https://gna.org/projects/
omorfi. Järjestelmän toteutus ei nähdäkseni sisällä ei-triviaaleja sääntöjä tai
uusia kehityksiä, joten sen selvittäminen lähdekoodista lienee helppoa.


\subsubsection{XML-sanalistan muunnos SFST-leksikoksi XSLT-menetelmällä}

Lähteenä käytettävä XML-sanalista ei ole sellaista muotoa, joka toimisi SFSTohjelman syötteenä, joten se on esikäsiteltävä ja muunnettava SFST:n käyttämään
leksikkomuotoon, joka on yksinkertainen rivipohjainen tekstiformaatti. XML-lähteen muuntamiseen on tässä päätetty käyttää myöskin XML-pohjaista XSLTmuunnoskieltä, jonka ensisijainen tarkoitus on puurakenteisten dokumenttien kuvaaminen. Tässä tapauksessa kohdemuoto ei ole varsinaisesti niinkään puurakenteinen kuin suora lista, mutta kuvaus toimii silti. Kuvauksen periaatteenahan
on ottaa XML-sanalistapuusta jokainen alkio ja lisätä siitä sanakirjamuoto sekä
taivutusluokitus sanalistaan. Lisäksi poikkeustapauksista joutuu tekemään pieniä
muunnoksia, kuten monta taivutusluokkaa sisältävien sanojen tulostaminen monesti.

XSLT-muunnoskieltä käytetään tutkielman projektissa kahteen tarkoitukseen, sen
lisäksi että sillä tehdään muunnos XML-muodosteesta SFST:n ymmärtämäksi lexluetelmaksi, käytettävissä on joukko XSLT-muunnoksia, jotka sisältävät tiettyjä
automaattisia korjauksia sanalistaan. Jälkimmäinen kohennusosio ei ole pakollista suorittaa ennen muunnosta lex-listaksi, mutta se parantaa monikkosanojen
ja vierasperäisten sanojen käsittelytarkkuutta. Sanalistamuunnoksen koodi on tiedostossa kotus.xslt.

Automaattisen kohennuksen suorittama koodi on jaettu tiedostoihin
xmlattribuutit.xslt, monikkosanat.xslt, monikkotaivutus.xslt ja
vartalovokaali.xslt.  Kuvauksien ohella on pätkiä koodeista, jota käytetään
muunnokseen, mutta täydet koodit kommenteine konteksteissaan löytyvät
verkko-osoitteesta https://gna.org/projects/omorfi.

XML-attribuutit lisäävä muunnos xmlattribuutit.xslt ei tee mitään muuta kuin tarkistaa tiedoston XML-muotovaatimuksia, kuten luonnollisen kielen määrittelevän @xml:lang-attribuutin sisältävän suomen kielikoodin, nimiavaruuden osoitteen @xmlns-attribuutin, johon olen väliaikaisesti sijoittanut oman osoitteeni http://www.helsinki.fi/~tapirine/xmlns/
experimental/kotus-sanalista, sekä uniikin tunnisteen @xml:idattribuutin, jolla olen merkinnyt version listasta omakseni ns. id-signature-arvolla
www-helsinki-fi-tapirinen juurialkiossa. XSLT-koodina tämä muunnos on triviaalisti templaatti, jossa on yllä mainitut attribuutit:
\begin{verbatim}
...
<xsl:element name="kotus-sanalista"
namespace="{$kotus-xmlns}">
<xsl:copy-of select="@*"/>
<xsl:attribute name="xml:id"
namespace="http://www.w3.org/XML/1998/namespace"
select="$xmlid-signature"/>
<xsl:attribute name="xml:lang"
namespace="http://www.w3.org/XML/1998/namespace"
select="’fi’"/>
...
\end{verbatim}

Monikkosanat arvaava käsittelin monikkosanat.xslt yrittää arvata sanamuodot,
jotka näyttävät monikoilta, eli karkeasti ne, joiden perusmuoto päättyy
t-kirjaimeen vaikka sen ei normaalisti kuuluisi, ja merkitsee näihin sanoihin
attribuutin @muoto, jonka arvona on NOM PL?, merkitsemässä arvattua nominatiivin
monikkoa. Esimerkiksi sana aivot kuten myös sana aivokuollut merkittäisiin tässä
mahdollisesti monikkosanaksi. Kysymysmerkki merkitseekin sellaista muotoa, joka
tulisi tarkistaa ja korjata käsin, ennen kuin sitä käytetään järjestelmässä.
Korjaaminen tapahtuu poistamalla kysymysmerkki, jos arvaus on oikein, tai
muuttamalla sen muodoksi NOM SG, jos kyseessä on yksiköllinen sana. Jos
attribuutti poistetaan niin tämä automaattinen tarkistus lisää sen uudelleen
seuraavalla tarkistuskerralla, joten poistaminen ei kannata. Lisäksi olen
kokeillut attribuuttia @perusmuoto, johon arvataan sanan rekonstruoitu
perusmuoto, eli sellainen jonka voisi olettaa yksikön nominatiivin olevan, jotta
sanan käsittely morfologisessa jäsentimessä sujuisi oikein. Tätä kenttää voisi
käyttää vaikkapa silloin, kun morfologisen järjestelmän taivutusosa ei osaa
palauttaa monikkomuotoista sanakirjamuotoa oikein käsittelyä varten. Uusista
attribuuteista on huomattava, että alustavasti olen sijoittanut ne toiseen
nimiavaruuteen kuin muut attribuutit käyttäjille vihjeeksi siitä, että kyseessä
on muualta tullutta, epävarmempaa dataa kuin Kotimaisten kielten
tutkimuskeskuksen antamaa. Monikkosanojen arvaus käsittellään XSLT 2.0:n
analyze-string-ominaisuudella jokaista sanan jokaista taivutusmuotoa kohden
seuraavasti:

\begin{verbatim}
<xsl:analyze-string select="s" regex="^(.*)t$">
<xsl:matching-substring>
<xsl:call-template name="tap:monikkoattribuutit">
<xsl:with-param name="yksikkö"
select="regex-group(1)"/>
</xsl:call-template>
</xsl:matching-substring>
</xsl:analyze-string>
\end{verbatim}

Monikkotaivutuksen merkitsevä käsittelin monikkotaivutus.xslt täydentää
monikkosanakäsittelintä sillä, että se merkitsee monikkosanojen taivutuksen
koskevan vain sanan monikkomuotoja, eli käyttää t-alkion olemassaolevan

@taivutus-attribuutin ennalta tunnettua arvoa monikossa. Monikkotaivutuksen
lisäyksen koodi tarkistaa yksinkertaisesti monikkovaiheessa lisätyn attribuutin:

\begin{verbatim}
<xsl:if test="../s/@tap:muoto = ’NOM PL’">
<xsl:attribute name="taivutus" select="’monikossa’"/>
</xsl:if>
\end{verbatim}

Vartalovokaalin arvaava käsittelin vartalovokaali.xslt on yksinkertainen
luokan 22 vierassanojen taivutusta helpottava koodi, joka pyrkii arvaamaan tietyntyyppisten sitaattilainojen äännerakenteen käyttäen pohjaoletuksenaan, että nämä ovat lainasanoja ranskasta tai englannista, kuten alkuperäisessä Nykysuomen
sanalistassa kaikille tämän luokan sanoille pätee. Vartalovokaali tarkoittaa tässä sitä sanan lopussa ääntyvää vokaalia, joka ratkaisee vokaalisoinnun ja illatiivin tunnuksen, esim: show : show’ta : show’hun tai bordeaux : bordeaux’ta :
bordeaux’hon. Tämä tunnistus on XSLT:ssä yksinkertainen merkkijonontäsmäysfunktio, jota kutsutaan jos sanan taivutusluokka on 22. Tunnistus on lähinnä esimerkinomainen, ja toimii vain sanoille, jotka olivat Nykysuomen sanalistan versiossa 1 mukana, muissa sitaattilainoissa sen arvaukset eivät välttämättä ole oikeita.

\begin{verbatim}
<xsl:if test="tn = 22">
<xsl:attribute name="vartalovokaali"
select="tap:vartalovokaali22(../s)"
namespace="{$minun-xmlns}"/>
</xsl:if>
...
<xsl:function name="tap:vartalovokaali22" as="xs:string">
<xsl:param name="sana" as="xs:string"/>
<xsl:choose>
<xsl:when test="matches($sana, ’illes$’)">i</xsl:when>
<xsl:when test="matches($sana, ’eaux$’)">o</xsl:when>
<xsl:when test="matches($sana, ’ait$’)">e</xsl:when>
<xsl:when test="matches($sana, ’et$’)">e</xsl:when>
<xsl:when test="matches($sana, ’at$’)">a</xsl:when>
<xsl:when test="matches($sana, ’ut$’)">u</xsl:when>
<xsl:when test="matches($sana, ’w$’)">u</xsl:when>
<xsl:otherwise><xsl:message>
Tunnistamaton vartalovokaali sanassa
<xsl:value-of select="$sana"/>
</xsl:message>?</xsl:otherwise>
</xsl:choose>
</xsl:function>
\end{verbatim}


Itse varsinainen muunnos XML-muodosta SFST-muotoon toimii suurpiirteissään
niin, että luetaan kotus-sanalista-puusta, eli tämän version juuresta, jokainen lapsi, joka on tyyppiä st, eli kaikki sanatietueet nykyisen määritelmän mukaan. Näistä tarkastetaan t-alkion eli taivutusdatan olemassaolo, ja viskataan pois
ne, joista se puuttuu. Seuraavaksi käydään läpi jokaisen tietueen kaikki t-lapset,
eli käsitellään saman sanan eri taivutusluokat eri olioina. Näistä tn-lasten sisällön perusteella lisätään muunnoksen tulokseen joko nominirivi, sanaluokille 1—
49 tai verbirivi sanaluokille 52—78, poimimalla tulosriviin sisällöt käsiteltävän
t-alkion sisaresta s (sanakirjamuotoinen sana) ja lapsesta tn (taivutusnumero)
sekä av (astevaihtelu), jos sellainen on olemassa. Lisäksi jos astevaihtelussa on
attribuutti @astevaihtelu=’valinnainen’, tulostetaan sama sana kahtena rivinä, kerran astevaihtelun kera ja kerran ilman. Jos sanalistaa on kohennettu
XSLT-skripteilläni ja monikkosanat merkitty monikkotaivutus-merkillä, lisätään
tämä tieto mukaan riville. Kuvassa 4 on otos muunnoksen vaiheista. Koodi, joka toteuttaa muunnoksen, koostuu jokaisen sanatietueen jokaisen taivutustiedon
läpikäynnistä ja datojen ylöskirjoituksesta seuraavasti:

\begin{verbatim}
<xsl:template match="st">
<xsl:if test="not(t)">
<xsl:message terminate="no">
Sanaa <xsl:value-of select="s"/> ei tulostettu
leksikkoon, koska sen taivutusluokka uupuu.
</xsl:message>
</xsl:if>
<xsl:for-each select="t">
<xsl:choose>
<xsl:when test="1 &lt;= tn and tn &lt;= 49">
<xsl:value-of select="../s"/>
«N<xsl:value-of select="tn"/>»
<xsl:if test="av">
«AV<xsl:value-of select="av"/>»
</xsl:if>
<xsl:if test="@taivutus=’monikossa’">«PLT»</xsl:if>
</xsl:when>
<xsl:when test="51 &lt;= tn and tn &lt;= 78">
...
</xsl:when
</xsl:choose>
</xsl:for-each>
</xsl:template>
\end{verbatim}



\begin{figure}
    \label{XSLT-muunnoksen vaiheita}

kotus-sanalista\_v1.xml:\\
    \begin{verbatim}
<kotus-sanalista>
<st><s>valo</s><t><tn>1</tn></t></st>
<st><s>valoaallot</s></st>
<st><s>valohämy</s></st>
<st><s>valoisasti</s><t><tn>99</tn></t></st>
<st><s>valottaa</s><t><tn>53</tn><av>C</av></t></st>
<st><s>valvojaiset</s><t><tn>38</tn></t></st>
<st><s>uros</s><t><tn>39</tn></t>
<t><tn>41</tn></t></st>
    \end{verbatim}
→kotus-sanalista\_v1-r1.xml:\\
    \begin{verbatim}
<kotus-sanalista
xmlns="http://www.helsinki.fi...
xmlns:tap="http://www.helsinki.fi/~tapirine/..."
xml:id="www-helsinki-fi-tapirine1" xml:lang="fi">
<st><s>valo</s><t><tn>1</tn></t></st>
<st><s tap:muoto="NOM PL?"
tap:perusmuoto="valoaallo">valoaallot</s></st>
<st><s>valohämy</s></st>
<st><s>valoisasti</s><t><tn>99</tn></t></st>
<st><s>valottaa</s><t><tn>53</tn><av>C</av></t></st>
<st><s tap:muoto="NOM PL?"
tap:perusmuoto="valvojainen">valvojaiset</s><t
taivutus="monikossa"><tn>38</tn></t></st>
<st><s>uros</s><t><tn>39</tn></t>
<t><tn>41</tn></t></st>
    \end{verbatim}
→kotus-sanalista.lex:\\
    \begin{verbatim}
valo<N1>
valottaa<N53><AVC>
valvojaiset<N38><PLT>
uros<N39>
uros<N41>
    \end{verbatim}
\end{figure}

\subsubsection{Äärellistilaisen järjestelmän konkreettiset SFST-moduulit}

Äärellistilainen järjestelmä, joka käsittelee sanalistaa toimii siten, että se lukee
sanalistan ulkoisesta datalähteestä, leksikkotiedostosta, joka on kappaleessa 3.2.1
kuvatun XSLT-muunnoksen tulostiedosto. Tästä sanalistasta järjestelmä toteuttaa
vartaloissa esiintyvät äännemuutokset, liittää sanoihin taivutuspäätteet ja toteuttaa
niissä esiintyvät ja niiden aiheuttamat äännemuutokset. Lopputuloksena saadaan
yksi transduktori, joka kuvaa sanojen kaikki taivutusmuodot niiden perusmuotoihin ja taivutusdataan, eli transduktori, joka on ekvivalentti kaikkien sanamuotojen
disjunktion parin sanakirjamuotojen ja kuvausten kanssa.
Systemaattinen toteutus, jolla viitetaulukkojeni (1 s. 20 ja 2 s. 24) kautta pääsee
perusmuotolistasta tynkä—vartalo—päätemallin kautta lopputulokseen, on sellainen, että aluksi perusmuodosta leikataan vartalon variantti osa taivutusluokkanumeron perusteella. Esimerkiksi nyt sanasta lapsi leikataan luokan 29 perusteella
pois loppuosa psi, jolloin jäljelle jää vaihtelematon osa la sekä taivutusluokkanumeron tunnus 29. Tämän jälkeen lisätään täydennykseksi eri vartaloihin täydentävät variantit osat vartalosta, esimerkiksi la-tynkään s konsonanttivartaloa varten ja
pse yksikkövartaloa varten. Soveltuviin vartaloihin edelleen lisätään taivutuspäätteet, siis konsonanttivartaloon las muun muassa monikon genetiivin tunnus ten tai
yksikän partitiivin tunnus ta. Taivutuspäätteistä osassa käytetään morfofonologisia muotoja, joissa esimerkiksi sointuvokaali a tai ä on kuvattu yhteisellä merkillä, jonka oikea muoto päätetään seuraavassa vaiheessa. Esimerkiksi muodon lasta
partitiivin tunnus on muodossa t<~A>, ennen kuin vokaalisointu käsitellään. Tarvittavat yleistykset saadaan katsomalla tuleeko joihinkin taivutusluokkanumeroihin samoja sääntöjä tai lisäyksiä. Tässä mallissa vartalot ovat lähellä koulukielioppien kuvauksia sanojen taivutusvartaloista, esimerkiksi nomineille muodostuvat
monikkovartalot ja usein konsonanttivartalot ovat täsmälleen odotuksenmukaisia.
Käytännön tasolla äärellistilainen järjestelmä on jaettu aiheittain moduuleihin, ja
jokaisessa moduulissa on joukko transduktoreja, jotka toteuttavat yhden konseptuaalisen muunnoksen sanalistalle. Jokainen moduuli on toteutettu siten, että se
ottaa syötteekseen sanalistan edellisestä moduulista tulleen version, ja yleensä
replace-säännöstön ja kompositio-operaation tai konkatenaatio-operaation ja disjunktioluettelotransduktorin avulla muodostaa uuden sanalistan.
Moduulit on jaettu siten, että jos jonkin osan järjestelmästä voisi toteuttaa toisella lähestymistavalla, olisi se toteutettavissa nimenomaan joitakin moduuleja korvaamalla. Esimerkiksi muunnos sanalistasta taivutusvartaloihin on tässä järjestelmässä kahden moduulin avulla tehty typistys-ja-täydennysoperaatio, mahdollista
olisi myös hoitaa sama yhdellä replace-operaatiolla ja korvata nämä kaksi moduulia sillä. Teoriassa jokaisen moduulin transduktori olisi mahdollista toteuttaa
myös sanalistasta riippumattomasti ja kääntää erilliseen tiedostoon, ja kompositoida keskeisesti sanalistan kanssa, mutta tätä koettaessani törmäsin rajoitteisiin
SFST-järjestelmässä, joita kuvaan tarkemmin kappaleessa 5. Tämä rajoite tekee
järjestelmän muuttelemisen vähemmän triviaaliksi, mutta toteutettevaksi tehtäväksi.

Järjestelmän
moduulit
ovat
suoritusjärjestyksessä
sanat.sfst,
yksikoksi.sfst,
paikanna-av.sfst,
typista.sfst,
vartaloiksi.sfst,
taivutus.sfst,
fonologia.sfst
ja
siisti.sfst. Lisäksi järjestelmässä on omorfi.sfst, joka keräilee
lopullisen sanatransduktorin kasaan. Lisäksi järjestelmässä on erillisessä moduulissa aakkosto.sfst, joka sisältää apumäärityksiä järjestelmässä käytettyjen
merkkien ryhmittelyjen suhteen.

Koodia kuvattaessa olen esittänyt pätkiä koodeista, jotka ovat relevantteja välttäen
samalla saman koodin toistoa tai trivialiteettien esittelyä. Täydet ja ajantasaisimmat koodit on helpointa saada luettavakseen osoitteesta https://gna.org/
projects/omorfi.

Käytännön tasolla moduulit yhdistyvät järjestyksessä toisiinsa kaavalla, joka on
yksinkertaistetussa muodossaan kuvassa 5. Kuvan SFST-tyylinen pseudokoodinotaatio kuvaa niitä operaatioita, joita järjestelmän transduktoreihin sovelletaan.
Koodin lopussa on havainnollisia esimerkkejä siitä, mitä tyyppiä itse transduktorit
aina ovat.

Järjestelmä toimii järjestyksessä siten, että sanat.sfst lukee sanalistan identiteettikuvausten disjunktio -transduktoriksi, jossa on sanat muodossa
\verb|<wb>sana<tl><av><plt><^x>|, jossa <wb> on sanaraja, sana on sana
sanakirjamuodossaan, <tl> taivutusluokan tunnus, <av> astevaihtelun tunnus,
jos sellainen on, <plt> monikkosanatunnus ja \verb|<^x>11| poikkeavan vartalovokaalin tai vokaalisoinnun merkki — esimerkiksi tikkaat41−A luettaisiin muodossa
\verb|<wb>tikkaat<N41><AVA><PLT>|.

Yksikoksi.sfst on kompositoitava säännöstö, joka keksii monikkosanojen
yksikön nominatiiveja taivutusta varten, esimerkiksi tikkaat-sanasta säännöt tekisivät muodon <wb>tikas<N41><AVA><PLT>.
Paikanna-av.sfst on kompositoitava replace-säännöstö, joka etsii astevaihtelun alaisen äänteen taivutusluokkanumeron ja astevaihtelukirjaimen perusteella
ja merkitsee sen astevaihtelevaksi symbolilla <~k>, <~p> tai <~t>, esimerkiksi tikkaat-sanasta säännöt tekisivät <wb>tik<~k>as<N41><AVA><PLT>.

\footnote{11
käytetyissä tunnuksissa ylipäätään on noudatettu sellaista kaavaa, että
suuraakkosin kirjoitetaan lähinnä konkatenoinnissa ja filtteröinnissä käytetyt
leksikkokoodin kaltaiset merkit, sirkumfleksillä alkavat merkinnät ovat
muistiinpanoja, jotka eivät realisoidu vaan esittävät kontekstin ja
aaltoviivalla alkavat ovat morfofoneemeja, joilla on useita mahdollisia
reaalistumia
}


\begin{figure}
    \caption{Järjestelmän koodit kaavana}
    \begin{verbatim}
sanat || yksikoksi1 || · · · || yksikoksi24
|| paikanna-av1 || · · · || paikanna-av26
|| typista1 || · · · || typista12
typistetyt vartaloiksi
(vartalot || vartalofiltteri1 ) |
··· |
(vartalot || vartalofiltteri18 )
vartalot taivutukset.
(taivutetut || taivutusfiltteri1 ) |
··· |
(taivutetut || taivutusfiltteri12 ) ||
· · · ||
taivutusfiltteri18 ||
fonologia1 ||
· · · ||
fonologia36 .
siisti1 || fonologisoidut || siisti2 .
siisti3 || siisti4 || siistityt.\end{verbatim}
, jossa: $yksikoksi_n$ tyyppiä
\verb|(t:<> ˆ→ (__<NX><PLT>) )|
$paikanna-av_n$ tyyppiä
\verb|(k:<k> ˆ→ (__[ˆk]$vahva$<AVA>]) )|
$typista_n$ tyyppiä
\verb|(i:<> ˆ→ (__<N5>) )|
vartaloiksi on
\verb_<N5>i<VNNOMI> | <N5>e<VNYKSI> | · · ·_
vartalofiltterin tyyppiä
\verb_.* <N1> .* <N1> .*_
taivutukset on
\verb_<VNYKSI>lle<sg><all> | <VNMONI>ille<pl><all> | · · ·_
$taivutusfiltteri_n$ , n ∈ [1, 12] tyyppiä
\verb_.* <PLT> .* <pl> .* | [ˆ<PLT><pl>]_
$taivutusfiltteri_i$ , i ∈ [13, 18] tyyppiä
\verb_.* <VNNOMI> .* <VNNOMI> .*_
$fonologia_n$ tyyppiä
\verb|(<n>:l ˆ→ (l [ˆ#aakkoset#]* __)|
$siisti_n$ tyyppiä
\verb_.*_
\end{figure}

Typista.sfst on kompositoitava replace-säännöstö, joka kuvaa sanaluokittain
sanojen vaihtelun alaisen osan vartalosta tyhjäksi ja ottaa muistiin
mahdollisesti kadotetun vartalovokaalin, esimerkiksi sanasta tikkaat säännöillä
tulisi muoto \verb|<wb>tik<~k><^a><N41><AVA><PLT>|.

Vartaloiksi.sfst on konkatenoitava luetelma niistä vaihteluvarianteista vartalon
osista, joista nominatiivia vastaava osa typistettäessä poistettiin, esimerkiksi
sanasta tikkaat saisi muun muassa monikkovartalon
\verb|<wb>tik<~k><^a><N41><AVA><~a><VNMONI>|. Lisäämällä nämä taivutusluokkakohtaisesti
saadaan aikaan kaikki sanan mahdolliset taivutusvartalot, jotka suunnilleen
vastaavat Nykysuomen sanalistan mallitaivutustaulukon rivejä pienin muutoksin ja
yleistyksin.

Taivutus.sfst on konkatenoitava luetelma kaikista taivutuspäätekombinaatioista,
jotka lisätään soveltuviin taivutusvartaloihin, eli sijamuotojen, possessiivien
ja kliittien kombinaatioista nomineille ja aika-, tapa ja persoonamuotojen sekä
kliittien ja infiniittimuotojen sekä nominaalitaivutusten kombinaatioista
verbeille. Sanaan tikkaat lisäämällä saisi muun muassa monikon inessiivin
\verb|<wb>tik<~k><^a><N41><AVA><PLT><~a><pl>i<ine>ss<~a>|.

Fonologia.sfst sisältää säännöt niille morfofonologisille yleistyksille, jotka järjestelmässä ovat olleet konkatenaatiokäsittelyn yksinkertaistamiseksi, kuten vokaalisoinnun ratkaiseminen päätteiden tunnuksissa tai astevaihtelun toteutus. Tässä vaiheessa sanan tikkaat monikon inessiivi päätyisi muotoon
\verb|<wb>tikk<^a><N41><AVA><PLT>a<pl>i<ine>ssa|.

Siisti.sfst poistaa vielä transduktorista kaikki väliaikaiset symbolit, jota
transduktorien rakennuksessa, yhdistelyssä ja suodattamisessa on käytetty apumerkkeinä, kuten leksikkosymbolit ja odotuksenvastaisen vokaalisoinnun merkitsevät symbolit, jolloin sanan tikkaat generointitasolle saadaan tikkaissa.
Tässä vaiheessa lienee huomion arvoista, että analyysitasoon ei ole koskettukaan paitsi potentiaalisesti konkatenaatio-operaatioissa, joten analyysitasolle jää sanat.sfst:n tikkaat<N41> ja taivutus.sfst:ssä molemmille tasoille konkatenoitu tunnistejono <pl><ine>, eli haluttu analyysi
tikkaat<N41><pl><ine>.

Yhteenveto taivutusprosessista on taulukossa 6. Taulukossa on mallisana, joka
kuvaa järjestelmässä etenemistä taso tasolta kuten se varsinaisessa järjestelmässä toimii siten, että ylemmällä rivillä on transduktorin analyysitaso ja alemmalla rivillä on transduktorin generaatiotaso. Merkintöjä on yksinkertaistettu siten
että vain merkitykselliset merkit näkyvät kullakin tasolla, käytännössä merkit
<N41><AVA><PLT> kulkevat mukana aina siistintään asti molemmilla tasoilla, samoin kuin uudet mukaan tulevat merkit, kuten leksikkosymboli <VNMONI>,
mutta tarkemmin toteutuksesta saa selville itse lähdekoodista.

\begin{figure}
    \caption{Äärellistilaisen järjestelmän moduulit ja niiden tuottamat
    transduktorit
    järjestyksessä}
    \begin{tabular}{|l|l|}
        \hline
        \bf Moduuli & \bf Esimerkki \\
        \hline
        sanat & \verb|<wb>tikkaat<N41><AVA><PLT>| \\
        & \verb|<wb>tikkaat<N41><AVA><PLT>| \\
        \hline
        yksikoksi & \verb|tikkaat<N41><AVA>| \\
        & \verb|tik as <N41><AVA>| \\
        \hline
        paikanna-av & \verb|tik k aat<N41><AVA>| \\
        & \verb|tik <~k>as <N41><AVA>| \\
        \hline
        typista & \verb|tik k aat <N41>| \\
        & \verb|tik <~k><^a><N41>| \\
        \hline
        vartaloiksi & \verb|tikk aat <N41>| \\
        & \verb_tik<~k><^a><N41><~a><VNMONI> | . . ._ \\
        \hline
        taivutus & \verb|tikk aat <N41><pl><ine>| \\
        & \verb_tik<~k><^a><~a>iss<~a><N41><pl><ine> | . . ._ \\
        \hline
        fonologia & \verb|tikkaat <N41><pl><ine>| \\
        & \verb_tikkaissa<N41><pl><ine> | . . ._ \\
        \hline
        siisti & \verb_tikkaat<N41><pl><ine>_ \\
        & \verb_tikkaissa | . . ._ \\
        \hline
    \end{tabular}
\end{figure}


\subsection{Korpusmenetelmät}

Testaamiseen käytetty korpusdata on CSC — Tieteellinen laskenta Oy:n ylläpitämässä Suomen tekstipankissa olevaa TEI XML:stä laajennettua XML-merkattua
dataa, ja sitä käsitellään testattavaksi samankaltaisilla XSLT-menetelmillä kuin sanalistan koodi. XSLT-käsittelyn tuloksena saatu data käsitellään Python-kielisellä
sovelluksella, joka käyttää pysfst12 -nimistä liitäntää analysoidakseen korpuksen
sanamuotoja tekemälläni transduktorilla, sekä verratakseen analyysijärjestelmäni
antamia tuloksia tekstipankissa oleviin merkintöihin, jotka eivät ole täysin vastaavia. XSLT-käsittelin on kuvattu lyhyesti kappaleessa 4, ja python-käsittelimen lähdekoodit löytää helpoiten omorfin jakelupaketista tai verkosta https://gna.
org/projects/omorfi.

\footnote{12

https://gna.org/projects/pysfst
}

\section{Testiasetelma ja evaluointi}

Rakentamastani morfologisesta järjestelmästä testaan kahta asiaa. Ensin selvitän
käytetyn muoto-opin kuvauksen toimivuutta ja onnistuneisuutta, sekä sanaslistan
kattavuutta testaamalla korpus-aineistoa vasten jäsentimen saantia, tarkkuutta ja
kattavuutta. Toiseksi testaan käytetyn SFST-järjestelmän suorituskykyä ajastamalla ja mittaamalla muistinkäyttöä sekä järjestelmän suoritustilanteessa, että järjestelmän rakentaessa transduktoria säännöstöistä ja operaatioista sekä sanalistasta.

Korpustestissä käytin aineistona CSC:n ylläpitämästä kielipankista löytyvää vapaaseen käyttöön tarkoitettua, valmiiksi morfosyntaktisesti jäsennettyä korpusta,
jonka morfologisia tulkintoja ei ole käsin tarkistettu. Morfologisen tulkinnan vertausaineistona on siis toisen automaattisen jäsentimen antamia tulkintoja juoksevasta sanomalehtitekstistä.

Suorituskykymittaukset olen tehnyt yksinkertaisesti mittaamalla suorittimen- ja
muistinkäyttöä järjestelmästä tavallisilla GNU-työkaluilla. Tällä tavoin olen saanut riittävästi suuntaa antavia tuloksia, joita voi käyttää suorituskyvyn havaitsemisessa ja vertailussa, mutta tarkemmat tutkimukset olen sivuuttanut. Suorituskykymittauksissa vertailukohteena on toisen äärellistilaisen järjestelmän mittaukset
olosuhteissa, jotka on pyritty mahdollisimman tarkkaan toisintamaan vastaamaan
omaa järjestelmääni SFST:ssä.

Kaikki kappaleessa kuvatut testit on suoritettu CSC — tieteellinen laskenta Oy:n
koneessa, joka sijaitsi verkko-osoitteessa corpus3.csc.fi. Testauskoneen
valinta perustuu siihen, että korpukset on lisensoitu tavoin, joka ei salli käyttää
niitä muualla, joten corpus3 on ainoa palvelin jolla testauksen voi suorittaa.

\subsection{Korpusmittaukset}

Korpusmittauksissa vertaan järjestelmäni antamia analyysejä koneellisesti yhteen
kielipankin korpusaineistossa vapaalla B-lisenssillä käytössä olevaan, valmiiksi analysoituun aineistoon, joka on otettu sanomalehti Karjalaisen vuosikerrasta 1992. Se sisältää 3 miljoonaa juoksevaa sanetta, joista 2 716 651 on laskettu mukaan analysoitavaksi. Korpusaineiston esikäsittelyn olen toteuttanut XSLTohjelmointikielellä, sillä esivalmisteltu korpusaineisto on laajennetussa TEI.2
XML -muodossa. Esikäsittelyni koostuu ainoastaan sanojen poiminnasta analyysiä varten siten, että jos sanan luokaksi on merkitty substantiivi, adjektiivi tai verbi, se lisätään analysoitavaksi. Koodi joka tuottaa analyysirivit XML-lähteestä on
seuraava:

\begin{verbatim}
<xsl:template match="s">
<xsl:for-each select="w">
<xsl:if test="@type=’Noun’ or
@type=’Verb’ or
@type=’Adjective’">
<xsl:value-of select="."/>,
<xsl:value-of select="@lemma"/>,
<xsl:value-of select="@msd"/>
<xsl:text>&#xA;</xsl:text>
</xsl:if>
</xsl:for-each>
</xsl:template>
\end{verbatim}

Näin suoritetulla koodilla saadaan aineistosta tulokseksi yksinkertainen csvtyyppinen tiedosto, jossa jokaisella rivillä on kolme pilkuin erotettua kenttää, jotka sisältävät järjestyksessä saneen, sanan sanakirjamuodon, ja morfologiset merkinnät. Tässä tiedostossa jokainen yksittäinen sanamuoto saattaa esiintyä mielivaltaisen monta kertaa, ja yleiset sanat ja sanamuodot vievätkin suuren määrän
esiintymistä. Tästä aiheutuvan virhetulkinnan kiertämiseksi suoritin myös toisen
testauksen, johon jokainen sanamuoto otetaan mukaan vain kerran. Tästä tiedostomuodosta tällainen aineisto, jota nimitän tyypittäiseksi sanalistaksi, tuotetaan
komennolla sort | uniq; tässä sanalistassa jokaisen sanan esiintymä on ainoalaatuinen, eli kahta saman sanan samamuotoista analyysiä ei esiinny kahdesti.

Automaattinen analyysi on suoritettu python-kielisellä sovelluksella pysfstmoduulia käyttäen. Analyysi toimii siten, että jokaisesta sanamuodosta tuotetaan
järjestelmällä kaikki tulkinnat, ja niistä tulkinnoista, joiden perusmuodot ovat samat, yritetään yhdistää kaikki kielipankissa olevat morfologiset merkinnät toteuttamani järjestelmän vastaaviin. Koska kielipankin aineiston tulkinnat eivät aivan
täsmää järjestelmässä käytettyihin, olen pyrkinyt olettamaan niitä samaan suuntaan, siten, että esim. genetiivin tai nominatiivin possessiivin vaihtelut hyväksytään aina yhtenäisinä — itse merkitsen ne kahtena eri analyysinä ja aineisto
käyttää erillistä genom-symbolia. Tarkka suoritettu testauskoodi löytyy osoitteesta https://gna.org/projects/omorfi.

Korpusmittaukset on suoritettu järjestelmän sellaisella versiolla, johon on otettu
mukaan vapaasti kliittejä sanojen loppuun mielivaltainen määrä, sekä partisiippimuotojen ja tekemisjohdoksen eli vanhan IV infinitiivin vapaa produktio ja taivutus. Yhdyssanajohdosta mukana on hyvin karkea nominipareja yhdistelevä versio.
Se miten tämä vaikuttaa mittaustuloksiin on kuvattu seikkaperäisemmin kappaleessa 5.4.

Seuraavissa kaavoissa merkitsen sanoiksi jokaista yksittäistä sanamuotoa ja analyyseiksi jokaista erillistä analyysia, joita minun järjestelmässäni tulee 1—n sanaa
kohti ja korpuksessa on 1 per sana. Kielipankin tuntemiksi analyyseiksi niitä, jotka löytyvät korpuksesta ja joille on annettu analyysi, jonka olen ottanut mittaukseen. Oikea analyysi on sellainen, jossa minun järjestelmäni on tuottanut oikean
perusmuodon, ja morfologiset analyysit ovat samat tai vastaavat kuin korpuksessa, kaikki muut analyysit ovat vääriä. Nykysuomen sanalistan sanoiksi lasketaan
sellaiset, joista kielipankin korpuksen analyysin mukainen perusmuoto löytyy Nykysuomen sanalistasta.
Tulosten kattavuus aineiston yli on laskettu yksinkertaisesti kaavalla

\begin{equation}
\mathrm{Kattavuus} =
    \frac{\mathrm{Järjestelmäni oikeat analyysit}}{\mathrm{Kielipankin
    analysoidut sanat}}\mathrm{,}
\end{equation}

joka kertoo miten suuren osan korpuksen kaikista analysoiduista, mukaan lasketuista sanoista järjestelmästä löytyy oikeina tulkintoina.

Toiset korpusanalyysitestit on suoritettu poimimalla korpuksesta niiden sanojen
muodot, jotka löytyvät myös Nykysuomen sanalistasta, ja testattu niiden analyysien kannalta kuinka tarkka jäsennin on ja miten paljon siinä on vielä puutteita tai
virheitä. Tavoitearvona oli tietysti, että järjestelmä tunnistaa siinä olevien sanojen
sanamuodoista kaikki 100 \%, mutta systemaattisten puutteiden lisäksi molempien
järjestelmien virheet ja eri analyysit tietysti laskevat tätä arvoa.

\begin{equation}
    \mathrm{Saanti} =
    \frac{\mathrm{Järjestelmäni oikeat analyysit}}{\mathrm{Nykysuomen sanalistan
    sanat ∩ Kielipankin analysoidut sanat}}\mathrm{,}
\end{equation}

joka kertoo miten suuri osa niistä analyyseistä, joiden pitäisi onnistua, menee oikein.
Järjestelmän tarkkuus taas on arvo, joka kertoo, että miten suuri osa annetuista
analyyseistä menee oikein niiden sanojen kohdalla, joista tulee useita analyysejä. Koska vertailukohtana tässä on disambiguoiva järjestelmä ja oma järjestelmäni
antaa kaikki mahdolliset tulkinnat, kuvaa tarkkuus-tulos lähinnä suomen kielen
taivutusopissa esiintyvää ambiguiteettipotentiaalia, ja saatu tulos on siis odotuksenmukainen.

\begin{equation}
    \mathrm{Tarkkuus} =
    \frac{\mathrm{Järjestelmäni oikeat analyysit}}{\mathrm{Järjestelmäni kaikki
    analyysit}}\mathrm{,}
\end{equation}

\begin{figure}
    \caption{Järjestelmän analyysien kattavuus}
    \begin{tabular}{c|l|l|l}
        \hline
        \bf Testi→ & \bf Kattavuus & \bf Saanti &\bf Tarkkuus \\
        \bf ↓ Korpus & & &\\
        \bf Karjalainen 1992 & 76,8 \% & 97,0 \% & 54,0 \% \\
        2 716 651 sanamuotoa & & & \\
        \hline
        367 242 tyyppiä & 50,1 \% & 96,2 \% & 47,0 \% \\
        \hline
    \end{tabular}
\end{figure}

\subsection{Suorituskykymittaukset}

Suorituskykymittaukseni perustuvat yksinkertaisiin muistin ja prosessoriajan käytön mittauksiin järjestelmää rakennettaessa ja käytettäessä. Mittaustyökaluina toimivat GNU-järjestelmistä löytyvät muistin- ja ajankäytön mittaustyökalukomennot time ja top. Mitatut ajat ovat timen mukaisia käyttäjän kuluttamia prosessoriaikoja, eivätkä varsinaisia suoritusaikoja. Todellisuudessa esim. FSM-kääntö on
huomattavasti hitaampi kuin SFST-kääntö ja SFST:n analyysi optimoimattomalla transduktorilla kestää kertaluokittain enemmän kuin optimoidulla. Laskettaessa suorituskykyä järjestelmästä käytettiin sellaista versiota, joista otettiin mukaan
vain perusmuotojen taivutukset, eli substantiiveista sijamuodot ja possessiivisuffiksit sekä verbeistä persoona- ja tapamudot sekä infiniittimuodoista perusmuodot;
kaikenkaikkiaan 7 541 661 sanamuotoa. Mukana ei ollut yhdyssanamekanismia,
verbin nominaalimuotojen edelleentaivutusta tai arvainta. Versiossa, joka muodostaa tämän on SFST:n tulostiedostossa 25 371 tilaa ja 60 389 kaarta. Järjestelmän aakkoston koko on 456 merkkiä, johon sisältyvät suomen kielessä käytettävät
aakkosmerkit, joitakin lainasanoissa käytettyjä latinalaisten aakkosten merkkejä,
taivutusluokkien ja muotojen analyysimerkit sekä järjestelmän sisäisiä
väliaikaisia analyysimerkkejä, kuten astevaihtelun merkinnät. SFST:stä on
käytetty mittaustesteissä versiota 1.1 ja AT\&T:n FSMlibistä versiota 4.0. Mittaustuloksissa on
vähintään kolmen eri mittauksen keskiarvo; tuloksia mitatessa ei kuitenkaan eri
mittauskertojen välillä ollut merkittävää hajontaa.

AT\&T:n FSMlibin muodostama transduktori on kooltaan 29 250 tilaa ja 463 121
kaarta ja käytetty aakkosto sama kuin mikä SFST:ssä. Aakkoston osalta huomattavaa kuitenkin on, että FSM määrittelee aakkoston erikseen generointi- ja analyysitasolle, kun taas SFST määrittelee aakkoston symboliparien aakkostona, mikä
selittänee ainakin osan siitä miksi samalla menetelmällä tuotetut lopulliset transduktorit järjestelmien välillä ovat merkittävästi eri kokoiset.

AT\&T:n FSMlib-järjestelmän testaaminen on suoritettu ottamalla SFSTjärjestelmästä erilliset transduktorit irti »-operaattorilla, joka kirjoittaa käsiteltävän transduktorin binääritiedostoksi, ja sen jälkeen muuntamalla SFST:n
automaatti fst-print-komennolla erääseen tekstimuotoon ja muuntamalla
siitä skriptein FSM:n tekstimuotoon. Tästä lukemalla fsmcompilella saadaan FSM:n binäärimuoto. Mitatussa ajassa on mukana se fsmcompose-,
fsmconcat- ja fsmunion-operaatioiden summa, jotka korvaavat SFSTjärjestelmässä
olleita ||-, konkatenaatio- ja \&-operaatiota, vastaavasti. Näiden
operaatioiden lisäksi käytin fsmrmepsilon- ja fsmdeterminize-käskyjä
vastaavasti kuin SFST-järjestelmässä, eli joka askeleen jälkeen, ja lisäksi
transduktorien muotoina oli \verb|input_indexed| tai \verb|output_indexed|.Lopullisen transduktorin pyrin lisäksi optimoimaan fsmencode- ja fsmminimizekomennoilla.

Vastaavasti SFST:n käännön nopeustestaaminen on suoritettu samoilla, valmiiksi determinisoiduilla ja minimoiduilla transduktoreilla. Jäljelle jääneiden erojen
merkitystä ja syitä pohdin tarkemmin kappaleessa 5.

Sekä AT\&T:n FSM että SFST tukevat erilaisia transduktoriformaatteja, eli käytännössä eri tavalla optimoituun muotoon kirjoitettuja transduktoribinäärejä. Testeissä SFST:ltä on kokeiltu kaikkia kolmeja formaatteja siten, että lopullinen transduktori on tuotettu fst-compiler-utf8 -c- tai fst-compiler-utf8
-l-komenolla sarakkeen mukaan. Näistä -c tuottaa kompaktoidun transduktorin, mutta lisäksi muuttaa läpikäyntialgoritmia tavallisesta backtrack-algoritimiksi
(Schmid, 2007a). Komennolla -l tuloksena on lowmem-transduktori, joka on
myös kooltaan pieni, ja lisäksi sitä ei lueta käsiteltäessä muistiin vaan käytetään
mahdollisimman paljon dataa suoraan kovalevyltä.

Vastaavasti FSM-testi on tuotettu valikoimalla transduktorien muodoksi InputIndexed tai OutputIndexed käsin parhaan lopputuloksen takaamiseksi. SFST:n transduktoreista optimoituja versioita ei voi käyttää lainkaan muuhun kuin analyysiin.
Generointitestin komento SFST:llä on fst-generate kotus.sfsta
> /dev/null, ja korpusanalyysitestille fst-infl kotus.sfsta
karjalainen1992.wordlist > /dev/null,
missä
karjalainen1992.wordlist on karjalaisesta kappaleessa 4.1 kuvatulla tavalla irrotetusta sanalistasta 10000 ensimmäistä tietuetta13 . FSM:llä analyysitesti
on tehty farcompilestring-komennolla ja indexed-tyypillä, sekä kompositoimalla tulos farfilterillä, ja generointi vastaavasti.
\footnote{
13

Koko korpuksen analysointi transduktorilla, jonka käännössä ei ole käytetty asetusta -c tai -l
kestää n. 24 tuntia
}

\begin{figure}
    \caption{Järjestelmien suoritinajan ja muistin käytön vertailutaulukko}
    \begin{tabular}{c|l|l|l||l}
        \hline
        % \multicol{5}{c}{Suoritinaika}
        \hline
        \bf Järjestelmä→ & SFST & SFST & SFST & AT\&T \\
        \bf ↓Toiminto & & -c & -l & FSMlib \\
        \hline
        \bf Kääntäminen & 309 s & 309 s & 309 s & 125 s \\
        \hline
        \bf Generointi (kaikki) & 17 s & — & — & 9s \\
        \bf Analyysi (korpus) & 251 s & 0,3 s & 10 s & 1,3 s \\
        \hline
        %\multicol{5}{c}{Muistinkäyttö}
        \hline
        \bf Kääntäminen & 55 MiB & 55 MiB & 55 MiB & 83 MiB \\
        \hline
        \bf Generointi (kaikki) & 2,8 MiB & — & — & 1,4 MiB \\
        \bf Analyysi (korpus) & 2,8 MiB & 1,8 MiB & 1,1 MiB & 1,4 MiB \\
        \hline
    \end{tabular}
\end{figure}

\section{Keskustelu}

Vapaan ja avoimen morfologisen järjestelmän kehittäminen vapaista komponenteista lienee ensimmäisiä laajamittaisia suomen kielen kohdalla, vaikka aiempiakin kuvauksia muoto-opista on jo varhainkin (Koskenniemi, 1983). Muissa kielissä tällaisia järjestelmiä on toteutettu jonkin verran, esimerkiksi unkarin kohdalla
Trón et al. (2005). Tekemäni järjestelmä on uusi ja varhaisessa vaiheessa, joten
seuraavassa kappaleessa yritän kuvata kaikkea mahdollista jatkotutkimusta.

\subsection{Jatkotutkimuksesta}

Nykyinen järjestelmä generoi kieliopin ja sanalistan kuvauksen mukaisesti kaikista sanoista kaikki muodot allomorfeineen riippumatta siitä ovatko jotkin muodoista ja allomorfeista harvakäyttöisiä tai jopa kuolleita. Tällaisessa testaussovelluksessa se ei tietenkään ole ongelma, mutta jos morfologiaa käyttää käytännön
sovelluksissa asia lienee aihetta huomioida. Sama pätee myös järjestelmän vapaaseen yhdyssanamuodostukseen, joka tuottaa huomattavan määrän käyttämättömiä muodosteita, jotka tapauksittain häirinnevät varsinaista sovellusta. Esimerkiksi nominit utu ja ilta yhdyssanan jälkiosina tuottavat sellaisia sanoja, jotka peittävät olemassaolevia muodosteita verbijohtimelta UtU ja taivutusmuodolta monikon ablatiivi.

Vastaavasti nyt morfologisena analysaattorina järjestelmä pyrkii tunnistamaan ja
muodostamaan kaikista sanoista kaikki sanamuodot, joka on käytännöllistä silloin, kun pyritään mahdollisimman kattavasti saamaan tietoa analysoitavan datan
morfologisesta luonteesta, mutta muilla sovelluksilla voi olla muunlaisia tavoitteita. Esimerkiksi tekstin oikaisuluvussa voi olla hyödyllistä pitää tiettyjä sanamuotoja tai tunnusten allomorfeja harvinaisina tai epäsuotavina.

Ratkaisuksi tässä on mahdollista käyttää sana- ja muotokohtaisia rajoitteita, mutta nykyisillä järjestelmillä on mahdollista kehittää myös sellaisia äärellistilaisia
transduktoreja, joihin on liitetty painoja tai todennäköisyyksiä. Eräs tärkeä jatkotutkimuksen aihe lienee, jos tällaista järjestelmää rakentaa, todennäköisyyksien ja
painojen perusteiden laskeminen.

Toteutettu järjestelmä generoi sanamuodot käyttäen vahvasti hyväkseen Nykysuomen sanalistan taivutusluokitukseen sisältyvää tietoa, esimerkiksi astevaihtelu,
allomorfivalinta ja vartalovaihtelu ovat kaikki toteutettu hyödyntämällä taivutusluokkanumeroita ja astevaihtelukirjaimia eksplisiittisenä datana, vaikka suomen
kielen muoto-opin taivutuksesta valtaosa on fonologisesti motivoitunutta, ja siten
toteutettavissa myös vaikkapa käyttämällä sanan rakennetta näiden taivutusluokkien asemesta, joka saattaa olla sovelluksittain sopivampi ratkaisu. Toteutuksesta
ks. esim. (Koskenniemi, 1983).

Nykysuomen sanalistan sanoista käytin järjestelmässä vain reilusti alle puolia,
käyttämättä jäivät luokittelemattomat yhdyssanat, luokitellut yhdyssanat, adverbit ja pronominit. Näistä pronominien käsittely ja osiltaan adverbien vaatisi lähinnä pieniä lisäyksiä järjestelmään. Yhdyssanojen käsittely on kuitenkin vielä ratkaistava asia, samoin kuin adjektiivisuuden tunnistaminen ja adjektiivijohdosten
tekeminen. Nimitän tässä adjektiiveja johdoksiksi, sillä morfologisen järjestelmäni kannalta on niin, että komparatiivin ja superlatiivin — sekä tavallansa positiivin
— tunnukset ovat sellaisia, joiden tuotokset on lisättävä takaisin säännöstökäsittelyn alkukohtaan, jotta niistä saadaan kaikki tarvittavat taivutusmuodot ulos.

Vastaavasti on auki sanojen epäproduktiivisemman morfologian soveltaminen, eli
sellaisten johdostapojen päätteleminen, jotka eivät systemaattisesti käy edes teoriassa sataan prosenttiin morfofonologisista kannoista, vaan rajoittuvat esim. semanttisleksikaalisin perustein. Myös erisnimet vaativat omanlaisensa käsittelytavan, johon järjestelmässä ei ole otettu kantaa.

Nykyinen järjestelmä myös tuottaa jokaiselle sanalle kaikki teoreettisesti mahdolliset tulkinnat, joka on käytännöllistä joihinkin tarkoituksiin, mutta esimerkiksi jos järjestelmää käyttäisi kokonaisten tekstikorpusten morfologiseen merkkaamiseen tarvitsisi kehittää disambiguaatiomenetelmiä, joilla saisi vain lausekontekstissaan oikean tulkinnan aikaiseksi. Disambiguaatiokeinot ovat melko laajasti
tutkittu alue, esimerkiksi morfosyntaktisen disambiguaation osalta (Voutilainen,
1997; Karlsson et al., 1995) ja sitten semanttisen (Lindén, 2005).

Järjestelmän laadun ja suorituskyvyn evaluoinnin osalta tarkemmat ja kattavammat kuvaukset olisivat tarpeen (esim. Kanthak ja Ney (2004)). Tutkielmassa
esitetyt mittaustulokset ovat kuitenkin parhaimmillaankin yksittäisiä ja
suuntaa-antavia, suorituskykymittaukset on esimerkiksi suoritettu käyttöympäristössä jonka tasalaatuisuutta ei ole voinut taata. Korpustestit on vastaavasti suoritettu vain
yhdellä satunnaisesti valitulla korpuksella.

Sanalistan ylläpito ja laajentaminen ovat myös tarpeen. Kuten testaustuloksista
selvisi ei sanalistan kattavuus juoksevan tekstin yli kuitenkaan ole vielä kovin
suuri. Sanalistan ylläpitoa on tutkittu esimerkiksi pro gradu -tutkielmassa Seppälä
(2006).

Eräs puute mikä SFST:tä käyttäessä on huomattavissa on varsinaisen leksikkokäsittelimen puute. Siinä missä esim. tutkielmassa Koskenniemi (1983) ja kirjassa
Beesley ja Karttunen (2004) kuvatuissa järjestelyissä sanalistojen ja taivutusmuotojen lisäys hoituu erillisellä leksikkokäsittelyllä, toteuttamassani järjestelmässä
tämä vaihe on hoidettu äärellistilaisilla operaatioilla yhdistämällä yligeneroiva
konkatenaatio kompositoitavaan filtteriin.

\subsection{Nykysuomen sanalistan käytännöllisyydestä}

Sanalistassa käytetty XML-formaatti on tällä hetkellä yksinkertainen, mahdollisesti kertakäyttöinen datarakenne, joka ajaa hyvin asiansa. Jos kuitenkin formaattia ajattelisi pitkäaikaisena säilytysformaattina samankaltaisille sovelluksille, se
voisi hyötyä muutamista lisäyksistä, jotta se olisi itsenäisenä riittävä. Ensinnäkin käsiteltävien yhdyssanojen kannalta yhdyssanarajan merkintä, eritoten kaikista sanarajoista taipuvien yhdyssanojen osalta olisi helpottava lisä. Astevaihtelun luokituksessa jokainen luokka kuvaa tällä hetkellä vain tyypin. Kuvaamalla
myös suunnan olisi sanasta valmiiksi käsiteltävissä jo tarvittava tieto astevaihtelusta, vaikka toki jos suomen kielen morfofonologia oletetaan tunnetuksi jo pelkkä tieto astevaihtelullisuudesta tai -vaihteluttomuudesta riittää oikean luokan ja
suunnan arvaamiseksi. Sitaattilainojen taivutusluokan osalta oikea taivutus vaatisi tiedon foneettisesta asusta, joka puuttuu kokonaan, eikä ole juurikaan arvattavissa. Monikkosanoista olisi hyvä tietää että sanakirjamuotona on käytetty listassa monikkoa, jotta sitä ei tarvitsisi arvata. Myös sanan osat olisi hyödyllistä
merkata erikseen, vaikka ne sananalkuisista yhdysmerkeistä tunnistaakin. Sanalistassa ei myöskään ole varsinaisesti merkitty adjektiiveja erikseen, vaan vain osa
kuuluu tiettyihin taivutusluokkiin yksinomaisesti, mutta osa jakaa taivutusluokkansa substantiivien kanssa. Tämän perusteella komparaatiotaivutus morfologiatoteutuksesta puuttuu, vaikka sen tekeminen ei sinänsä monimutkainen ole.

Lisäksi osa sanoista on joko väärässä taivutusluokassa, tai taipuu muutoin odotuksenvastaisesti, joten pitää olla keino sekä sivuuttaa ne, että antaa korvaava vastine. Näitä sanoja ovat nähdäkseni ainakin numeraalit ja jotkin pronominit, nominit veri, meri, aika ja poika yhdyssanamuotoineen sekä olla-verbi. Yleistäen
voisi sanoa sanakirjan osalta, että kaikki sanat, joissa kielitoimiston sanakirjassa
on sana-artikkelin proosassa selostettu taivutuksen poikkeuksista, tulisi luokitella
omiin luokkiinsa sen sijaan että niitä kohdeltaisiin erityisinä.

Toinen morfologisen jäsentimen kannalta mahdollisesti kiinnostava kehitys taivutusluokkanumeroihin on niiden yhdisteleminen yleistysten pohjalta. Triviaalisti
on yhdistettävissä ainakin sellaiset kuvaukset, kuten taivutusluokat 24 ja 26, jotka
kuvaavat täsmälleen saman taivutuksen, jolla saattaa olla yhdessä allomorfiparissa
eri keskinäinen yleisyysjärjestys. Laajemmin, kuten kappaleessa 2 selvitettiin, oikeasti luokittelu kuvaa morfofonologisten piirteiden kombinaatioita, ja usein sellaisia, jotka selviävät sanamuotoja tarkastelemalla, joten siltä pohjalta yleistysten
tekeminen voisi johtaa tehokkaampaan järjestelmään.

Myöskin vaikka sanalista onkin laaja, ei se tietenkään käytössä olevan kielen tapauksessa voi olla kattava. Yksi kiinnostava ongelma lieneekin tämän järjestelmän
laajentaminen sellaisiin sanoihin joita listassa ei ole. Tekemästäni sanalistaluokitteluselvityksestä saattaa olla hyötyä uusien sanojen luokittelussa ja tuonnissa.
Lisäksi voisi laajentaa järjestelmää säännöllisestä taivutuksesta myös johto-opin
morfologian puolelle, millä saisi kiinni uusia ja tulevia ja muita väliaikaisia sanoja, jotka sanalistasta uupuvat.

Näitä XML-muodossa arvioimiani puutteita olen testinomaisesti koettanut järjestelmässäni korjatakin. Monikkosanoja varten lisäsin XML-rakenteeseen attribuutin sanakirjassa olevan sanan muodolle (s@muoto), joka monikkosanoilla on
monikon nominatiivi (PL NOM), ja attribuutin rekonstruoidulle perusmuodolle
(s@perusmuoto), joka kuvaa sanasta sellaisen muodon, mikä olisi sanakirjan
tyypillisen perusmuodon mukainen muoto eli tässä yksikön nominatiivi. Monissa
tapauksissahan monikkosanan yksikkömuodot esiintyvät vain yhdyssanan osina
tai eivät esiinny kielessä laisinkaan, mutta tässä tapauksessa järjestelmä voi silti
käyttää rekonstruoitua yksikköä taivutuksen lähtökohtana, ja poistaa yksikkömuodot lopuksi.

Poikkeuksellisesti taipuvien sitaattilainojen taivutusta varten kokeilin attribuuttia
t@vartalovokaali, johon tulee sisällöksi sanan ääntöasun mukainen vartalovokaali silloin kun se poikkeaa kirjoitusasun mukaisesta, eli luokan 22 sanoilla
aina (esimerkiksi show’lla u ja parfait’lla e).

\subsection{SFST:llä toteutetun suomen kielen morfologian suorituskyky}

SFST:n suorituskyvyn tarkastelemiseksi suoritin joitain alkeellisia mittauksia,
joiden tulokset on kuvattu kappaleessa 4.2. Lopulta SFST-järjestelmä kääntyy hitaammin kuin FSM-järjestelmä, mutta suorituskyky valmiilla transduktorilla
on hyvä jos käyttää oikeita optimointiasetuksia. Erityisesti on huomattava, että
compact-tyypin transduktori analysoi dataa eri algoritmilla kuin muut, ja on useita kertaluokkia nopeampi.
Suorituskyvyn tarkempi evaluointi ja parantaminen vaatii vielä systemaattisempaa
ja järjestellympää tarkastelua ja tutkimusta.

\subsection{SFST:lla toteutetun suomen kielen morfologian laadusta}

SFST-järjestelmän laadullista tasoa mittasin kappaleessa 4.1. Järjestelmä toimi n.
97 \% saannilla — verrattuna aiempaan morfosyntaktisesti jäsentävään ja disambiguoivaan järjestelmään — niillä sanoilla, jotka sen oli tarkoitus tuntea. Jäljelle
jäävän 3 \%:n jakauma on kuvattu taulukossa 9. Taulukossa on järjestetty eroavat analyysit yleisyyden mukaiseen järjestykseen. Merkittävä osa, liki kolmannes
on sellaisia on-verbin analyyseja, joita järjestelmäni ei tee laisinkaan, eli passiiviksi merkittyjä muotoja, jotka näyttävät yksikön kolmannen muodoilta. Tämän
kuvittelen johtuvan siitä, että vertailujärjestelmä käyttää syntaktista jäsennintä,
joka poimii passiivisen tulkinnan muualta. Toiseksi eniten on superlatiiveja, jotka järjestelmästäni oli jätetty tarkoituksellisesti pois, koska kuten aiemmin kuvataan, sanalista ei erottele adjektiiveja substantiiveista. Komparatiivien pienempi
lukumäärä johtuu siitä, että mitatussa järjestelmässä oli alkeellinen komparatiivin
muodostus lisäämällä mpi yksikkövartaloon. 17 \% oli MA-infinitiiviksi merkittyjä
agenttipartisiippeja, eli sellaisiksi merkittyjä MA-infinitiivin muotoja, joita MAinfinitiivi ei saa, kuten partitiivia tai genetiiviakkusatiivia (Hakulinen et al., 2004).
Kieltoverbiä ei ei ole sanalistassa, joten sen muodostus on järjestelmässä osin vajaata, osin sen virhetulkinnat ovat samanlaisia passiiveiksi merkittyjä persoonataivutettuja muotoja kuin olla-sanalla. Sanat aika ja poika on Nykysuomen sanalistassa merkitty säännölliseen taivutukseen, vaikka niillä on säännöistä poikkeava i
: j -vaihtelu, jota en ole korjannut. Korpuksessa on myös merkittävä osa johdoksia,
joiden kantasana on arvattu aika kaukaa, mitä järjestelmäni ei tee. Näistä
merkittävät 3 \% on kieltopartisiipiksi nykyään nimettyyn mAtOn-johdostyyppiin, eli mainfinitiivin karitiiviin, kuuluvia. Tarkistamattomia monikkosanoja, jotka sanalistasta löytyvät, mutta joiden monikkous on käsin tarkastamatta, on jopa vajaa prosentti virheistä. Minen-johdoksen tulkintavirheistä kaikki kuuluvat mis-loppuisiin
yhdyssanamuotoihin, joita jostain syystä on korpuksesta löytynyt itsenäisinä. Lopuissa noin kolmessa prosentissa virhetulkintoja lähinnä yksittäisiä virhetyyppejä,
esimerkiksi kirjoitus- tai ajatusvirheellisiä muotoja, tyyppiä oikeaseen illatiivina,
tai väärää vokaalisointua, tyyppiä analyyseissa. Paljon on myös monikoita, jotka
on merkitty yksiköiksi, kuten alueiden.

\begin{figure}
    \caption{Järjestelmäni ja korpusdatan analyysien erot Karjalainen 1992
    -korpuksessa}
    \begin{tabular}{l|r|r}
        \hline
        \bf Eron tyyppi & \bf N & \bf \% \\
        \hline
        Olla-passiivit & 19 056 & 29,3 \% \\
        \hline
        Superlatiivit & 14 103 & 21,7 \% \\
        \hline
        IIIinfiksi merkitty partisiippi & 10945 & 16,9 \% \\
        \hline
        Ei-muodot & 6 728 & 10,4 \% \\
        \hline
        Komparatiivit & 3 024 & 4,7 \% \\
        \hline
        Muut olla-muodot & 2 770 & 4,3 \% \\
        \hline
        Aika & 2629 & 4.1 \% \\
        \hline
        IIIinfiksi merkitty karitiivi & 1 794 & 2,8 \% \\
        \hline
        Monikkosanat & 600 & 0,9 \% \\
        \hline
        Poika & 501 & 0,8 \% \\
        \hline
        Pass 3P & 185 & 0,3 \% \\
        \hline
        Minen-johdokset & 169 & 0,3 \% \\
        \hline
        Mon.gen -in & 159 & 0,2 \% \\
        \hline
        Kons.vartalot & 99 & 0,1 \% \\
        \hline
        Iparticiksi merkitty johdos & 86 & 0,1 \% \\
        \hline
        \hline
        Muita & 2 076 & 3,2 \% \\
        \hline
        Yhteensä & 64922 & 100 \% \\
        \hline
    \end{tabular}
\end{figure}

Järjestelmäni kattavuudeksi sain 75 \%, tämä tarkoittaa, että korpuksen kaikista
saneista, jotka korpusta tehdessä käytetty järjestelmä tunnisti, minun järjestelmäni tunnisti oikein kolme neljännestä. Jäljelle jäävään neljännekseen kuuluu edellä
mainitun 3 \%:n lisäksi yhdyssanoja, nimiä ja muita vierassanoja. Näistä olen poiminut 1000 sanan otoksesta esimerkiksi taulukon 10. Taulukossa yhdyssanoiksi
on merkitty kaikki sellaiset sanojen sanalistassa olevien perusmuotojen yhdistelmät, jotka ovat jääneet tunnistumatta, joissa ei ole sellaista johdosainesta, jota
järjestelmäni ei käsitellyt. Vastaavasti johdoksiksi on merkitty paitsi yksisanaiset
johdostyypit, joita ei ollut sanalistassa, kuten UUs-loppuiset ominaisuuden johdokset, myös yhdyssanat, joiden osassa on esimerkiksi jA-johtiminen tekijänjohdos. Tästä esimerkiksi palkinnonsaaja on järjestelmälle vieras johdos, koska saaja
ei ole sanakirjasana vaan johdos mutta taantuma-aika on yhdyssana, koska taantua on sanakirjassa ja taantuma on järjestelmän käsittämä säännöllinen taivutusmuoto (so. agenttipartisiippi ja ma-infinitiivi). Kirjoitusvirheet sisältävät sellaisia
tunnistumattomia sanamuotoja, jotka voisi lauseyhteyden perusteella päätellä selväksi kirjoitusvirheeksi ja joiden osoittamaa sanaa ei löydy sanakirjasta. Etuliitteisiin kuuluvat sanat, joissa alkuosa on tyyppiä ylä-, ala-, etu-, läpi- jne.

\begin{figure}
    \caption{Järjestelmäni Karjalainen 1992 -korpuksen puuttuvat sanatyypit (1000:n
    otoksesta)}
    \begin{tabular}{l|r|r}
    \hline
        \bf Puutteen tyyppi & N & \% \\
\hline
\hline
\bf Yhdyssanat & 858 & 85,8~\% \\
\hline
        \bf Johdokset & 70 & 7,0~\% \\
\hline
        \bf Puuttuvat & 31 & 3,1~\% \\
\hline
\bf Etuliitteelliset & 27 & 2,7~\% \\
\hline
\bf Tulkintavirheelliset & 5 & 0,5~\% \\
\hline
\bf Komparatiivit & 3 & 0,3~\% \\
\hline
\bf Kirjoitusvirhe & 3 & 0,3~\% \\
\hline
\bf Superlatiivit & 2 & 0,2~\% \\
\hline
\bf Monisanaiset & 1 & 0,1~\% \\
\hline
\hline
\bf Yhteensä & 1000 & 100~\% \\
\hline
    \end{tabular}
\end{figure}

        Järjestelmän tarkkuus oli 50 \%, mikä tarkoittaa käytännössä sitä, että keskimäärin yhtä sanetta vastaisi yksi oikea ja yksi väärä analyysi. Suurelta osin tulos selittyy jo sillä, että suomen kielessä esiintyy monimerkityksisyyttä morfofonologisten analyysien suhteen siinä, että yhtä sanetta voi vastata usean sanan usea
taivutusmuoto. Lisäksi on joitain järjestelmässäni olevia systemaattisia monitulkintaisuuksia, jotka pienentävät tarkkuutta, kun verrataan disambiguoituihin tuloksiin. Esimerkiksi nominin n-sijasta luodaan lähes poikkeuksetta genetiivin ja
akkusatiivin tulkinta, samoin verbien partisiippimuodoista tulee aina sekä verbiä
että partisiippia vastaavan sanan sanamuodon tulkinnat.

\section{Yhteenveto}

Tutkielmassani olen esitellyt suomen kielen äärellistilaisen automaattisen jäsentimen kehitystä ja testausta. Tutkielman lähtäkohtana oli rakentaa avointa ja vapaata
lähdekoodia oleva kokonaisjärjestelmä. Tähän tarkoitukseen sopiviksi resursseiksi valittiin SFST-niminen äärellistilaisten transduktorien ohjelmointijärjestelmä,
ja sanalähteeksi Kotimaisten kielten tutkimuskeskuksen julkaisema Nykysuomen
sanalista.

Nykysuomen sanalista on taivutustyypeittäin luokiteltu 78 eri luokkaan, joita jäsentimessä on käytetty suomen kielen taivutuksen toteuttamiseen. Taivutusluokittelun käyttäminen jäsentimen lähtökohtana onnistui toteuttaa hyvin, ja siltä pohjalta muodostetun järjestelmä korpustestien perusteella taivutti oikein ne sanat
mitkä sen tulikin osata, eli sanalistan luokitellut sanat. Systemaattisia puutteita sanalistassa oli yhdyssanamerkintöjen kanssa, sekä sanaluokkajaossa. Luokittelussa
käytetyt sanaluokat olivat vain verbejä ja nomineja, joten esimerkiksi adjektiivitaivutuksen toteutus on lopullisessa järjestelmässä vajaa.

Muita nykysuomen sanalistan taivutusluokitteluun liittyviä ongelmia olivat satunnaiset poikkeuksellisuudet osassa sanoista. Esimerkiksi sanojen meri ja veri epäsäännöllinen partitiivin vokaalisointu tai sanan olla epäsäännölliset taivutusmuodot vaatisivat sanalistaankin merkintänsä.

Kokonaisuutena siis sanalistan ja sen XML-formaatin osalta morfrologisen järjestelmän kannalta hyödyllistä olisi ainakin lisätä tiedot sanaluokista. Mahdollisesti
järkevää olisi myös järjestellä erikoistieto muusta poikkeustaivutuksesta, vaikka
toisaalta se olisi mahdollista sisällyttää myös sanaluokitusta muokkaamalla.

Kokonaisen suomen kielen kattavan jäsentimen toteuttamiseen SFST:stä löytyvä
yksittäisten replace-sääntöjen kääntömahdollisuus myös riitti ja niiden käyttö sarjallistetusti sanalistatransduktoriin kompositoiden on suorituskyvyltään riittävän
nopea. Tietenkin tulevaisuuden kannalta esimerkiksi mahdollisuus säännöstöjen
hallintaan voi olla hyvä asia.

Korpusten osalta tilanne jäi tutkielman kannalta melko vajaaksi, sillä vapaasti ja
avoimesti hyödynnettäviä testaamiseen soveltuvia korpuksia ei ollut saatavilla.
Rajoitetuin ehdoin saatavilla olevista korpuksistakin kuitenkin havaittiin, että laajahko Nykysuomen sanalistakaan ei sinänsä riitä kattamaan juoksevasta tekstistä
kovin suurta osaa uniikeista sanoista, vaikka sivuutettaisiinkin puutteet adjektiivitaivutuksessa ja yhdyssanamuodoissa. Sanaston ylläpitokin jää siis selvitettäväksi
kysymykseksi.

Kaikkiaan tutkielman tuloksena on kuvattu järjestelmä, joka kattaa hyvän osan
suomen kielen morfologisesta jäsentimestä, ja toiminee hyvänä pohjana tulevan
täydemmän suomen kielen jäsentimen kehitykselle.

\bibliography{gradu}  % viittaa tiedostoon gradu.bib


\appendix

\section{Äärellistilaisten sovellusten ominaisuuksien vertailu}

Tässä liitteessä on taulukoitu tutkielman kirjoitushetkellä löydettyjen äärellistilaisten ohjelmistojen ominaisuuksia, joiden perusteella tutkimuksessa käytetty ohjelmisto SFST on valittu, ja jonka perusteella jatkotutkimusta on mahdollista suunnitella. Taulukossa oleva data on peräisin sovellusten kotisivuilla annetuista tiedoista ja järjestelmiä kuvaavista artikkeleista. Taulukon lähtökohtana on käytetty
aineistoa Yli-Jyrä (2007).

Taulukossa lisenssit on jaettu luokkiin avoimuuden perusteella. GNU-lisenssit
GPL ja LGPL on merkitty erikseen. Muut avoimet lisenssit on merkitty taulukkoon vain merkinnällä avoin. Lisenssit, joiden ehdoilla järjestelmät ovat käytettävissä maksutta, mutta jotka rajoittavat käyttöä on merkitty suljetuiksi. Suuri osa
suljetuista lisensseistä rajoittaa käytön nimenomaan akateemiseen ja tutkimustarkoitukseen, nämä on erikseen merkitty tutkimus-merkinnällä. Joitakin sovelluksia
on myös markkinoilla täysin kaupallisina, siten, etteivät ne ole mitenkään vapaasti
käytettävissä. Nämä on merkitty kaupallinen-merkinnällä.

Replace-sääntösarakkeeseen, jota on käytetty toisena tärkeänä valintaperusteena, on merkitty tukeeko järjestelmä replace-sääntöjä millään tavoin. Esimerkiksi SFST tukee yksittäisen replace-säännön muuntamista transduktoriksi ja sen
kompositointia toiseen transduktoriin, mikä järjestelmän toteuttamiseen riitti, joten tätä tarkemmin en asiaa tarkastellut.

Kohtaan painollisuus on kirjoitettu tukeeko järjestelmä lainkaan painollisia
transduktoreja. Kohtaan ohjelmointikielet on merkitty ensisijainen ohjelmiston
ohjelmointiin käytetty kieli, joka todennäköisestikin on se, jota voi jatkokehityksessä käyttää. Jos ohjelmistoon liittyy muihin kieliin tehtyjä rajapintoja, näitä ei
ole otettu huomioon. Sekä painollisuus että ohjelmointikieli lienevät jatkotutkimuksen kannalta relevantteja.

\begin{table}
    \caption{Tarkasteltavat äärellistilaiset koodikannat}
    \bf TODO:
Ominaisuus→
↓Sovellus
AT\&T FSM \&
GRM library
lextools
SFST
ALE-RA
ASTL

Lisenssi

Replace- Painollisuus
säännöt
Kaupallinen Ei
On
Kaupallinen On
On
Kaupallinen Ei
—
GPL
On
Ei
(L)GPL
Ei
—
LGPL
Ei
On
(jatkuu seuraavalla sivulla)
57

Ohjelmointikieli
C++

C++
C++, Prolog
C++

Ominaisuus→
↓Sovellus
AX
Attias FS Tools
Carmel
DFKI FSM Toolkit
Edinburgh FSA
Fadd library
FIRE Engine
FIRE Lite
FIRE Station
FIRE Works
fskit
GFSM
GFSMT
Gdansk FSA
GRAIL
Groningen FSA
INTEX
Jacaranda
lintouch
MAP-3.1
MDP
MMORPH
Open FIRE
OpenFST
PC-KIMMO
kgen
REGI
Potsdam FSM tools
RuleCompile
SRILM
Skeema Parser
Speech Tools Library
Statechart
TULIP
MITFST
RWTH FSA
UCFSM

Lisenssi

Replace- Painollisuus
säännöt
Suljettu
—
—
Tutkimus
—
—
Tutkimus
—
—
Ei saatavilla —
Ei
GPL
—
—
Tutkimus
—
—
Tutkimus
—
—
Tutkimus
—
—
Tutkimus
—
—
Suljettu
—
—
Ei saatavilla —
—
LGPL
–
On
—
–
On
GPL
—
—
Tutkimus
Ei
Ei
GPL
–
On
Tutkimus
—
—
Ei saatavilla —
—
GPL
Ei
On
Kaupallinen —
—
—
—
—
GPL
On
Ei
Avoin
—
—
Avoin
On
On
Tutkimus
Ei
Ei
Ei
Ei
Avoin
—
—
GPL
Ei
On
Ei saatavilla —
—
Avoin
Ei
On
Ei saatavilla —
—
Avoin
Ei
On
—
—
—
Avoin
On
Ei
Avoin
Ei
–
Avoin
–
On
LGPL
Ei
On
(jatkuu seuraavalla sivulla)
58

Ohjelmointikieli
—
C++, Boost
—

C++
C++
C++
C++
—
C
LISP
C++
C++
Prolog
C
Java
C
CLISP
Matlab
C
C++
C
C
Prolog
C++, boost
C
C++
—
C
C++, boost
Prolog
C++
C++
C++

Ominaisuus→
↓Sovellus
UNITEX
Vaucanson
XFST
LEXC
XFST2FSA

Lisenssi
LGPL
GPL
Kaupallinen
Kaupallinen
Vapaa

Replacesäännöt
—
—
Ei
Ei
Ei

Painollisuus Ohjelmointikieli
On
Java, C, C++
On
C++
On
—
—
—
C
\end{table}

\section{Äärellistilaisten sovellusten syntaksin ja ohjeiden vertailu}

Taulukko selitetty ohjeessa Eisner (2007) eli sivulla http://www.cs.jhu.
edu/~jason/405/software.html — SFST-sarake minun, suomennokset
minun. Taulukon muodon kuvaus löytyy myös viitatulta sivulta, mutta lyhyesti
siinä on kerrottu jokaisen toteutuksen saatavilla olevista ohjeista ja syötteen syntaksista.

\begin{table}
    \bf TODO:
\end{table}


\end{document}
