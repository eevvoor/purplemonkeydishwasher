\documentclass[final,hyperref={pdfpagelabels}]{beamer}

\usepackage{fontspec}
\usepackage[absolute,overlay]{textpos}
\usepackage{amsmath,amsthm, amssymb, latexsym}
\boldmath%

\usepackage{array,booktabs,tabularx}

\usepackage[orientation=portrait,size=a0,scale=1.4]{beamerposter}
\mode<presentation>
{%
    \usetheme{HZSK}
}
\setbeamertemplate{bibliography item}[triangle]

\title[apertium-fin-eng]{{\huge Shared-Task Driven RBMT development}\\
apertium-fin-eng experiments at WMT 2015--2019
\url{https://github.com/apertium/apertium-fin-eng}}
\author[flammie@iki.fi]{Flammie$^\star$ }
\institute[Uni. Hamburg]{$^\star$ Universität Hamburg}
\date{\today}

%\logo{\includegraphics[height=7.5cm]{AbumatranLogo}}

\newlength{\columnheight}
\setlength{\columnheight}{105cm}


\begin{document}

\begin{frame}
      %\maketitle
      %\vfill
      \begin{columns}
      \begin{column}{.49\textwidth}
      \begin{beamercolorbox}[center,wd=\textwidth]{postercolumn}
          \begin{minipage}[T]{.95\textwidth}  % tweaks the width, makes a new \textwidth
          \parbox[t][\columnheight]{\textwidth}{% must be some better way to set t
        \begin{block}{What is this?}
            \begin{itemize}
                \item Rule-based, shallow transfer, machine translation system
                    for Finnish---English using Apertium
                \item A dictionary and rule development \emph{workflow} for RBMT
                \item Probably not the best RBMT for fin---eng
            \end{itemize}
        \end{block}


        \begin{block}{Why? Why not?}
            \begin{itemize}
                \item Advantages of rule-based systems:
                    \begin{itemize}
                        \item Such advanced lexicography is \textit{fun}, like
                            collecting Pokémon or stamps!
                            \includegraphics{pikachu}
                        \item Working on the rules is also \textit{fun},
                            like using your brain to understand linguistics
                            and all!
                        \item Does not destroy our planet at an alarming pace!
                            (c.f.~\cite{strubell2019energy} few days earlier in
                            this ACL 2019~;-)
                            \includegraphics{mario3sun}
                        \item Does not need to leak your data through internet
                            and to big companies!
                        \item Predictable errors, easy fixes!
                    \end{itemize}
                \item Some disadvantages of Shallow RBMT here:
                    \begin{itemize}
                        \item \textcolor{blue}{BLEU} score always low
                        \item Shallow transfer RBMT not particularly suitable
                            for Finnish---English; needs more depth
                        \item Finnish---English is a very well resourced
                            language pair indeed, so simply SMT and NMT makes
                            a lot of sense
                    \end{itemize}
                \item Therefore:
                    \begin{itemize}
                        \item I mainly use work on fin---eng as an additional
                            data point on measuring workflow efficiency
                        \item come to Dublin for MTsummit / LoResMT workshop to
                            see another data point with interestinger
                            languages~\cite{pirinen2019workflows} (no English!)
                    \end{itemize}
            \end{itemize}
        \end{block}

        \begin{block}{How do you do it?}
            \begin{itemize}
                \item It's a workflow, based on shared tasks, so
                    \textit{Shared Task Driven} development
                \item Do this:
                \begin{enumerate}
                    \item Collect all lexemes unknown to source language
                        dictionary, and add them with necessary morpholexical
                        information
                    \item Collect all lexemes unknown to bilingual translation
                        dictionary, and add their translations
                    \item Collect all lexemes unknown to the target language
                        dictionary, and add them to the dictionary with
                        necessary morpholexical information
                \end{enumerate}
            \item repeat until all words from shared task are in all
                dictionaries
            \item Can be semi-automated to great extent
            \item Grammar rules, on the other hand, are mostly manual, expert
                labour still
            \end{itemize}
        \end{block}
        \begin{block}{References}
            \small
            \bibliography{acl2019}
            \bibliographystyle{alpha}
        \end{block}
         }
        \end{minipage}
      \end{beamercolorbox}
  \end{column}
    %\end{textblock}

    %\begin{textblock}{60}{0.1,1.6}
   \begin{column}{.49\textwidth}
      \begin{beamercolorbox}[center,wd=\textwidth]{postercolumn}
        \begin{minipage}[T]{.95\textwidth} % tweaks the width, makes a new \textwidth
          \parbox[t][\columnheight]{\textwidth}{% must be some better way to set the the height, width and textwidth simultaneously

        \begin{block}{Figure}
            \includegraphics{diagrammi}
        \end{block}

        \begin{block}{Results}

\begin{table}
\begin{center}
    \begin{tabular}{lr}
        \toprule
        Error & count \\
        \midrule
        OOVs in Finnish & 763 \\
        OOVs in English & 943 \\
        OOVs in Fin↔Eng & 2696 \\
        \bottomrule
    \end{tabular}
    \caption{Classification of mainly lexical errors in apertium-fin-eng
    submissions for 2019\label{table:errors}}
\end{center}
\end{table}


\begin{table}
\begin{center}
    \begin{tabular}{lr}
        \toprule
        \bf Corpus & BLEU-cased  \\
        \midrule
        apertium-eng-fin 2015 & 2.9 \\
        \hfill 2017 & 3.5  \\
        \hfill 2019 & 4.3  \\
        \midrule
        apertium-fin-eng 2015 & 6.9\\
        \hfill 2017 & 6.3 \\
        \hfill 2019 & 7.6 \\
        \bottomrule
    \end{tabular}
    \caption{Progress of apertium-fin-eng over the years using only the WMT
    shared task driven development method.\label{table:progress}}
\end{center}
\end{table}

        \end{block}

        \begin{block}{Some cherry-picked examples for fun and entertainment}
            \includegraphics{cherries} \\
            \begin{tabular}{lr}
                \toprule
                Source & Aika nopeasti saatiin hommat sovittua, Kouki sanoi \\
                \midrule
                Apertium-fin-eng & Kinda swiftly let jobs agreed, Kouki said. \\
                an NMT at WMT2019 & Pretty quickly we got the
                \textcolor{red}{gays} agreed, Kouki said.\\
                \midrule
                Reference & We reached a pretty quick agreement, Kouki said.\\
                \bottomrule
            \end{tabular}
            \small
            \begin{tabular}{lr}
                \toprule
                Source & Natural disasters make logistics even more complicated.
                \\
                \midrule
                Apertium-eng-fin & Luontevat tuhot malli- logistiikka vielä
                enemmän sekava.  \\
                (my re-translation) & Natural destruction model- logistics even
                more confusing. \\
                an NMT at WMT 2019 & Luonnonkatastrofit tekevät
                \textcolor{red}{saasteista} entistä monimutkaisempia.\\
                (my re-translation) &  Natural disasters make
                \textcolor{red}{pollution} even more complicated \\
                \midrule
                Reference & Luonnonkatastrofit tekevät
                logistiikasta vieläkin monimutkaisempaa. \\
                \bottomrule
            \end{tabular}
        \end{block}

        \begin{block}{Acknowledgments}
            The author was employed by CLARIN-D during WMT 2019. The free/libre
            open source RBMT systems have been made possible by contributors of
            omorfi, apertium-fin, apertium-eng and apertium-fin-eng packages.

            Some pictures in this poster are taken from Nintendo-based memes and
            rights to the characters and pixel art are owned by Nintendo (and/or
            related companies).
            \\
            \includegraphics{apertium}
            \includegraphics{flammie}
            \includegraphics[width=0.2\textwidth]{qrcode}
        \end{block}
    %\end{textblock}
         }
          % ---------------------------------------------------------%
          % end the column
        \end{minipage}
      \end{beamercolorbox}
    \end{column}
  \end{columns}

  \end{frame}
\end{document}
